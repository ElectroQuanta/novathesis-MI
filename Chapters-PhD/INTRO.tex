%!TEX root = ../template.tex
%%%%%%%%%%%%%%%%%%%%%%%%%%%%%%%%%%%%%%%%%%%%%%%%%%%%%%%%%%%%%%%%%%%
%% chapter1.tex
%% NOVA thesis document file
%%
%% Chapter with introduction
%%%%%%%%%%%%%%%%%%%%%%%%%%%%%%%%%%%%%%%%%%%%%%%%%%%%%%%%%%%%%%%%%%%

\typeout{NT FILE INTRO.tex}%
%\chapter{Introduction}
%Hello
%
%%\prependtographicspath{{Chapters/Figures/Covers/}}
%\definecolor{Schlgray}{gray}{0.95}

% CHAPTER - Introduction -------------------------
\chapter{Introduction}%
\label{cha:introduction}
Functional design is a highly desirable approach to design, optimising the
functional performance of a product while minimising the resources usage and the
cost.

However, functional design requires a shift in the manufacturing design
paradigm from being process-centered to be function-centered, i.e., product
design should focus on the functionality and overall properties of the
manufactured component, instead of the technologies required to achieve this.

Furthermore, the adoption of functional design may dictate the use of multiple
materials or a combination of them, which is hindered by the current
manufacturing methodologies. Conventional manufacturing technologies are mono-material and
subtractive, starting from a pre-shape, incurring in wasted material and energy. 

Thus, if functional design is the aspiring concept, then additive manufacturing
is the vehicle to bring it to life. By adding material only where it is
functionally required, the components' properties can be tailored for optimal
performance while minimising materials and energy. Intricates geometries can be
achieved, which would be otherwise unfeasible by conventional manufacturing
technologies.

Nonetheless, the multi-material additive manufacturing is still in its infancy,
especially when using metallic and ceramic materials. The current panorama shows that only a handful of commercial equipments can
be used for this purpose~\cite{aconityMIDI, slm280}. However, they require a significant initial
investment and are only suited for metals, which limits the scope of applications.

On the other hand, functionally design components are are highly attractive to top-shelf industries like
the aeronautical, aerospace and biomedical ones, as they address specific
problems in these fields. 
A paradigmatic example in the
biomedical field is the hip implant, whose goal is to mimic the natural
behaviour of the bone. As such and not surprisingly, the material composition,
properties and structure needs to be varied, as the bone is a great example of
an \gls{fgm}'s component.

However, the design and fabrication of these implants are still far
from the desired behaviour, as can be proved from the number of forced
retreament surgeries spanning a short period of time (ten to fifteen years after
the implantation surgery)~\cite{soliman2022review}.
An explanation can be given by the fact that
the multi-material processing of metals and ceramics needed, as well as the
respective design methodologies, are still in a premature stage.
% \ifdef{\comm}{\mbox{}\marginpar{Goal}}

\gls{lpbf}, a standardization of all laser powder fusion processes, like
\gls{sls} and \gls{slm}, is the most promising technology for multi-material
fabrication using metallic and composite powders. However, it is extremely complex, with hundreds of parameters~\cite{schneck2021review,
  thompson2015overview}, even in the mono-material scenario.
The problem becomes even worst when addressing the controlled combination of
multiple metals or ceramics, as this requires some special form of the material
(e.g powder) and a great amount of energy to produce it (e.g.~metals require at
least semi-pasty characteristics to be morphed). In order to successfully
produce multi-material components from metals, ceramics, or both, a specific
combination of controlled deposition of mass and controlled supply of energy
must be achieved, while still managing the interactions with the environment
(oxidation, heat transfer, etc.) and among materials (e.g.~delamination). This
is clearly a multi-physics problem and, as the timing and order of material
addition and energy supply is crucial, a topological problem is added up to the
stack, indicating why the progress in this field is still fairly limited.

As a result, the research field primarily focuses on upgrading mono-material
equipment to add multi-material capabilities~\cite{bareth2022Implem, schneck2022capability,
  nadimpali2019MMSteels, anstaett2017fabrication, sing2015interfacial,
  liu2014interfacial},  which undesirably also requires manual workarounds to
setup the manufacturing chain and equipments.
This is further aggravated by the lack of specific design guidelines for the 
\gls{mmlpbf} process and by the fact that most pre-processing procedures use a
manual method based on an inadequate data format~\cite{schneck2021review}.
%
The main takeway is that the \gls{mmlpbf} fabrication is a multi-objective
problem, requiring specially designed equipments and toolchains, and
experimental design.

Thus, the present work aims to close the gap between design and fabrication of
multi-material components like the aforementioned implants by proposing a holistic
approach to the multi-material fabrication of metals/ceramics that can leverage
the process knowledge and support functional design. An appropriate workflow is
required to clearly guide end-users and an accompanying toolchain capable of
handling the intricacies of the \gls{mmlpbf} process. Furthermore, a custom
equipment is also required to enable the combination of metals and ceramics in
the same component, which is not yet addressed by the commercial field. This
implies the usage of different types of lasers integrated into a single
equipment, which can be further used for procedures such as preheat treatments
to alleviate mechanical tensions or to promote better bonding of the materials.

Lastly, and often the most overlooked aspect in any emergent technology, is the
process knowledge potential. \gls{ai} tools, like deep learning or machine
learning, can provide critical insights for the process evolution, mining and
leveraging the \gls{mmlpbf} process data.
A paradigmatic example was reported last year, when Exponential Technologies, in
a collaboration with Aerosint, announced a two-fold improvement in component
density in a multi-metallic component, within 2 print jobs and 46 samples
printed without any prior statistical knowledge, saving time and money~\cite{xtSAAM2022AI}.

%\ifdef{\comm}{\mbox{}\marginpar{multi-material processing of metals and
%ceramics problem}}

%\ifdef{\comm}{\mbox{}\marginpar{Designer's perspective: The need for an easier path}}


\section{Motivation}
The functional design of components is a complex topic, with a myriad of
questions to be answered: what is the function of the component?; what design
criteria must be met to fulfil its function?; how will the component be
produced, and what data does it require?; how will the component's performance
be measured?, among others. The answers are often not clear or simple as they
dictate the use of several materials and several manufacturing technologies,
increasing severely the complexity of producing such components: how to
effectively combine two or more materials into a single component in a
synergistic way?

From the designer's perspective it becomes even worst as it must be intimately
acquainted with the manufacturing process and apply corrective factors to the
design which still might fail due to the complexity of the process, limiting its
creativity and the ability to produce functional parts as desired. A good
analogy to the multi-material product designer would be of an early programmer:
the latter needed to know in full-depth all the intricacies of the underlying
hardware and the moment of its ``liberation'' was the emergence of the so-called
high-level languages; for the former no such liberation is available yet, but
will certainly boost the functional design and manufacturing.

The possibility of controlling composition or structure and thus obtain
components with desired local properties, as regarding mechanical, tribological,
thermal properties, and others are of great interest, as material is only added
where it functionally needed, minimising waste and enhancing the overall
properties of the component being built. 
This idea meets its pinnacle with the concept of an \gls{mmfgm} --- multi-material components with materials gradations in between. High value products
would benefit tremendously from this opportunity, namely biomedical implants,
like the hip implant, where the outer layers contact directly the living tissue,
designing it for osseointegration, but the inner layers act merely as a
supporting structure.

The multi-material component's fabrication from metals and/or ceramics is still
in its infancy, due to the lack of an unsupporting structure that tackles its
high complexity, while guiding the end-user. Thus, the present work aims to
provide such a supporting structure, building the necessary toolchain, and providing as
effective means to materialise the design, by building a proof-of-concept
equipment.

\section{Main objectives}
The goal of the present work is to close the gap between design and fabrication
of multi-material components from metallic/ceramic materials using
the \gls{lpbf} technology.
To this end several main objectives have been outlined:
\begin{enumerate}
  \item Investigate the current panorama of multi-material fabrication of
    metal/ceramics, with special focus on the \gls{lpbf} technology, and
    identify its main gaps and shortcomings.
  \item Develop a design methodology for multi-material fabrication of
    metals/ceramics via the \gls{lpbf} technology, using a holistic approach
    to support functional design and leverage the process knowledge throughout
    the whole manufacturing chain.
  \item Instantiate a practical workflow from the design methodology, taking
    into account the project constraints and restrictions;
  \item Develop an integrated toolchain for the \gls{mmlpbf} process, supporting
    all stages of the manufacturing chain.
  \item Develop and build a proof-of-concept's equipment for the fabrication of
    multi-material metallic/ceramic components.
  \item Test the production of such components using the proposed
    workflow/toolchain and the equipment built;
  \item Develop a process knowledge database that can be used for process
    improvement, using tools such as \gls{doe} and \gls{ai};
  \item Investigate and incorporate mechanisms for process optimisation, namely:
    \begin{itemize}
      \item Design strategy optimization;
      \item Manufacturing process optimization: parameters, manufacturing paths,
        etc;
      \item \gls{3dmmlpbf} machine optimization; 
      \item Data flow optimization: leverage the knowledge acquired through the
        process to cascade `naturally' and efficiently to the
        \gls{mmlpbf} process's optimization;
    \end{itemize}
\end{enumerate}

The main focus of this work is to provide a sustainable development path for
3D multi-material design and fabrication of metallic and ceramics components.
The author asserts that this is the most effective way to address the high
complexity inherent to this process. Accordingly, the development of a machine
capable of producing multi-material metallic/ceramics components is viewed as a
means to achieve the overall objective, which is functional design of
multi-material components.

The emergence of 3D printing technology provides a paradigmatic example of such
a transformation. The concept was first introduced in 1972, when
Ciraud~\cite{ciraud1972process} proposed a powder process capable of
constructing three-dimensional components from a variety of semi-meltable
materials. Although Stratasys started to commercialize \gls{fff} printers in the
early 1990s, the real boom only occurred in the mid-2000s when RepRap started to
release open-source 3D printers.
This enabled each user to produce 3D components, but also, and more importantly,
to customize, replicate, and enhance their machines. This revolutionized 3D printing, generating multiple projects and emergent companies, which eventually led to the shutdown of RepRap's business.

The legacy and lesson that RepRap taught us, simillar to Prometheus\footnote{Titan who stole the fire
from the Olympian gods to give it to mankind, condemned for eternity by Zeus to
be tied to a rock and have its liver eaten}, was that knowledge is the most important and
expensive asset in any technology, and when knowledge is openly divulged, and
every user can effectively make use of it, the hardware and software costs drop
massively, and the process evolves greatly from users' tinkering and experience.

The same fundamental principle underlies the philosophy of this project. By
developing an ecosystem for multi-material functional design, knowledge from all
agents can leverage the process, mimicking natural processes and boosting its
evolution. Therefore, knowledge needs to be captured, tracked, and fed back
adequately and efficiently to the process by means of a global picture, a design
methodology that is applied by instantiating a practical workflow and brought to
life by a cost-effective equipment that can fabricate multi-material components
embodying this knowledge. A viable technology and a suitable cost-effective
manufacturing equipment can be achieved as side effects of this global
perspective, serving the greater good of functional design. This can open new
prospects in the research field, providing a bootstrapping environment for
the \gls{3dmmlpbf} process.

\section{Thesis organisation}
This thesis is organised as follows.
Chapter~\ref{ch:state-art} provides a comprehensive review of the current state of the art of
multi-material fabrication for metals and ceramics. The functional design
approach is introduced, followed by a discussion on the use of laser-based
additive manufacturing processes as a viable solution for metallic and composite
manufacturing, with particular attention given to the \glsxtrfull{lpbf} process. To
fully support functional design, the \glsxtrfull{mmlpbf} process is introduced,
which bridges the gap between \gls{lpbf} processes. An overview of \gls{mmlpbf}
is presented, including the current panorama, manufacturing chain, challenges,
and potential solutions. Finally, the specific applications envisioned in this work are listed.

Chapter~\ref{ch:prob-challenge} presents the multi-material fabrication
problem and its challenges using the \gls{lpbf} technology. A methodology
devised for multi-material production via \gls{lpbf} technology is introduced to tackle the high complexity of the process
and the lack of a supporting methodology, taking into account the key agents of
the process and leveraging the process information.

In Chapter~\ref{ch:development}, the knowledge acquired through the relevant models contained
in the proposed methodology is applied to the development of a specialised
workflow for the \gls{3dmmlpbf}'s process and respective toolchain,
contingent of the project's restrictions and resources, and to the development
of a machine capable of producing multi-material components from
metallic/ceramic powders matching the designed workflow. Based on this workflow,
the toolchain was assembled and the missing software components were developed,
tested, and validated. Lastly, a custom equipment for the production of
multi-material metallic and ceramic components was developed, tested and
validated across the mechanical, electronic, and control domains.

In Chapter~\ref{ch:application} the multi-material
fabrication of metallic and composites parts is tested, using one or multiple
lasers. The complete process, from inception to produced part, is tested as a
whole, to ensure the full validation of the designed \gls{3dmmlpbf} ecosystem,
i.e., methodology, workflow, toolchain, and equipment.
Lastly, the prospects for process improvement are outlined, leveraging the
process knowledge acquired for systematic and consistent evolution of the
\gls{3dmmlpbf} manufacturing chain.

The Chapter~\ref{ch:conclusion} gives a summary of this thesis as well as
prospect for future work.

Lastly, the appendices contain detailed information about the software \gls{api}, and the annexes contain the paper submitted, stemming out of this work.
