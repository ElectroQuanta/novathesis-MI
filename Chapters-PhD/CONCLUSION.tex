% !TEX root = ../template.tex
%
\typeout{NT FILE CONCLUSION.tex}%
	% CHAPTER - Conclusion/Future Work --------------
\chapter{Conclusion}%
\label{ch:conclusion}
In this chapter the conclusions and prospect for future work are outlined.

\section{Conclusions}
Functional design represents a shift in the product design paradigm, because the
focus is on overall components' functionality instead of on the manufacturing
process or materials to achieve it.
This means that material is only added where it is strictly required to perform
its function, optimizing resources, such as materials and energy, and frees the
designer to use its creativity and ingenuity.

On one hand, functional design may dictate the usage of several materials or a combination of them to fulfill its goal, which is hindered by the
current manufacturing methodologies.
On the other hand, functional design is very attractive for the design of high-value products, such
as the biomedical implants. As a consequence, in the present work, the functional design is applied to the
multi-material manufacturing using metals and ceramics to pursuit a path for the
easier design and manufacturing of such relevant components.

For this purpose, laser powder bed fusion proves to be
the most well suitable technology for the job. However, current \gls{lpbf}
technology only is guided towards mono-material fabrication. Adding to that,
functional design is hard and complex, especially if requiring multiple
materials, which drives away the end user, preventing widespread adoption.

Thus, the present work aimed to close the gap between design and fabrication of
multi-material components like the aforementioned implants, by: proposing a
design methodology for multi-material fabrication of metals/ceramics using the
\gls{lpbf} process --- \texttt{3DMMLPBF-CAD2Part};
instantiating a practical workflow and the respective toolchain from the
methodology; building a proof-of-concept low-cost equipment able to produce such
components.

The \texttt{3DMMLPBF-CAD2Part} methodology devised tries to address
the high complexity associated with the design and manufacturing of multi-material parts and fill the gap in this domain,
as no such methodology was found in the literature.
It uses a model-driven approach to reduce the problem's complexity, taking into account all key agents in the process and enabling
fast prototyping and testing. Four models and the relevant agents were
considered: design model --- designer; pre-manufacturing model ---
manufacturer; manufacturing model --- manufacturer; post-manufacturing model ---
designer, manufacturer, material engineer, mechanical engineer, control
engineer, physicist and data scientist.

At plain sight the methodology may seem trivial, but it provided a bird's eye
view over the whole process: the knowledge acquired
through its models was crucial for the development of a specialised
workflow for the \gls{3dmmlpbf} process and the respective toolchain, and to the
development of a machine capable of producing multi-material components from
metallic/ceramic powders matching the designed workflow.

Then, a toolchain was assembled and the missing software components were
developed, following the waterfall model, namely:
\begin{itemize}
\item \underline{Pre-Manufacturer}: The \texttt{3DMMLPBF-CAD2Part} methodology
  views each 3D multi-material model as an assembly of 3D mono-material models.
  Each model is exported into an \gls{stl}
  file and imported to the Pre-Manufacturer, where it is sliced and infilled
  with the manufacturing paths, according to the user-defined
  configuration. These infilled slices are then merged into a single
  manufacturing model, sorting them by z-value, and encoding the material and
  topological data into a custom version of an \gls{svg} file.
\item \underline{Manufacturer}: the manufacturing model is loaded into the
  \texttt{Manufacturer} \gls{sw}, where its topological and material information
  is mapped to the relevant process parameters, enabling the accurate
  and adequate manufacturing of the parts. Furthermore, it also allows the
  assignment of the component's layers into different lasers, widening the
  scope of multi-material fabrication to different types of materials.
  This application controls the
  \gls{3dmmlpbf}'s machine and the network of lasers, responsible for the actual
  multi-material fabrication. Lastly, process related information can be
  exported for storing and posterior analysis.
\item \underline{Post-Manufacturer}: the \emph{post-manufacturer} provides a
  process knowledge database that can be used by all manufacturing agents for
  process improvement. It enables models, manufacturing files, and mechanical
  tests images to be directly imported, easing this process.
  Furthermore, it can be used to bootstrap the
  process, as all information is readily available to use and test the equipment and the toolchain.
\end{itemize}

The toolchain was extensively tested and validated. The \texttt{Pre-Manufacturer} tests
demonstrated that this tool is capable of slicing and generating paths for
various 3D models, irrespective of their provenance, the configuration used and the
number of materials. It preserves the models coordinates and supports different
parameters for each model, even different layer heights.
It also
pointed out that the fill density and infill extrusion width can be varied to
mimic the required path filling for \gls{lpbf} trajectories.
The \texttt{Manufacturer} tests showed that it accurately maps the geometrical
and topological data to process parameters and that it efficiently handles the
\gls{3dmmlpbf} manufacturing, driving both the machine and the array of
lasers. The \texttt{Post-Manufacturer} tests proved that it can efficiently
store the process related information and displaying it a convenient way to each
manufacturing agent for further analysis.

The machine was developed using the V-model methodology in the three domain
areas: mechanics, electronics and software. The overall equipment was divided
into smaller subsystems. The mechanics were designed by another member of the
laboratory staff and assembled. The electronics were designed to control the
machine operation: motions, temperature, shielding and powder removal. The
machine's firmware was developed to handle the low-level tasks of the system,
responding to \texttt{Manufacturer}'s commands.

The equipments tests performed allow to assess the mechanics, control and laser
domains on operating conditions. A testing dialog was created in the
\texttt{Manufacturer}'s application to enable on-line testing of the equipment
in a straightforward and easy way. 
The tests conducted demonstrated the good performance of these subsystems.

In the product manufacturing tests, several bi-material components were
fabricated using one or multiple lasers. The polymeric bi-material component manufactured
using the CO\textsubscript{2} laser was
correctly produced, in compliance with the process and geometrical data
provided. The cross-sectioning and \gls{sem} analysis performed on the part
demonstrated the tridimensional material variation in both horizontal and
vertical directions, alongside with good densification. The metallic alloy
bi-material component manufactured followed a manual pre-manufacturing
strategy as a quick workaround, which demonstrated the difficulty of generating
such files without proper tooling. This component was produced using an YAG-Nd
laser and the results showed mild densification and
some geometric discrepancy between the model and the produced component.
Lastly, a multi-material component made out of one polymeric and one metallic alloy
material was manufactured using a CO\textsubscript{2} laser and a YAG-Nd
one. The multi-laser setup was cumbersome and, in the present format, it induces
focus field distortion that must be corrected. The results showed poor
densification and higher geometrical deviations, which are mostly due to this
distortions. Nonetheless, the concept was proven and may be applied using lasers
with less bulky beam generation systems, such as solid-state.

Lastly, some paths for \gls{3dmmlpbf}'s process improvement were outlined,
leveraging from the process knowledge database established by the
\texttt{Post-Manufacturer}. Data mining can be applied using \gls{ai}, for
example, for the assessment of manufacturing quality within the available
\gls{sem} analysis sets. This can be used to infer heuristics
regarding process parameters. The assessment of the manufacturing quality can
also be performed on-line using infrared thermography or \gls{lsp} technology,
which can be used for the adaptative control of the process.

Overall, the main goals of this work were fulfilled.
A global \gls{3dmmlpbf} methodology was devised and layed the foundations for the instantiation of a practical workflow and the derived
toolchain, as an effective means to materialise the functional design using
multiple materials. A low-cost equipment was designed and built as
proof-of-concept enabling the multi-material fabrication of parts through the
\gls{lpbf} technology, for only a very small fraction of the cost --- 10
k\texteuro~when compared to the commercial equipments costing hundreds of
thousands dollars or more. The manufacturing chain information feeds a growing
process knowledge database that can be used for process
improvement. Furthermore, the \gls{3dmmlpbf} ecosystem assembled can be easily and conveniently bootstrapped and replicated, due to its the
open-source nature, enabling a more widespread adoption of multi-material
fabrication and functional design.
%
\section{Prospect for future work}
%
The \gls{3dmmlpbf} methodology devised and implemented is still on its
infancy. This work represents only the first stone on paving the way for
multi-material manufacturing of materials and ceramics in a sustained way,
approaching the problem from a wider perspective and including all the key
agents in the manufacturing chain. The envisioned future work goals are listed as
follows, accordingly to its priority:
\begin{enumerate}
\item \textbf{High}
\begin{itemize}
\item Perform more tests on multi-material fabrication with: different
  materials, namely metallic
  and ceramics; processing
  parameters; topologies, including custom ones; manufacturing strategies.
  The manufacturing strategies may include sacrificial substrate, preheat
  treatments or chemical treatments (oxidation, nitruration).
\item Improve shielding and temperature control, centrally managed by the microcontroller's firmware.
\end{itemize}
%
\item \textbf{Medium}
\begin{itemize}
\item Design a \gls{pcb} to integrate the electronic components in a single board.
\end{itemize}
%
\item \textbf{Low}
\begin{itemize}
\item Abstract from Laser API --- design an open-source controller for
lasers with a custom gcode language, as \emph{grbl} --- a no-compromise, high
performance, low cost alternative to parallel-port-based motion control for CNC
milling --- is already doing~\cite{grbl}.
\item Implement database mining for process optimization through heuristics and
  guidelines.
\item Investigate alternative and more compact methods to store the
  manufacturing file information, migrating from the user-readable \gls{svg}
  file to a more machine-readable version.
\end{itemize}
%
\end{enumerate}
%
%%% Local Variables:
%%% mode: latex
%%% TeX-master: "../template"
%%% End:

