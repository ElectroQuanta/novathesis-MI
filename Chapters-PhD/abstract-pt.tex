%!TEX root = ../template.tex
%%%%%%%%%%%%%%%%%%%%%%%%%%%%%%%%%%%%%%%%%%%%%%%%%%%%%%%%%%%%%%%%%%%%
%% abstrac-pt.tex
%% NOVA thesis document file
%%
%% Abstract in Portuguese
%%%%%%%%%%%%%%%%%%%%%%%%%%%%%%%%%%%%%%%%%%%%%%%%%%%%%%%%%%%%%%%%%%%%
\typeout{NT FILE abstrac-pt.tex}%
O design funcional é uma abordagem extremamente desejável para otimizar o desempenho de um produto, minimizando o uso de recursos e custos.
No entanto, a adoção do design funcional pode exigir o uso de vários materiais
ou uma combinação deles, o que é dificultado pelas tecnologias de fabrico
atuais. Os implantes biomédicos, como o implante de anca, são um exemplo
paradigmático da necessidade de design funcional.

Atualmente, apenas alguns equipamentos comerciais são capazes de usar vários materiais~\cite{aconityMIDI, slm280}.
Além disso, exigem um investimento inicial significativo e são adequados apenas
para metais, o que limita a gama de aplicações.
Consequentemente, a literatura científica centra-se principalmente na modficação
de equipamentos mono-materiais com sistemas de deposição
multimaterial~\cite{bareth2022Implem, schneck2022capability,
  nadimpali2019MMSteels, anstaett2017fabrication, sing2015interfacial,
  liu2014interfacial}, o que infelizmente também requer soluções manuais para
configurar a cadeia de manufatura e o equipamento.
Isto é agravado pela falta de diretrizes de projeto específicas para o processo
\gls{mmlpbf} e pelo fato de que a maioria dos procedimentos de pré-processamento
usar um método manual baseado num formato de dados inadequado~\cite{schneck2021review}.
A principal conclusão é que o fabrico via \gls{mmlpbf} é um problema
multiobjetivo que requer equipamentos e ferramentas especializadas e design experimental.

Assim, o presente trabalho visa reduzir o hiato entre o design e o fabrico
de componentes multimaterial, como os implantes de anca, propondo uma
abordagem holística para a cadeia de manufatura que possa
alavancar o conhecimento do processo e apoiar o design funcional. Foi
desenvolvida uma metodologia baseada em modelos para lidar com a elevada
complexidade associada ao design e fabrico de componentes multimaterial e
colmatar a lacuna nesse domínio.
O conhecimento adquirido através dos modelos foi usado para a instanciação dum
fluxo de trabalho especializado pra o processo \gls{mmlpbf} e o desenvolvimento
de ferramentas de software de suporte.
Em seguida, foi desenvolvido um equipamento personalizado que integra vários
lasers de diferentes tipos, que podem ser efetivamente usados para o fabrico de
componentes multimaterial usando pós metálicos e cerâmicos. Pela análise
efetuada, trata-se do primeiro equipamento com esta característica.
%
Foram projetados e fabricados vários componentes multimaterial, seguindo a
metodologia desenvolvida. Esses testes validaram todo o ecossistema,
demonstrando a sua adequabilidade para o suporte do design funcional de
componentes multimaterial usando o processo \gls{mmlpbf}. Por fim, são
apresentadas algumas perspectivas para alavancar a base de conhecimento do
processo criada, onde ferramentas como a inteligência articial podem ser usadas
para melhorar de forma expedita o processo \gls{mmlpbf}.

% Palavras-chave do resumo em Português
\begin{keywords}
design funcional, 3D multimaterial, manufatura aditiva baseada em laser,
metodologia, software, equipamento
\end{keywords}
% to add an extra black line
