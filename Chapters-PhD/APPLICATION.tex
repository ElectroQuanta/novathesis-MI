% !TEX root = ../template.tex
%
\typeout{NT FILE APPLICATION.tex}%
\chapter{Tests}%
\label{ch:application}
In this chapter, the \gls{3dmmlpbf} methodology's is applied to multi-material
fabrication, using one or multiple lasers.
Thus, the complete process, from inception to produced part is tested as a
whole, to ensure the full validation of the designed ecosystem, i.e.,
methodology, workflow, toolchain, and equipment.
Lastly, the prospects for process improvement are outlined, leveraging the
process knowledge acquired for systematic and consistent evolution of the
\gls{3dmmlpbf} manufacturing chain.

\section{Multi-material mono-laser}%
\label{sec:multi-material-mono-laser}
The multi-material fabrication was tested out using a
\texttt{CO\textsubscript{2}} laser or a \texttt{YAG-Nd} one. These tests are
described next.

\subsection{CO\textsubscript{2} Laser}%
\label{sec:co2-laser-test}
As a proof-of-concept, a bi-material model --- cylinder and cross --- was
designed in FreeCAD (see Fig.~\ref{fig:cylCross-co2-model-freecad}), and each material was exported to an \gls{stl} file. The
cylinder has 25 millimeters in diameter and 11 millimeters in height. The cross
is an ``assembly'' of a parallelepiped with 7 x 7 x 2 mm, totaling a bounding
volume of 21 x 21 x 6 mm. The cross
referenced to the top plane of the cylinder, so that the cylinder's first layers
act as support material.

\begin{figure}[htbp!]
  \centering
  %
  \includegraphics[width=0.6\textwidth]{./img/cylCross-co2-model-freecad.png}
  \caption{CO\textsubscript{2} laser --- bi-material manufacturing test: FreeCAD modelling}%
  \label{fig:cylCross-co2-model-freecad}
\end{figure}

\subsubsection{Pre-Manufacturing}%
\label{sec:pre-manufacturing-co2}
The \gls{stl} files corresponding to each material were loaded in the
\underline{Pre-Manufacturer} \gls{sw} for processing
(Fig.~\ref{fig:cylCross-co2-model}) and the results are displayed in Fig.~\ref{fig:cylCross-co2-test}. The same slicing and path generation
parameters were defined for both entities (Fig.~\ref{fig:cylCross-co2-cfg}):
first layer height and layer height of $50~\mu m$, null fill angle, fill density
of $20\%$, and infill extrusion width of $50~\mu m$, without connected
paths.
%
The manufacturing model was generated with 579 slices
(Fig.~\ref{fig:cylCross-co2-main}). Layer $0$, located at $z = 50~\mu m$, shows the
cylinder's bottom layers, without any intersecting slices belonging to the cross
(Fig.~\ref{fig:cylCross-co2-layer0}).
Layer 100, located at
$z = 5025~\mu m$, shows, as expected, the first bi-material layers (Fig.~\ref{fig:cylCross-co2-layer100-mat2}, and
Fig.~\ref{fig:cylCross-co2-layer100-mat1}).
Lastly, layer 120, located at $z = 7025~\mu m$, shows the first of the internal
layers where the cross is fully displayed
(Fig.~\ref{fig:cylCross-co2-layer180-mat2} and Fig.~\ref{fig:cylCross-co2-layer180-mat1}).

Thus, the slicing and path generation complied to the geometrical data (input
\gls{stl} models) and the configuration established, further validating the \underline{Pre-Manufacturer} \gls{sw}.

% subfigures
\begin{figure}[htbp!]
  \centering
  %
  \begin{subfigure}[t]{0.48\textwidth}
  \centering
  \includegraphics[width=0.75\textwidth]{./img/cylCross-co2-model.png}
  \caption{3D model preview}%
  \label{fig:cylCross-co2-model}
  \end{subfigure}
%
  \begin{subfigure}[t]{0.48\textwidth}
  \centering
  \includegraphics[width=0.66\textwidth]{./img/cylCross-co2-cfg.png}
  \caption{Slicing and path generation setup}%
  \label{fig:cylCross-co2-cfg}
\end{subfigure}
%
  \begin{subfigure}[t]{0.48\textwidth}
  \centering
  \includegraphics[width=1.0\textwidth]{./img/cylCross-co2-main-crop.png}
  \caption{Output: main view}%
  \label{fig:cylCross-co2-main}
\end{subfigure}
%
  \begin{subfigure}[t]{0.48\textwidth}
  \centering
  \includegraphics[width=1.0\textwidth]{./img/cylCross-co2-layer0.png}
  \caption{Output: Layer 0}%
  \label{fig:cylCross-co2-layer0}
\end{subfigure}
%
  \begin{subfigure}[t]{0.48\textwidth}
  \centering
  \includegraphics[width=1.0\textwidth]{./img/cylCross-co2-layer100-mat1.png}
  \caption{Output: Layer 100 --- material 1}%
  \label{fig:cylCross-co2-layer100-mat1}
\end{subfigure}
%
%
  \begin{subfigure}[t]{0.48\textwidth}
  \centering
  \includegraphics[width=1.0\textwidth]{./img/cylCross-co2-layer100-mat2.png}
  \caption{Output: Layer 100 --- material 2}%
  \label{fig:cylCross-co2-layer100-mat2}
\end{subfigure}
%
  \begin{subfigure}[t]{0.48\textwidth}
  \centering
  \includegraphics[width=1.0\textwidth]{./img/cylCross-co2-layer180-mat2.png}
  \caption{Output: Layer 180 --- material 2}%
  \label{fig:cylCross-co2-layer180-mat2}
\end{subfigure}
%
  \begin{subfigure}[t]{0.48\textwidth}
  \centering
  \includegraphics[width=1.0\textwidth]{./img/cylCross-co2-layer180-mat1.png}
  \caption{Output: Layer 180 --- material 1}%
  \label{fig:cylCross-co2-layer180-mat1}
\end{subfigure}
  % 
  \caption{CO\textsubscript{2} laser --- bi-material manufacturing test: Pre-Manufacturing
  processing}%
  \label{fig:cylCross-co2-test}
\end{figure}

\subsubsection{Manufacturing}%
\label{sec:manufacturing-co2}
The resulting manufacturing model was then loaded to the \texttt{Manufacturer}'s
\gls{sw} as illustrated in Fig.~\ref{fig:cylCross-co2-manuf-init}. The master system
was connected to the \gls{3dmmlpbf} machine via COM port.
Upon the successful connection, the machine was automatically homed.
The process parameters
were mapped to each material through the \texttt{Manage Pens} pushbutton, as
depicted in Fig.~\ref{fig:cylCross-co2-manuf-pens}.
A calibration was also performed to minimize powder usage.

% subfigures
\begin{figure}[htbp!]
%
  \begin{subfigure}[t]{0.70\textwidth}
    \centering
  \includegraphics[width=1.0\textwidth]{./img/cylCross-co2-manuf-init.png}
  \caption{Initialization}%
  \label{fig:cylCross-co2-manuf-init}
\end{subfigure}
%
  \begin{subfigure}[t]{0.28\textwidth}
  \centering
  \includegraphics[width=1.0\textwidth]{./img/cylCross-co2-manuf-pens.png}
  %\caption{Mapping geometry to process parameters}%
  \caption{Process parameters}%
  \label{fig:cylCross-co2-manuf-pens}
  \end{subfigure}
%
  %
  \begin{subfigure}[t]{.70\textwidth}
  \centering
    \includegraphics[width=1.0\textwidth]{./img/cylCross-co2-manuf-middle.png}
  \caption{Manufacturing on-going}%
  \label{fig:cylCross-co2-manuf-middle}
  \end{subfigure}
%
  \centering
  \begin{subfigure}[t]{.28\textwidth}
    \includegraphics[width=1.0\textwidth]{./img/cross-cube-bed-01.jpg}
  \caption{Machine bed}%
  \label{fig:cylCross-co2-manuf-middle-bed}
  \end{subfigure}
  \centering
  \begin{subfigure}[t]{0.9\textwidth}
    \includegraphics[width=1.0\textwidth]{./img/cylCross-co2-manuf-end.png}
  \caption{Manufacturing complete}%
  \label{fig:cylCross-co2-manuf-end}
  \end{subfigure}
  % 
  \caption{CO\textsubscript{2} laser --- bi-material manufacturing test: Manufacturing}%
  \label{fig:cylCross-co2-manuf}
\end{figure}

The filling procedure was executed for both materials. 
For easier
demonstration, the same material was used for the cylinder and the cross --- an electrostatic polymeric coating
based on TGIC polyester~\cite{eastwoodPowder} --- but with different colors for better visualization.
After concluding these
steps and validating the calibration, the machine's initialization was complete.

The manufacturing was then initiated by selecting all layers to manufacture and
pressing the \texttt{Run} pushbutton. 
Fig.~\ref{fig:cylCross-co2-manuf-middle} and Fig.~\ref{fig:cylCross-co2-manuf-middle-bed} illustrates the manufacturing: on the left the visualization
of the current layer and the corresponding \gls{ui} status; on the right the
result of the layer after being printed. As it can be seen, the layer is
correctly printed, in compliance with the process and geometrical data provided.
After more than two hours, the 579 layers were manufactured (Fig.~\ref{fig:cylCross-co2-manuf-end}).

Fig.~\ref{fig:cylCross-co2-produced-part} illustrates the bi-material part produced, in
cross-section (Fig.~\ref{fig:cylCross-co2-part-crossSection}) and orthogonal views (Fig.~\ref{fig:cylCross-co2-part-ortho}), where it is clearly visible the
tridimensional material variation, as defined by the original 3D CAD
model.

%
\begin{figure}[hbtp!]
  \centering
  \begin{subfigure}[t]{.48\textwidth}
    \includegraphics[width=0.8\textwidth]{./img/cross-cube-02.jpg}
  \caption{Cross-section view}%
  \label{fig:cylCross-co2-part-crossSection}
  \end{subfigure}
%
  \centering
  \begin{subfigure}[t]{.48\textwidth}
  \centering
    \includegraphics[width=0.59\textwidth]{./img/cylCross-part-ortho.png}
  \caption{Orthogonal view}%
  \label{fig:cylCross-co2-part-ortho}
  \end{subfigure}
  % 
  \caption{CO\textsubscript{2} laser --- bi-material manufacturing test:
    Produced part}%
  \label{fig:cylCross-co2-produced-part}
\end{figure}

\subsubsection{Post-Manufacturing}%
\label{sec:post-manufacturing-co2}
After manufacturing was completed, several analysis were performed on the
produced part to assess the manufacturing quality, namely on geometrical
compliance and densification.

Geometrical compliance was assessed using a image processing \gls{sw} package
based on \texttt{ImageJ} ---
\texttt{Fiji}~\cite{fijiImageJ,schindelin2012fiji} --- using a 1 cent coin as a
base for the measurements (Fig.~\ref{fig:cylCross-co2-imagej}). The measurements
performed showed a slight discrepancy, especially in the z-axis, which may be
due to the densification effect or measurement errors, but overall the dimensions of the produced part
match the ones from the 3D \gls{cad} model.

% subfigures
\begin{figure}[htbp!]
  \centering
  %
  \includegraphics[width=1.0\textwidth]{./img/cylCross-co2-imagej.png}%
  % \caption{\gls{gui}}%
  \caption{CO\textsubscript{2} laser --- bi-material manufacturing test:
    Geometrical measurements using Fiji's SW}%
  \label{fig:cylCross-co2-imagej}
\end{figure}

The \gls{sem} analysis performed on the part
(Fig.~\ref{fig:sem-analysis}) showed good densification, demonstrating the good
manufacturing performance of the equipment.

Thus, the equipment is capable of
producing multi-material components using the \gls{3dmmlpbf} process, in
compliance to the 3D models geometry, and with good overall manufacturing
performance by enabling the mapping of geometrical and topological data to
process parameters and through robust control of the process.
%
% subfigures
\begin{figure}[htb!]
  \centering
  %
  \begin{subfigure}[t]{.48\textwidth}
    \includegraphics[width=1.0\textwidth]{./img/SEM.jpg}%
  %\caption{\gls{gui}}%
  \label{fig:sem}
  \end{subfigure}
%
  \begin{subfigure}[t]{.48\textwidth}
    \includegraphics[width=1.0\textwidth]{./img/SEM-2.jpg}%
  %\caption{Machine bed}%
  \label{fig:sem-2}
  \end{subfigure}
  % 
  %
  \begin{subfigure}[t]{.48\textwidth}
    \includegraphics[width=1.0\textwidth]{./img/SEM-3.jpg}%
  %\caption{\gls{gui}}%
  \label{fig:sem-3}
\end{subfigure}
  %
  \begin{subfigure}[t]{.48\textwidth}
    \includegraphics[width=1.0\textwidth]{./img/SEM-4.jpg}%
  %\caption{\gls{gui}}%
  \label{fig:sem-4}
  \end{subfigure}
  \caption{CO\textsubscript{2} laser --- bi-material manufacturing test:
    SEM analysis}%
  \label{fig:sem-analysis}
\end{figure}

The information concerning the manufacturing process as a whole was then
collected and added to the \emph{Post\=/Manufacturer}.
Fig.~\ref{fig:postManuf-premanuf-co2} shows the \underline{Pre-Manufacturer}'s data
collection, storing the input and output files' related information,
respectively. It allows quick navigation and visualisation of these information
flows, which are indexed to a produced part.

% subfigures
\begin{figure}[htb!]
  \centering
  %
  \begin{subfigure}[t]{.8\textwidth}
    \includegraphics[width=1.0\textwidth]{./img/postManuf-premanuf-input-co2.png}%
  \caption{Pre-Manufacturer input files manager}%
  \label{fig:postManuf-premanuf-input-co2}
  \end{subfigure}
%
  \begin{subfigure}[t]{.8\textwidth}
    \includegraphics[width=1.0\textwidth]{./img/postManuf-premanuf-output-co2.png}%
  \caption{Pre-Manufacturer output files manager}%
  \label{fig:postManuf-premanuf-output-co2}
\end{subfigure}
%
  \caption{CO\textsubscript{2} laser --- Post-Manufacturer: Pre-Manufacturer
    data management}%
  \label{fig:postManuf-premanuf-co2}
\end{figure}

Fig.~\ref{fig:postManuf-manuf-co2} shows the \underline{Manufacturer}'s data
collection, storing the input, configuration and logs' related information,
respectively. It allows quick navigation and visualisation of these information
flows, which are also indexed to a produced part. The input and configuration
data can be used in conjunction with the mechanical tests to analyse the
manufacturing performance and behavior. On the other hand, the logs can be used
by the control/systems engineer to analyse the equipment and software behavior,
and improve it.

% subfigures
\begin{figure}[htb!]
  \centering
  %
  \begin{subfigure}[t]{.6\textwidth}
    \includegraphics[width=1.0\textwidth]{./img/postManuf-manuf-layers-co2.png}%
  \caption{Manufacturer input manager}%
  \label{fig:postManuf-manuf-layers-co2}
  \end{subfigure}
%
  \begin{subfigure}[t]{.6\textwidth}
    \includegraphics[width=1.0\textwidth]{./img/postManuf-manuf-pens-co2.png}%
  \caption{Manufacturer configuration manager}%
  \label{fig:postManuf-manuf-pens-co2}
\end{subfigure}
%
  \begin{subfigure}[t]{.6\textwidth}
    \includegraphics[width=1.0\textwidth]{./img/postManuf-manuf-logs-co2.png}%
  \caption{Manufacturer logs manager}%
  \label{fig:postManuf-manuf-logs-co2}
\end{subfigure}
%
  \caption{CO\textsubscript{2} laser --- Post-Manufacturer: Manufacturer
    data management}%
  \label{fig:postManuf-manuf-co2}
\end{figure}

Fig.~\ref{fig:postManuf-mechTestManager} depicts the \texttt{Mechanical tests
  manager} \gls{ui} added to the \texttt{Post-Manufacturer} software to handle
the mechanical tests performed on the produced components. These tests can be
added to the database with an accompanying image, for example \gls{sem}, which
can then be downloaded and used for analysis, for example, using \gls{ai}. The analysis results can then be uploaded back to the database for
tracking and process improvement.
\begin{figure}[hbtp!]
  \centering
    \includegraphics[width=1.0\textwidth]{./img/postManuf-mechTestManager.png}
  %A
  \caption{CO\textsubscript{2} laser --- Post-Manufacturer: Mechanical tests manager}%
  \label{fig:postManuf-mechTestManager}
\end{figure}


\subsection{YAG-Nd Laser}%
\label{sec:yag-nd-laser-test}
After proving the correctness of the overall process --- methodology, workflow,
toolchain and equipment --- more multi-material parts can be produced, but now,
using metals.

\subsubsection{Pre-Manufacturing}%
\label{sec:pre-manufacturing-yag}
The \texttt{CO\textsubscript{2}} laser's
spot diameter is very large (50~$\mu$m) for metallic powders granulometry, thus
requiring the usage of a different laser type, like the \texttt{YAG-Nd} one.
In that sense, a multi-material metallic component was conceptualized for
manufacturing.

However, this time, the conventional method was used to illustrate its
difficulties. Thus, instead of 3D modelling and running the multi-material model
through the Pre\=/Manufacturer software, the multi-material component was
designed directly in a 2D vector drawing \gls{sw}, EzCAD, in this case.
This forced the designer to draw 2D layers and stack them on top of each other
--- mimicking the slicing --- and manually perform the hatching --- mimicking
the path generation.
With this approach the designer loses the 3D perspective, which makes it significantly harder for the
conceptual design of the component. This downside is especially critical if
multi-material is added to the equation, as now, the designer needs to be aware
and track, not only, the tridimensional shape of the component, but also the
materials' placement and its interfaces.
As it becomes obvious, this completely defeats the purpose of functional design.
Nonetheless, the \texttt{Manufacturer} is still able to process such models, as
long as the layer naming convention is respected. 

Fig.~\ref{fig:custom-yag-ezcad} illustrates the modelling procedure,
with some layers containing sets of grouped points (pattern), and others being
hatched (to act as a supporting layer).

\begin{figure}[hbtp!]
  \centering
    \includegraphics[width=0.7\textwidth]{./img/custom-yag-ezcad.png}
  %A
  \caption{YAG-Nd laser: bi-material manufacturing test --- Modelling in EzCAD}%
  \label{fig:custom-yag-ezcad}
\end{figure}

This model was then loaded into the \texttt{Manufacturer} \gls{sw} for
validation, as it eases the visualisation
process. Fig.~\ref{fig:custom-yag-model-preview} shows the basic stratification
of the model. The model has 62 slices totaling $4275~\mu m$ in height
(Fig.~\ref{fig:custom-yag-model-overview}).
The first ten layers (Fig~\ref{fig:custom-yag-model-init}), with $150~\mu m$ in height, correspond to a
cross pattern replicated across the whole area. Then, from layer 11 to 20 (Fig.~\ref{fig:custom-yag-model-layer11}), with
$75~\mu m$ in height, we have a hatched pattern to act as a supporting substrate
for the multi-material layers. From layer 21 to 31, we have slices containing
both materials with a chess board pattern, but offsetted, to promote mechanical
interlocking of the layers (see Fig.~\ref{fig:custom-yag-model-layer21-mat2},
and Fig.~\ref{fig:custom-yag-model-layer21-mat1}, respectively).
Lastly, from layer 32 to the end, we have a hatched pattern to close off the
part (Fig.~\ref{fig:custom-yag-model-preview}).

% subfigures
\begin{figure}[htbp!]
  \centering
  %
  \begin{subfigure}[t]{0.48\textwidth}
  \centering
  \includegraphics[width=1.0\textwidth]{./img/custom-yag-model-overview.png}
  \caption{3D model preview}%
  \label{fig:custom-yag-model-overview}
  \end{subfigure}
%
  \begin{subfigure}[t]{0.48\textwidth}
  \centering
  \includegraphics[width=1.0\textwidth]{./img/custom-yag-model-init.png}
  \caption{Layer 0}%
  \label{fig:custom-yag-model-init}
\end{subfigure}
  % 
  \begin{subfigure}[t]{0.48\textwidth}
  \centering
  \includegraphics[width=1.0\textwidth]{./img/custom-yag-model-middle-mat2.png}
  \caption{Layer 11}%
  \label{fig:custom-yag-model-layer11}
\end{subfigure}
  % 
  \begin{subfigure}[t]{0.48\textwidth}
  \centering
  \includegraphics[width=1.0\textwidth]{./img/custom-yag-model-layer21-mat2.png}
  \caption{Layer 21 --- material 2}%
  \label{fig:custom-yag-model-layer21-mat2}
\end{subfigure}
  % 
  \begin{subfigure}[t]{0.48\textwidth}
  \centering
  \includegraphics[width=1.0\textwidth]{./img/custom-yag-model-layer21-mat1.png}
  \caption{Layer 21 --- material 1}%
  \label{fig:custom-yag-model-layer21-mat1}
\end{subfigure}
  % 
  % 
  \begin{subfigure}[t]{0.48\textwidth}
  \centering
  \includegraphics[width=1.0\textwidth]{./img/custom-yag-model-layer43.png}
  \caption{Layer 32}%
  \label{fig:custom-yag-model-layer43}
\end{subfigure}
%
  \caption{YAG-Nd laser --- bi-material manufacturing test: Model preview}%
  \label{fig:custom-yag-model-preview}
\end{figure}

\subsubsection{Manufacturing}%
\label{sec:manufacturing-yag}
After loading the model, the initialization procedure was performed, as
illustrated in Fig.~\ref{fig:custom-yag-manuf-init}, comprising the
\gls{3dmmlpbf} machine communication setup and homing, the calibration for
powder minimisation, and powder filling.
The process parameters
were mapped to each material, Ti6Al4V and CoCrMo, as shown in
Fig.~\ref{fig:custom-yag-manuf-pens}. The same set of parameters were applied to
both materials, with significantly slower marking speeds ($10~mm/s$) and higher
power ratios ($55\%$).

% subfigures
\begin{figure}[htbp!]
%
  \begin{subfigure}[t]{0.70\textwidth}
    \centering
  \includegraphics[width=1.0\textwidth]{./img/custom-yag-manuf-init.png}
  \caption{Initialization}%
  \label{fig:custom-yag-manuf-init}
\end{subfigure}
%
  \begin{subfigure}[t]{0.28\textwidth}
  \centering
  \includegraphics[width=1.0\textwidth]{./img/custom-yag-manuf-pens.png}
  %\caption{Mapping geometry to process parameters}%
  \caption{Process parameters}%
  \label{fig:custom-yag-manuf-pens}
  \end{subfigure}
%
  \centering
  \begin{subfigure}[t]{0.7\textwidth}
    \includegraphics[width=1.0\textwidth]{./img/custom-yag-manuf-end.png}
  \caption{Manufacturing halted}%
  \label{fig:custom-yag-manuf-end}
  \end{subfigure}
%
  \centering
  \begin{subfigure}[t]{.28\textwidth}
    \includegraphics[width=1.0\textwidth]{./img/custom-yag-manuf-bed.jpg}
  \caption{Machine bed}%
  \label{fig:custom-yag-manuf-bed}
\end{subfigure}
%
  \caption{YAG-Nd laser --- bi-material manufacturing test: Manufacturing}%
  \label{fig:custom-yag-manuf}
\end{figure}

The manufacturing was then initiated by selecting all layers to manufacture and
pressing the \texttt{Run} pushbutton. More than half the layers were correctly
manufactured, however, at layer 29, the process was halted (Fig.~\ref{fig:custom-yag-manuf-end}), as some significant
distortion were occurring at the interfaces between the two
metals. Fig.~\ref{fig:custom-yag-manuf-bed} illustrates the machine bed when the
halting occurred.

Fig.~\ref{fig:custom-yag-produced-part} shows the bi-material part produced,
where it is clearly visible some of the distortions that took place and led to
process aborting. Thus, more work in the 3D model design and manufacturing setup
for this geometry with the designated metallic pair.
%
\begin{figure}[hbtp!]
  \centering
    \includegraphics[width=0.5\textwidth]{./img/custom-yag-part.jpg}
%
  \caption{YAG-Nd laser --- bi-material manufacturing test:
    Produced part}%
  \label{fig:custom-yag-produced-part}
\end{figure}

\subsubsection{Post-Manufacturing}%
\label{sec:post-manufacturing-yag}
As important as documenting the successes, is the failed trials
documentation. As such, the process information was collected and stored in the
\texttt{Post\=/Manufacturer}.

Fig.~\ref{fig:postManuf-manuf-yag} shows the \underline{Manufacturer}'s data
collection, storing the input, configuration and logs' related information,
respectively. As can be seen, this information is appended to the already stored data.

% subfigures
\begin{figure}[htb!]
  \centering
  %
  \begin{subfigure}[t]{.6\textwidth}
    \includegraphics[width=1.0\textwidth]{./img/postManuf-manuf-layers-yag.png}%
  \caption{Manufacturer input manager}%
  \label{fig:postManuf-manuf-layers-yag}
  \end{subfigure}
%
  \begin{subfigure}[t]{.6\textwidth}
    \includegraphics[width=1.0\textwidth]{./img/postManuf-manuf-pens-yag.png}%
  \caption{Manufacturer configuration manager}%
  \label{fig:postManuf-manuf-pens-yag}
\end{subfigure}
%
  \begin{subfigure}[t]{.6\textwidth}
    \includegraphics[width=1.0\textwidth]{./img/postManuf-manuf-logs-yag.png}%
  \caption{Manufacturer logs manager}%
  \label{fig:postManuf-manuf-logs-yag}
\end{subfigure}
%
  \caption{YAG-Nd laser --- Post-Manufacturer: Manufacturer
    data management}%
  \label{fig:postManuf-manuf-yag}
\end{figure}

Fig.~\ref{fig:postManuf-mechTestManager-yag} shows the \gls{em} image
analysis performed on the component added to the database. These images further
support the manufacturing's halting decision, as the defects become
clearer. Once again, data is appended, contributing to the increasing process
knowledge base.

\begin{figure}[hbtp!]
  \centering
    \includegraphics[width=1.0\textwidth]{./img/postManuf-mechTestManager-yag.png}
  %A
  \caption{YAG-Nd laser --- Post-Manufacturer: Mechanical tests manager}%
  \label{fig:postManuf-mechTestManager-yag}
\end{figure}

\section{Multi-material multi-laser}%
\label{sec:multi-material-multi}
The multi-material fabrication was also tested using multiple lasers, in this
case, CO2 and YAG-Nd. This is important because polymeric/composite materials
have different absorption wavelength spectrum than the metallic ones, which requires the
usage of different laser types.
Thus, for multi-material fabrication consisting
of a combination of polymeric/composite materials with metallic ones, multiple
lasers must be used efficiently.

The cylinder and cross model was used once again, but
scaled, (see Fig.~\ref{fig:cylCross-co2-model-freecad}), and each material was exported to an \gls{stl} file. The
cylinder has 11 millimeters in diameter and 2.5 millimeters in height. The cross
is an ``assembly'' of a parallelepiped with 2 x 2 x 0.4 mm, totaling a bounding
volume of 6 x 6 x 1.2 mm. The cross
referenced to the top plane of the cylinder, so that the cylinder's first layers
act as support material.

\begin{figure}[htbp!]
  \centering
  %
  \includegraphics[width=0.6\textwidth]{./img/cylCross-multi-model-freecad.png}
  \caption{Multi-laser --- bi-material manufacturing test: FreeCAD modelling}%
  \label{fig:cylCross-multi-model-freecad}
\end{figure}

One very interesting idea behind the usage of dissimilar types of materials is
the \underline{sacrificial substrate}. In this concrete example, the cross can be produced
using a polymeric material, which could then be removed to yield its negative
made out of a metallic alloy. The sacrificial substrate is significantly cheaper
than the base material, and enables the production of a negative without
manufacturing defects such as the ones resulting from hanging slices (without
support material). This also limits cross contamination to the base material.

\subsection{Pre-Manufacturing}%
\label{sec:pre-manufacturing-multiLaser}
The \gls{stl} files corresponding to each material were loaded in the
\underline{Pre-Manufacturer} \gls{sw} for processing
(Fig.~\ref{fig:cylCross-multi-model}) and the results are displayed in Fig.~\ref{fig:cylCross-multi-test}. The same slicing and path generation
parameters were defined for both entities (Fig.~\ref{fig:cylCross-multi-cfg}):
first layer height and layer height of $50~\mu m$, null fill angle, fill density
of $10\%$, and infill extrusion width of $20~\mu m$, without connected
paths.
%
The manufacturing model was generated with 103 slices
(Fig.~\ref{fig:cylCross-multi-main}). Layer $0$, located at $z = 50~\mu m$, shows the
cylinder's bottom layers, without any intersecting slices belonging to the cross
(Fig.~\ref{fig:cylCross-multi-layer0}).
Layer 27, located at
$z = 1325~\mu m$, shows, as expected, the first bi-material layers (Fig.~\ref{fig:cylCross-multi-layer27-mat1}, and
Fig.~\ref{fig:cylCross-multi-layer27-mat2}).
Lastly, layer 42, located at $z = 1725~\mu m$, shows the first of the internal
layers where the cross is fully displayed
(Fig.~\ref{fig:cylCross-multi-layer42-mat1} and Fig.~\ref{fig:cylCross-multi-layer42-mat2}).

Thus, once again, the slicing and path generation complied to the geometrical data (input
\gls{stl} models) and the defined configuration, further validating the \underline{Pre-Manufacturer} \gls{sw}.

% subfigures
\begin{figure}[htbp!]
  \centering
  %
  \begin{subfigure}[t]{0.48\textwidth}
  \centering
  \includegraphics[width=0.9\textwidth]{./img/cylCross-multi-model.png}
  \caption{3D model preview}%
  \label{fig:cylCross-multi-model}
  \end{subfigure}
%
  \begin{subfigure}[t]{0.48\textwidth}
  \centering
  \includegraphics[width=0.66\textwidth]{./img/cylCross-multi-cfg.png}
  \caption{Slicing and path generation setup}%
  \label{fig:cylCross-multi-cfg}
\end{subfigure}
%
  \begin{subfigure}[t]{0.48\textwidth}
  \centering
  \includegraphics[width=1.0\textwidth]{./img/cylCross-multi-main.png}
  \caption{Output: main view}%
  \label{fig:cylCross-multi-main}
\end{subfigure}
%
  \begin{subfigure}[t]{0.48\textwidth}
  \centering
  \includegraphics[width=1.0\textwidth]{./img/cylCross-multi-layer0.png}
  \caption{Output: Layer 0}%
  \label{fig:cylCross-multi-layer0}
\end{subfigure}
%
  \begin{subfigure}[t]{0.48\textwidth}
  \centering
  \includegraphics[width=1.0\textwidth]{./img/cylCross-multi-layer27-mat1.png}
  \caption{Output: Layer 27 --- material 1}%
  \label{fig:cylCross-multi-layer27-mat1}
\end{subfigure}
%
%
  \begin{subfigure}[t]{0.48\textwidth}
  \centering
  \includegraphics[width=1.0\textwidth]{./img/cylCross-multi-layer27-mat2.png}
  \caption{Output: Layer 27 --- material 2}%
  \label{fig:cylCross-multi-layer27-mat2}
\end{subfigure}
%
  \begin{subfigure}[t]{0.48\textwidth}
  \centering
  \includegraphics[width=1.0\textwidth]{./img/cylCross-multi-layer42-mat1.png}
  \caption{Output: Layer 42 --- material 1}%
  \label{fig:cylCross-multi-layer42-mat1}
\end{subfigure}
%
  \begin{subfigure}[t]{0.48\textwidth}
  \centering
  \includegraphics[width=1.0\textwidth]{./img/cylCross-multi-layer42-mat2.png}
  \caption{Output: Layer 42 --- material 2}%
  \label{fig:cylCross-multi-layer42-mat2}
\end{subfigure}
  % 
  \caption{Multi-laser --- bi-material manufacturing test: Pre-Manufacturing
  processing}%
  \label{fig:cylCross-multi-test}
\end{figure}


\subsection{Manufacturing}%
\label{sec:manufacturing-multiLaser}
Multi-laser manufacturing requires a special setup, due to the available lasers'
characteristics and its specifics, namely, bulky laser generation systems, and
different focus distances.

Thus, firstly, the lasers arrangement was modelled in a
\gls{cad} software for quick iteration (see Fig.~\ref{fig:multiLaser-setup-model}).
The bulkiness and different
focus distances dictate that the lasers must be tilted in order to obtain a
working overlapping area. However, the laser tilting produces focus
field distortion, which must be corrected to limit geometrical deviations.

The lasers' focus cones were placed at the focus height and with a minimum
offset apart (Fig.~\ref{fig:multiLaser-setup-model-1}), and the lasers were
tilted (Fig.~\ref{fig:multiLaser-setup-model-2}), determining the
intersection between them (Fig.~\ref{fig:multiLaser-setup-model-3}). The intersection plane must have at least the area of
the printing bed --- a circle of 28 mm in diameter (Fig.~\ref{fig:multiLaser-setup-model-4}). The minimum tilting angle
that yields a working overlapping area of 28 mm in diameter is circa 27 degrees.

% subfigures
\begin{figure}[htbp!]
%
  \begin{subfigure}[t]{0.48\textwidth}
    \centering
  \includegraphics[width=0.5\textwidth]{./img/multiLaser-setup-model-1.png}
  \caption{Initial state}%
  \label{fig:multiLaser-setup-model-1}
\end{subfigure}
%
  \begin{subfigure}[t]{0.48\textwidth}
  \centering
  \includegraphics[width=0.8\textwidth]{./img/multiLaser-setup-model-2.png}
  %\caption{Mapping geometry to process parameters}%
  \caption{Tilting}%
  \label{fig:multiLaser-setup-model-2}
  \end{subfigure}
%
  %
  \begin{subfigure}[t]{.48\textwidth}
  \centering
    \includegraphics[width=0.6\textwidth]{./img/multiLaser-setup-model-3.png}
  \caption{Determining intersection plane}%
  \label{fig:multiLaser-setup-model-3}
  \end{subfigure}
%
  \centering
  \begin{subfigure}[t]{.48\textwidth}
    \includegraphics[width=0.7\textwidth]{./img/multiLaser-setup-model-4.png}
  \caption{Measuring the intersection area}%
  \label{fig:multiLaser-setup-model-4}
  \end{subfigure}
  \centering
  % 
  \caption{Multi-laser --- bi-material manufacturing test: Lasers' setup modelling}%
  \label{fig:multiLaser-setup-model}
\end{figure}

However, for practical reasons, such as secure placement of both lasers and the
collision with the powder recovery systems, only the CO\textsubscript{2} laser
was tilted (Fig~\ref{fig:multiLaser-setup-1}). Then, both laser beams were
calibrated and aligned to the centre of printing bed (Fig.~\ref{fig:multiLaser-setup-2}).

% subfigures
\begin{figure}[htbp!]
%
  \centering
  \begin{subfigure}[t]{0.48\textwidth}
    \centering
  \includegraphics[width=0.6\textwidth]{./img/multiLaser-alignment-side.jpg}
  \caption{Laser tilting}%
  \label{fig:multiLaser-setup-1}
\end{subfigure}
  %
  \begin{subfigure}[t]{.48\textwidth}
  \centering
    \includegraphics[width=0.85\textwidth]{./img/multiLaser-alignment.jpg}
  \caption{Lasers' beam calibration}%
  \label{fig:multiLaser-setup-2}
  \end{subfigure}
  % 
  \caption{Multi-laser --- bi-material manufacturing test: Lasers' setup}%
  \label{fig:multiLaser-setup}
\end{figure}

Following the \texttt{Manufacturer}'s deployment diagram (see
Fig.~\ref{fig:manuf-deployment}), one instance of the software was deployed to
each laser's computer (Fig~\ref{fig:multiLaser-setup-annotated}). The computers were connected using an Ethernet cable, and
the main system (master) was connected to the machine via \gls{usb} cable. 

\begin{figure}[htbp!]
%
  \centering
    \includegraphics[width=1.0\textwidth]{./img/multiLaser-setup.png}
  \caption{Multi-laser --- bi-material manufacturing test: Manufacturing setup}%
  \label{fig:multiLaser-setup-annotated}
\end{figure}

The resulting manufacturing model was then loaded to each
\texttt{Manufacturer}'s \gls{sw} instance and the network was configured and
setup for master
and slave connection. Fig.~\ref{fig:cylCross-multi-networkSetup}
illustrates the network setup configuration: in the master system the slave
network address and port is added
(Fig.~\ref{fig:cylCross-multi-manuf-networkSetup-master}), and the connection is
established by pressing
the \texttt{Connect} pushbutton. If the network configuration is correct, the
master system connects to the slave, which then prints its network ID (Fig.~\ref{fig:cylCross-multi-manuf-networkSetup-slave}).

% subfigures
\begin{figure}[htbp!]
%
  \centering
  \begin{subfigure}[t]{0.80\textwidth}
    \centering
  \includegraphics[width=1.0\textwidth]{./img/cylCross-multi-manuf-networkSetup.png}
  \caption{Master configuration and connection establishment}%
  \label{fig:cylCross-multi-manuf-networkSetup-master}
\end{subfigure}
%
  %
  \begin{subfigure}[t]{.80\textwidth}
  \centering
    \includegraphics[width=1.0\textwidth]{./img/cylCross-multi-manuf-networkSetup-slave.png}
  \caption{Slave accepts connection and prints its ID}%
  \label{fig:cylCross-multi-manuf-networkSetup-slave}
  \end{subfigure}
  % 
  \caption{Multi-laser --- bi-material manufacturing test: Manufacturing
    (network setup)}%
  \label{fig:cylCross-multi-networkSetup}
\end{figure}
%

The manufacturing parameters were configured, as illustrated in
Fig.~\ref{fig:cylCross-multi-manuf-pens}, with pen $0$ being mapped to the
polymeric material, and pen $1$ to the metallic alloy.
Layers were then assigned to each laser, by sorting them by material and
defining the laser, as shown in Fig.~\ref{fig:cylCross-multi-manuf-pensMap}.

% subfigures
\begin{figure}[htbp!]
%
  \centering
  \begin{subfigure}[t]{0.48\textwidth}
    \centering
  \includegraphics[width=0.8\textwidth]{./img/cylCross-multi-manuf-pens.png}
  \caption{Parameters setup}%
  \label{fig:cylCross-multi-manuf-pens}
\end{subfigure}
%
  %
  \begin{subfigure}[t]{.48\textwidth}
  \centering
    \includegraphics[width=0.6\textwidth]{./img/cylCross-multi-manuf-pensMap.png}
  \caption{Layers assignment to lasers}%
  \label{fig:cylCross-multi-manuf-pensMap}
  \end{subfigure}
  % 
  \caption{Multi-laser --- bi-material manufacturing test: Manufacturing
    (Parameters and lasers setup)}%
  \label{fig:cylCross-multi-manuf-pensLasers}
\end{figure}

The master system
was connected to the \gls{3dmmlpbf} machine via COM port.
Upon the successful connection, the machine was automatically homed.
A calibration was performed to minimize powder usage and the filling procedure was executed for both materials. 
After concluding these
steps and validating the calibration, the machine's initialization was complete.

Fig.~\ref{fig:cylCross-multi-manuf-procedure} illustrates the manufacturing
procedure. The manufacturing was initiated by selecting all layers to manufacture and
pressing the \texttt{Run} pushbutton~\ref{fig:cylCross-multi-manuf-init}.
Fig.~\ref{fig:cylCross-multi-manuf-bed} and Fig.~\ref{fig:cylCross-multi-manuf-bed-2} illustrates the on-going manufacturing,
where it can be seen the multi-material part being built, first adding the
material and then marking the paths with the laser.
Fig.~\ref{fig:cylCross-multi-manuf-end} and
Fig.~\ref{fig:cylCross-multi-manuf-slave} shows the final manufacturing state:
on layer 70 it was detected some defects on the polymeric sintering, which
forced the process to be aborted. These defects may arise from the selected
layer height (the same as the metallic alloy), from the powder's bed
temperature, from the different processing parameters, or from a combination of
all of them.

% subfigures
\begin{figure}[htbp!]
%
  \centering
  \begin{subfigure}[t]{1.0\textwidth}
    \centering
  \includegraphics[width=0.65\textwidth]{./img/cylCross-multi-manuf-init.png}
  \caption{Start}%
  \label{fig:cylCross-multi-manuf-init}
\end{subfigure}
%
  \begin{subfigure}[t]{0.3\textwidth}
    %\centering
  \includegraphics[width=0.66\textwidth, center]{./img/multiLaser-manuf-bed.jpg}
  \caption{Printing bed: Adding material}%
  \label{fig:cylCross-multi-manuf-bed}
\end{subfigure}
%
  \begin{subfigure}[t]{0.3\textwidth}
    \raggedright%
  \includegraphics[width=0.8\textwidth, center]{./img/cylCross-multi-manuf-bed2-copy.png}
    %\centering
  \caption{Printing bed: Laser actuation}%
  \label{fig:cylCross-multi-manuf-bed-2}
\end{subfigure}
  %
  \begin{subfigure}[t]{1.0\textwidth}
  \centering
    \includegraphics[width=0.65\textwidth]{./img/cylCross-multi-manuf-end.png}
  \caption{Master final state}%
  \label{fig:cylCross-multi-manuf-end}
  \end{subfigure}
  %
  \begin{subfigure}[t]{1.0\textwidth}
  \centering
    \includegraphics[width=0.65\textwidth]{./img/cylCross-multi-manuf-slave.png}
  \caption{Slave final state}%
  \label{fig:cylCross-multi-manuf-slave}
  \end{subfigure}
  % 
  \caption{Multi-laser --- bi-material manufacturing test: Manufacturing procedure}%
  \label{fig:cylCross-multi-manuf-procedure}
\end{figure}

Fig.~\ref{fig:cylCross-multi-produced-part} shows the bi-material part produced,
where it is clearly visible the defects in the polymeric manufacturing.
%
\begin{figure}[hbtp!]
  \centering
    \includegraphics[width=0.4\textwidth]{./img/cylCross-multi-part.jpg}
%
  \caption{Multi-laser --- bi-material manufacturing test:
    Produced part}%
  \label{fig:cylCross-multi-produced-part}
\end{figure}


\subsection{Post-Manufacturing}%
\label{sec:post-manufacturing-multiLaser}
Even though the part was not completely manufactured, it is important to
document and analyse the reasons behind this.
In that sense, several analysis were performed on the
produced part to assess the manufacturing quality, namely on geometrical
compliance and densification.

Geometrical compliance was assessed, once again, using \texttt{Fiji}~\cite{fijiImageJ} and using a 1 cent coin as a
base for the measurements (Fig.~\ref{fig:cylCross-co2-imagej}).
The measurements
performed showed a higher discrepancy, which can be due to the laser focus
distortion induced by the tilting. This must be further investigated and
corrected.
Nonetheless, overall, the geometry of the produced part matched the one from the
3D \gls{cad} model, but with a discrepancy in the dimensions.
% subfigures

\begin{figure}[htbp!]
  \centering
  %
  \includegraphics[width=0.6\textwidth]{./img/cylCross-multi-imagej.png}%
  \caption{Multi-laser --- bi-material manufacturing test:
    Geometrical measurements using Fiji's SW}%
  \label{fig:cylCross-multi-imagej}
\end{figure}

The \gls{em} analysis performed on the part
(Fig.~\ref{fig:cylCross-multi-EM-analysis}) showed poor densification, further supporting the
decision to halt the manufacturing. 

%
% subfigures
\begin{figure}[htb!]
  \centering
  %
  \begin{subfigure}[t]{.48\textwidth}
    \includegraphics[width=1.0\textwidth]{./img/cylCross-multi-em1.jpg}%
  %\caption{\gls{gui}}%
  \label{fig:cylCross-multi-EM1}
  \end{subfigure}
%
  \begin{subfigure}[t]{.48\textwidth}
    \includegraphics[width=1.0\textwidth]{./img/cylCross-multi-em2.jpg}%
  %\caption{Machine bed}%
  \label{fig:cylCross-multi-EM2}
  \end{subfigure}
  % 
  %
%  \begin{subfigure}[t]{.48\textwidth}
%    \includegraphics[width=1.0\textwidth]{./img/cylCross-multi-em3.jpg}%
%  %\caption{\gls{gui}}%
%  \label{fig:cylCross-multi-EM3}
%\end{subfigure}
  %
  \caption{Multi-laser --- bi-material manufacturing test:
    EM analysis}%
  \label{fig:cylCross-multi-EM-analysis}
\end{figure}

The information concerning the manufacturing process as a whole was then
collected and added to the \emph{Post\=/Manufacturer}.
Fig.~\ref{fig:postManuf-premanuf-multi} shows the \underline{Pre-Manufacturer}'s data
collection, storing the input and output files' related information,
respectively. It allows quick navigation and visualisation of these information
flows, which are indexed to a produced part.

% subfigures
\begin{figure}[htb!]
  \centering
  %
  \begin{subfigure}[t]{.8\textwidth}
    \includegraphics[width=1.0\textwidth]{./img/postManuf-premanuf-input-multi.png}%
  \caption{Pre-Manufacturer input files manager}%
  \label{fig:postManuf-premanuf-input-multi}
  \end{subfigure}
%
  \begin{subfigure}[t]{.8\textwidth}
    \includegraphics[width=1.0\textwidth]{./img/postManuf-premanuf-output-multi.png}%
  \caption{Pre-Manufacturer output files manager}%
  \label{fig:postManuf-premanuf-output-multi}
\end{subfigure}
%
  \caption{Multi-laser --- Post-Manufacturer: Pre-Manufacturer
    data management}%
  \label{fig:postManuf-premanuf-multi}
\end{figure}

Fig.~\ref{fig:postManuf-manuf-multi} shows the \underline{Manufacturer}'s data
collection, storing the input, configuration and logs' related information,
respectively. It allows quick navigation and visualisation of these information
flows, which are also indexed to a produced part. The input and configuration
data can be used in conjunction with the mechanical tests to analyse the
manufacturing performance and behavior. On the other hand, the logs can be used
by the control/systems engineer to analyse the equipment and software behavior,
and improve it.

% subfigures
\begin{figure}[htb!]
  \centering
  %
  \begin{subfigure}[t]{.6\textwidth}
    \includegraphics[width=1.0\textwidth]{./img/postManuf-manuf-layers-multi.png}%
  \caption{Manufacturer input manager}%
  \label{fig:postManuf-manuf-layers-multi}
  \end{subfigure}
%
  \begin{subfigure}[t]{.6\textwidth}
    \includegraphics[width=1.0\textwidth]{./img/postManuf-manuf-pens-multi.png}%
  \caption{Manufacturer configuration manager}%
  \label{fig:postManuf-manuf-pens-multi}
\end{subfigure}
%
  \begin{subfigure}[t]{.6\textwidth}
    \includegraphics[width=1.0\textwidth]{./img/postManuf-manuf-logs-co2.png}%
  \caption{Manufacturer logs manager}%
  \label{fig:postManuf-manuf-logs-multi}
\end{subfigure}
%
  \caption{Multi-laser --- Post-Manufacturer: Manufacturer
    data management}%
  \label{fig:postManuf-manuf-multi}
\end{figure}

Fig.~\ref{fig:postManuf-mechTestManager-multi} shows the electron microscope image
analysis performed on the component added to the database.
Once again, data is appended, contributing to the increasing process
knowledge base.

\begin{figure}[hbtp!]
  \centering
    \includegraphics[width=1.0\textwidth]{./img/postManuf-mechTestManager-multi.png}
  %A
  \caption{Multi-laser --- Post-Manufacturer: Mechanical tests manager}%
  \label{fig:postManuf-mechTestManager-multi}
\end{figure}




%\subsubsection{Discussion}
The equipment and manufacturing tests performed clearly demonstrates the
feasibility of \gls{3dmmlpbf} process and validates the equipment developed, as
well as the accompanying toolchain.



\section{3DMMLPBF Improvement}%
\label{sec:prosp-proc-impr}
The \texttt{Post-Manufacturer}'s software provides a knowledge base for
\gls{3dmmlpbf} process analysis and improvement in a myriad of ways. One such
example is illustrated in Fig.~\ref{fig:post-manuf-ai-application}, where
\gls{ai} is applied to the analysis of the \gls{sem} images that enable the
assessment of the
mechanical structure and properties of the produced parts.
It should be noted that this is only a possible venue, as tools and
methodologies may differ.

\begin{figure}[hbtp!]
  \centering
    \includegraphics[width=1.0\textwidth]{./img/post-manuf-ai-example.pdf}
  %A
  \caption{AI application for process improvement: \gls{sem} image analysis
    example}%
  \label{fig:post-manuf-ai-application}
\end{figure}

In this example, the \gls{sem} analysis is performed through deep learning using
\gls{cnn}. The conventional image processing methods,
such as traditional segmentation, threshold method, or watershed segmentation,
rely on the similarity or intensity (or both) of the image's pixels to locate or
delineate the boundaries of objects~\cite{chen2020DeepLearningSEM}. Thus, when
image's pixels lack intensity or contrast it becomes extremely difficult to
distinguish boundaries, even for trained professionals. 
On the other hand, deep learning tries to mimic the human expert's analysis. By
building computational models that can learn
representations of data without such requirements, they can be used to
perform tasks as efficiently as human experts in specialized fields, saving time
and money~\cite{chen2020DeepLearningSEM}. For this purpose, deep learning
network architecture comprises multiple layers of artificial neural net
units --- convolutional neural networks~\cite{chen2020DeepLearningSEM}.

One paradigmatic example of a fully convolutional network architecture is the
\emph{U-Net}, initially developed in 2014 for biomedical image
segmentation~\cite{Ronneberger2015UNet}. It provides more precise segmentation
maps using fewer training images~\cite{Ronneberger2015UNet}.
As machine-learning may still be a daunting subject, in 2019 Falk and Mai~\cite{Falk2019UNet}
designed a plugin module to the image processing software \texttt{ImageJ} that
enables non-expert personnal to design their own training model with the U-Net.

Since then the U-Net has been
used for segmenting image data of non-medical materials. For example, Chen
et.~al~\cite{chen2020DeepLearningSEM} presented a strategy to segment clay
particles from matrix mineral grains in \gls{sem} images of shale samples using
the U-Net architecture with a revised weighting algorithm, where no obvious grayscale contrast can be used to differentiate between the two interlocked minerals.

Based on these considerations, Fig.~\ref{fig:post-manuf-ai-application}
illustrates a possible \gls{ai} application to \gls{sem} images's analysis using
the \texttt{U-Net} deep learning \gls{cnn} architecture and the \texttt{U-Net}
\texttt{ImageJ} plugin. The plugin is the front-end, interacting directly with the \texttt{User}, while the deep learning
neural network runs on the back-end for performance reasons. These components
use the client-server model, connected by a \gls{tcp-ip} link. The plugin runs on
any main \gls{os} (Linux, MacOs, Windows), while the back-end may run on a
cloud-service or even a Linux workstation.

The \texttt{Post-Manufacturer} stores the \gls{sem}
images associated to the mechanical tests performed on the manufactured
components. These images can then be loaded into the \texttt{U-Net}
\texttt{ImageJ} plugin for pre-processing. For example, image annotation may be
required to identify relevant areas to account for or to ignore when the neural
network is learning. Then, the neural network is setup initially for training,
providing the neural network model, the weights, and the training data. The
training yields the trained model, which can now be used to segment or detect
features in the pre-processed images. After processing, the output image is
analysed and validated. If valid, it is stored in the \texttt{Post-Manufacturer}
database (see also Fig.~\ref{fig:postManuf-mechTestManager}). Otherwise, the model is fine-tuned until significant accuracy is
attained.

Another interesting option is the addition of more manufacturing data
collection, which could be used on-line and in real-time --- for adaptative control of the
process --- or off-line --- for structured improvement of the process's control
through heuristics or new control models sustained by \gls{ai} data mining. For
example, the inclusion of infrared thermal cameras or \gls{lsp}. The former
allows to assess the stability of the process, but not on the material
condition. On the other hand, \gls{lsp} analyses the change in the interference
patterns of the laser (speckles) to identify inhomonogeneities in 3D printing
and even ``invisible'' defects~\cite{chen2019laser}.
Obviously, this requires a significant investment and resources, but could
greatly benefit the manufacturing quality.

\section{Results}%
\label{sec:results}
The manufacturing tests performed validated the \gls{3dmmlpbf}'s methodology
devised. The \gls{3dmmlpbf} equipment was successfully tested and
deployed, serving as a means to produce multi-material components via laser
powder bed fusion process, thanks to the devised \gls{3dmmlpbf} methodology, the
instantiated workflow and the supporting toolchain. Furthermore, the
manufacturing is highly customizable and extensible due to the open source
nature of the toolchain, acting as a ramp for fast prototyping.

However, the manufacturing tests also suggested that more investigation is
required for attaining a reliable manufacturing quality, especially when
handling dissimilar types of materials, such as polymeric and metal alloys. The
scope of action can be varied, as different layer heights or path topologies can
be needed --- pre\=/manufacturing --- or different process parameters, such as
laser related, powder bed's temperature, shielding atmosphere control --- manufacturing.
This highlights the importance of the \texttt{Post\=/Manufacturer} software,
storing and tracking the different information flows, enabling quick analysis and iteration.

multi-material fabrication using the \gls{3dmmlpbf} process can also leverage
from the usage
of multiple lasers for different material types, in a scalable and reliable
architecture, although contingent to the proper arrangement of the lasers' set
over the printing bed. Using lasers with different beam generation systems, such
as solid-state, can significantly reduce its volume, and facilitate this
integration.


%
%\subsection{Discussion}
%The workflow tests performed over the toolchain, helped to improve the respective tools by feeding
%back relevant information.
%More importantly, these tests allowed to validate the
%toolchain:
%\begin{itemize}
%\item \underline{Pre-Manufacturer}: the slicer and path generator are
%  capable of slicing and generating paths for various 3D models with different
%  fill angle, fill density, infill extrusion width, layer height, and number of
%  materials. The slicer is agnostic about the input files. It was also seen that
%  a significant number of path topologies are available and that the fill
%  density and infill extrusion width can be varied to mimic the required path
%  filling for laser trajectories.
%  Due to the high number of path topologies available off-the-shelf and the
%  possible adaptation from the 3D printing area to the \gls{lpbf} one, fast
%  iteration on part production is possible. Lastly, due to the open source
%  nature of the slicer and path generator, the modification of the available
%  path topologies or the addition of new ones is relatively straightforward.
%\item \underline{Manufacturer}: the post-processor is capable of
%  processing the geometric and material data in the manufacturing file and
%  mapping them to the desired processing parameters, irrespective of the
%  material, layer height and layer number. The printer was successfully tested
%  on offline mode to produce the part. Additionally, it was also seen the tight
%  coupling between post-processor and printer, signalled by the restriction on
%  the production of parts with more than two materials.
%\item \underline{Post-Manufacturer}: the process information can be
%conveniently and correctly added to the process knowledge database for
%management and visualisation. The database can be exported separately to provide
%the relevant information for each agent individually. This approach also allows
%for tests reproducibility and for fast process bootstrapping: if a normal
%\texttt{User} wants to start using the equipment following the \texttt{3dmmlpbf}
%methodology, it can source all the toolchain, the models and the process
%parameters (via database exporting) to reproduce it or test it. Then, the
%\texttt{User} can choose to improve the database by sharing it with other users,
%which can then import those files, leveraging from the collaborative and
%open-source approach to boost the \gls{3dmmlpbf} enviroment.
%\end{itemize}


\section{Summary}
In this chapter the \gls{3dmmlpbf}'s manufacturing was tested, and by extension
also the whole ecosystem --- methodology, workflow, toolchain, and equipment.

The workflow and accompanying toolchain proved itself capable of: handling various 3D models ---
irrespective of the number of materials, layer height, path topology and layer
number; supporting different materials processing and with multiple lasers; and
conveniently storing all process related information for posterior analysis and
process improvement.

A manual and naive procedure for the manufacturing file generation was also used
--- in the YAG-Nd laser testing ---
to further illustrate the difficulty of generating such files without the
appropriate tooling. Even though it may serve as a quick workaround for testing
some geometries, it is extremely complex to apply this to functional design, as
it would require the \texttt{Designer} to think in a 2D fashion, which defeats
the purpose of using a 3D multi-material fabrication process.

The modifications performed on the slicer and path generator enabled the 
adaptation of 3D paths topologies for the \gls{3dmmlpbf} process. This paves the
way for quick iteration of the different path topologies and the associated
parameters to assess its impact on the overall manufacturing quality of the
produced component.

The \texttt{Pre-Manufacturer} slices and generates the manufacturing paths for each input
model (material) in compliance to the user-defined configuration and merges
them, yielding a 3D multi-material manufacturing file.

This file is then loaded into the \texttt{Manufacturer}'s \gls{sw},
and the manufacturing is configured by mapping the process parameters to each
layer and assigning it to one or multiple lasers simultaneously. This \gls{sw}
efficiently handles the manufacturing process, controlling the equipment and the laser network.
Furthermore, the generated data streams ---
layers, configuration, and logs --- can be easily exported for storing and analysis.

The \texttt{Post-Manufacturer}
allows the \texttt{User} to manage the process knowledge database by importing
manufacturing chain files (models, manufacturing output, mechanical tests) and to export database,
enabling the widespread of the \gls{3dmmlpbf} technology through bootstrapping.

The manufacturing tests yielded varied results. The polymeric bi-material
(CO\textsubscript{2} laser) showed good densification and geometrical
compliance to the 3D model. On the other hand, the metallic alloy bi-material and
polymeric/metallic alloy bi-material components, showed mild to poor
densification and discrepancies in geometrical compliance. These unsatisfactory
results mandate further investigation on the topic, especially on the
parametrization of the process.

Lastly, some venues were presented for \gls{3dmmlpbf}'s improvement, utilising
the collected data by the Post\=/Manufacturer application, namely
manufacturing quality assessment off-line --- using \gls{sem} analysis supported by
deep learning \gls{cnn} --- manufacturing quality assesment on-line and
adaptative control --- using infrared thermography or
\gls{lsp}.

%%% Local Variables:
%%% mode: latex
%%% TeX-master: "../template"
%%% End:
