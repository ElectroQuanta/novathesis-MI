%!TEX root = ../template.tex
%%%%%%%%%%%%%%%%%%%%%%%%%%%%%%%%%%%%%%%%%%%%%%%%%%%%%%%%%%%%%%%%%%%%
%% abstrac-en.tex
%% NOVA thesis document file
%%
%% Abstract in English([^%]*)
%%%%%%%%%%%%%%%%%%%%%%%%%%%%%%%%%%%%%%%%%%%%%%%%%%%%%%%%%%%%%%%%%%%%

\typeout{NT FILE abstrac-en.tex}%
%
Functional design is a highly desirable approach to design, optimising the
performance of a product while minimising the resources usage and
cost.
However, the adoption of functional design may dictate the use of multiple
materials or a combination of them, which is hindered by the current manufacturing methodologies. Biomedical implants, such as hip implants, represent a class of products where functional design is critical.

Currently, only a handful of commercial equipments is capable of multi-material
fabrication~\cite{aconityMIDI, slm280}. 
Moreover, they require a significant
initial investment and are only suited for metals, which limits the scope of
applications.
Consequently, research primarily focuses on upgrading mono-material equipment to
add multi-material capabilities~\cite{bareth2022Implem, schneck2022capability,
  nadimpali2019MMSteels, anstaett2017fabrication, sing2015interfacial,
  liu2014interfacial},  which undesirably also requires manual workarounds to
setup the manufacturing chain and equipments.
This is further aggravated by the lack of specific design guidelines for the 
\gls{mmlpbf} process and by the fact that most pre-processing procedures use a
manual method based on an inadequate data format~\cite{schneck2021review}.
%
The main takeway is that the \gls{mmlpbf} fabrication is a multi-objective
problem, requiring specially designed equipments and toolchains, and
experimental design.

Thus, the present work aims to close the gap between design and fabrication of
multi-material components like the aforementioned implants by proposing a holistic
approach to the multi-material fabrication of metals/ceramics that can leverage
the process knowledge and support functional design. A model-based methodology
was devised to address the high complexity associated with the design and manufacturing of multi-material parts and fill the
gap in this domain. 
The knowledge acquired through the relevant models was used for the
instantiation of a specialised workflow for the \gls{mmlpbf} process and the
development of a supporting toolchain. Then, a custom equipment was developed that
integrates multiple lasers
of different types, which can be effectively used for the fabrication of
multi-material components using metallic and ceramic powders. To the best of the
author's knowledge, this is the first equipment with this feature.

Several multi-material components were designed and manufactured following the
devised methodology. These tests validated the whole ecosystem, demonstrating
its suitability to support the functional design of multi-material components
using the \gls{mmlpbf} process.
Lastly, some prospects to leverage the process knowledge database created are
presented, where tools like \gls{ai} can be used for straightforward and fast
improvement of the \gls{mmlpbf} process.

% Palavras-chave do resumo em Inglês
\begin{keywords}
functional design, 3D multi-material laser powder bed fusion, 3D multi-material fabrication, methodology,
software toolchain, low-cost equipment
\end{keywords} 
