% !TEX root = ../template.tex
%
\typeout{NT FILE STATE-ART.tex}%
\chapter{State of the art}%
\label{ch:state-art}
This chapter provides a comprehensive review of the current state of the art of
multi-material fabrication for metals and ceramics. The functional design
approach is introduced, followed by a discussion on the use of laser-based
additive manufacturing processes as a viable solution for metallic and composite
manufacturing, with particular attention given to the \glsxtrfull{lpbf} process. To
fully support functional design, the \glsxtrfull{mmlpbf} process is introduced,
which bridges the gap between \gls{lpbf} processes. An overview of \gls{mmlpbf}
is presented, including the current panorama, manufacturing chain, challenges,
and potential solutions. Finally, the specific applications envisioned in this work are listed.

\section{Functional design}%
\label{sec:functional-design}
Functional design is a design approach that aims to optimise the functional
performance of a product while minimizing its cost and complexity. It is a
concept that transcends various fields --- biomedical~\cite{yang2017biomimetic,
  wang2016dual, el2013hydrogel}, aerospace~\cite{marques2022RocketEngine, zhu2018light, hu2017design}, food~\cite{portanguen2019toward},
chemistry~\cite{jin2023structural}, and even clothing~\cite{gupta2011design,mckinney2023inclusion}.
--- and is not limited to a specific
technology.

Functional design draws inspiration from nature, with its wide spectrum of
lightweight and functional structures, such as branched plant shapes, bone
tissues, and honeycomb patterns~\cite{anton2019-ReviewBiomimetic-AM}.
Historically, humans tried to replicate this design, but simplifications must be
made to accommodate manufacturing and assembly constraints. Nonetheless, the
fundamental premise is nature-inspired: how to design and manufacture components
in a sustainable way with an efficient usage of resources.

Thus, and although the concept of functional design is not new~\cite{chiang2001designing}, it has become
increasingly feasible with the emergence and evolution of technologies such as additive manufacturing~\cite{anton2019-ReviewBiomimetic-AM}, enabling designers
to create complex structures and functional parts that were previously difficult
or impossible to produce using traditional manufacturing techniques.
As a result, a paradigm shift in
product design and manufacturing is in motion, allowing for the exploration of
features, shapes, and geometries to comply with the required functionalities,
withoug being limited by production constraints~\cite{plocher2019review}.

\gls{am} was the main driver for functional design by showcasing how material
could be added to specific areas of a component to fulfil a function, which
would be otherwise impossible to produce using the traditional subtractive
methods, like \gls{cnc} machining.
Furthermore, the number and diversity of the \gls{am} technologies contributed
to functional design in a myriad of ways, and can be primarily categorized into
four distinct groups~\cite{leung2019challenges}: multi-material, multiscale, multiform, and multifunctional.

The first category, multi-material, aims to enhance the
mechanical or material properties of an object by depositing multiple dissimilar
materials within a single entity. This is required for components with
interface with specific properties. For example, in the hip
implant, the core material must provide the load bearing capabilities, but the
external interface should be composed of bioactive materials to promote
osseointegration (e.g., bioceramic)~\cite{rafiq2023MultiFunc-coatings-hip}.

The second category, multiscale, focuses on
the design of geometric features in various scales to achieve functional
requirements. For example, the wettability of an object is changed from
hydrophilic to super-hydrophobic when the surfaces are embedded with micropillars~\cite{yang20183d}.

The third category, multiform, is aimed at designing objects with
programmable shape-changing properties after fabrication. This means that the
shape of the component will evolve according to an external stimulus, e.g.
thermal~\cite{khoo20153d,chae2015four}, light~\cite{liu2017sequential}, or moisture~\cite{sydney2016biomimetic}.

Lastly, the fourth category, multifunctional, pertains to the design of objects
such as multifunctional flexible sensors, and electronics, 
where the primary focus is on non-structural properties, despite the objects still possessing
structural properties. These non-structural properties include sensing~\cite{senthil2019review},
thermal~\cite{tirado2021conductive, fornells2020integrated},
hydrodynamic~\cite{liu20223d}, biologic~\cite{freyman20203d}, and
electric~\cite{wajahat2022three, fu2020functional}.

The limitations of \gls{am} must be emphasized, particularly with respect to
product quality and productivity, system efficiency, and sustainability. AM
processes are constrained by certain manufacturing restrictions, such as the
presence of sharp corners, thin geometries requiring support structures, and
height errors that result in poor surface finish~\cite{danish2022experimental}. In contrast, conventional
\gls{sm} techniques, such as turning and milling, can yield parts with a
superior surface finish but exhibit low throughput in producing highly complex
geometries. Consequently, a \gls{hm} approach combining \gls{am} and \gls{sm} techniques is advocated as an optimal solution, as it can produce components with intricate geometries and the desired surface finish while minimizing both setup time and waste generation~\cite{hegab2023design, lauwers2014hybrid}.

It can easily be recognised that the extra freedom provided by hybrid
manufacturing suits better the purpose of functional design.
However, this comes at the cost of even greater complexity, especially if
multi-material applications are considered.
Furthermore,
as \gls{hm} is a superset of \gls{am}, it is expected that the improvements on
the latter will cascade to the former.

Thus, the present aims to approach functional design through the lens of
\gls{am} processes. In particular, the investigation centers on the application
of multi-material functional design utilising \gls{am} technology, which is
essential for the development of biomedical implants, such as the hip implant.

\section{Mono-material Laser-Powder Bed Fusion}%
\label{sec:monom-laser-powd}

%\subsection{Additive Manufacturing}%
%\label{sec:am}
The basic concept of additive manufacturing (AM) involves the addition and
bonding of material solely in the necessary regions to create the desired
component, typically in a layer-by-layer approach through \gls{cnc} displacement
using three-dimensional (3D) model data~\cite{thompson2015overview}. As each
layer of material is added, the 3D part is incrementally constructed by bonding
similar or dissimilar materials. To enable the manufacturing paths to be
generated, which dictate the CNC displacement, the 3D models are typically
represented in \gls{cad} form in the \gls{stl} file format and numerically
sliced into numerous virtual layers or cross-sectional data.
A wide variety of \gls{am} application have been reported namely:
aerospace~\cite{marques2022RocketEngine, hu2017design, boeing20173DPrinting},
\glspl{uav}~\cite{bronz2020mission, tadjdeh2014navy}, houses~\cite{stott2014chinese},
tooling~\cite{rannar2007efficient, dalgarno2001production}, biomedical
implants~\cite{bartolomeu2020additive, esmaeili2019artificial,cansiz2016computer}, among others.

To accomplish the effective material bonding, the successful combination of
material and energy delivery is required, differing with the material and the \gls{am}
process\cite{thompson2015overview,mccann2021situ}.
The \gls{am} processes can be classified by
~\cite{vayre2012metallic}:
\begin{itemize}
  \item \emph{state of raw material}: liquid, solid sheet or discrete particle;
  \item \emph{type of material}: metal (layer or direct deposition); polymer
    (\gls{fdm}, stereolithography, polyjet); paper (\gls{lom}; wood
    stratoconception).
\end{itemize}
The following methods are classified as \gls{am}
ones~\cite{thompson2015overview, sefene2022ham}: material extrusion, material jetting, sheet
lamination, vat polymerisation, binder jetting, \gls{ded} and
\gls{pbf}. Fig.~\ref{fig:am-proc} depicts these seven categories and the related
technologies with energy sources, materials used, advantages and drawbacks, and
their suitability to produce metallic parts~\cite{sefene2022ham}. The processes
delimited in red are often used for metallic parts.
%
\begin{figure}[!hbtp]
  \centering
    \includegraphics[width=0.9\textwidth]{./img/am-processes.png}%
    \caption[Additive manufacturing processes]{AM processes (withdrawn from~\cite{sefene2022ham})\footnotemark}%
      %\fnref{foot:cc-lic}}%
      %\textsuperscript{\ref{foot:cc-lic}}%
    \label{fig:am-proc}
\end{figure}
%
\fnlicReq{Springer}{5296320585850}

The \gls{am} started out from being a rapid prototyping tool, enabling fast
iteration over design at a reduced cost --- reducing the development cycle, and
providing resource efficiency~\cite{guo2013additive}. This remained the case for long, due to
production times being higher than for conventional methods, and to the
generally lower surface quality finishing, so it was
commonly cast aside for anything else than as prototyping tool. However,
\gls{am} provides highly desirable features for the production of high-value
products, where time constraints are not so relevant, thus justifying a strong
investment in its evolution.
This process is traditionally `open-loop' due to its lower complexity
and lower cost; however feedback control was introduced to ensure better
part quality, in some cases with real-time
characteristics~\cite{radel2019skeleton, garanger20183d,
  purtonen2014monitoring}.

%\subsection{Laser-Powder Bed Fusion}%
%\label{sec:lpbf}
When it comes to the additive manufacturing of metallic parts and composites
\gls{ded} and \gls{pbf} are the most proven and feasible
methods~\cite{thompson2015overview, vaezi2013multiple}.
%
Both processes involve the deposition of powder metal (or less common preforms
such as wire) and their simultaneous or subsequent melting, respectively, via a
focused thermal energy source, namely an electron beam or a laser beam.
In case a laser beam is used the processes can be referred as form of
\gls{lpbf}, while \gls{ded} can be further specified as
\gls{dld}~\cite{thompson2015overview}. The usage of an electron beam (\gls{ebm}) makes high scanning speed possible (up to several
km/s) due to the lack of moving parts to guide the building spot~\cite{vayre2012metallic}; however, the increased complexity and cost does not
make it commercially viable yet.

Besides being easier to use, laser sources are attractive for numerous other
reasons~\cite{yang2016role}: small spot diameter, minimising the molten pol and the surrounding area
(heat-affected zone); high energy density; accurate control of the energy flow.
Thus, they are widely used in \gls{am} processes, especially concerning metallic
materials with high melting points~\cite{collins2016microstructural}.

Comparing both laser-based technologies, \gls{lpbf} offers the following
advantages over \gls{dld}: smaller spot diameter, providing better processing
accuracy and a smaller
melt pool, thus yielding smaller surface roughness and greater geometric
resolution. For the opposite reasons, \gls{dld} can process larger parts,
providing the range of deposition head is large
enough~\cite{wei2020overview}. This can be extended to the multi-material
domain, with \gls{lpbf} being better suited to process high-precision small and
medium-size multi-metal parts~\cite{wei2020overview}. Thus, the \gls{lpbf}
process arises as the current bet for commercial and industrial applications for
the fabrication of high-precision multi-material parts using metals and
ceramics~\cite{aconityMachinesSite, slm500, eosM300, trumpfTruprint500,
  auroraLabsRMP1, mussatto2022research}.

\gls{lpbf} uses a focused laser beam to selectively melt (\gls{slm}) or to
coalesce (\gls{sls}) metallic or composite powders, layer-by-layer, to build 3D components.
As shown in Fig.~\ref{fig:sls-slm-process}, the powder is fed and
uniformly distributed on the powder bed using a spreading mechanism, like a
roller. The laser beam melts specifics areas of the powder bed, accordingly to
the laser scanning pattern. After a layer is completed, the powder bed lowers by
the height of the deposited layer, a new layer is deposited on the powder bed
and the process is repeated. Depending on the energy density of the laser beam
and the materials used, the material can be fully melted rather than sintered, allowing different properties (crystal structure, porosity, etc.)

This repetitive process results in unmelted excess metal, providing structural
support and additional protection from oxidation and thermal
stress, ideal for overhanging structures. Thus, the 
the powder bed is also usually heated to diminish temperature gradients which can weaken the part, leading to its collapse or fracture~\cite{thompson2015overview}.

The metallic powders are highly susceptible to oxidation, which is further
aggravated by the high thermal gradients induced by laser operation and the
powder bed heating. Thus, to minimise this effect, an enclosing and controlled
shielding system with an inert gas, typically argon or nitrogen, is used.

The part is generally built upon a substrate
plate to prevent powder bed platform damage, which must be removed to obtain the
finished part upon process completion.
%
\begin{figure}[!hbtp]
  \centering
 \includegraphics[width=0.6\textwidth]{./img/sls-slm-process.png} 
 \caption[LPBF process]{\gls{lpbf} process (withdrawn from~\cite{thompson2015overview})\footnotemark}%
\label{fig:sls-slm-process}
\end{figure}
%
\fnlicReq{Elsevier}{5296180869039}%

Figure~\ref{fig:lpbf-params} delineates the key processing parameters for the
\gls{lpbf} process. The thermal history, residual
stress, and microstructure of a fabricated part are profoundly influenced by the
laser processing parameters, ultimately impacting its
quality~\cite{pratt2008residual}. Laser power, laser scanning speed, hatch
distance, and energy density constitute the most notable parameters, affecting
the geometry of melt pools, thermal gradients, and cooling rates. The energy
density is the primary driver of melting and can be achieved through various
combinations of the aforementioned parameters~\cite{sing2016selective}. This
makes the process of determining optimal parameters a multi-objective
optimization problem. Scanning strategy is the primary driver of energy density
and can be leveraged to optimize surface roughness, refine microstructure, and
reduce porosity~\cite{yu2019influence,zou2020numerical}, while mitigating
residual stresses, thermal deformation, and
defects~\cite{kruth2004selective,cheng2016stress,wan2018effect}.

%
\begin{figure}[!hbtp]
  \centering
    \includegraphics[width=1.0\textwidth]{./img/mmlpbf-params.jpg}%
    \caption[LPBF process and key parameters]{\gls{lpbf} process and key
      parameters\footnotemark: white fonts indicate machine components while the black fonts
      illustrate the key processing parameters (withdrawn from~\cite{sing2021emerging})}%
      %\fnref{foot:cc-lic}}%
      %\textsuperscript{\ref{foot:cc-lic}}%
    \label{fig:lpbf-params}
\end{figure}
%
\fnlicReq{Elsevier}{5296470787752}

Zou et al.~conducted a numerical simulation to analyse the scanning strategy of single and multi-laser scanning for residual stress\cite{zou2020numerical}. They discovered that due to higher energy heat input, the residual stress of a sample increases with the number of lasers used, but careful planning of scanning sequence and direction can mitigate this. Moreover, they found that the two-scanning strategy can reduce residual stresses by more than 10\% when compared to the single-scanning strategy. This re-melting can improve surface roughness and porosity~\cite{yu2019influence}.

\section{Multi-material Laser-Powder Bed Fusion}%
\label{sec:mult-laser-powd}
The capability to fabricate multiple material parts is highly desirable as it
allows for the accurate placement of material according to its functionality,
providing custom-tailored parts for specific applications with enhancement of
its mechanical properties and behaviour in service.
However, the existing 3D printing techniques are mainly for
mono-material part fabrication~\cite{wei2020overview}.
The emerging \gls{mmam} technology can
enhance the AM parts performance by varying compositions or type within layers,
unachievable by conventional manufacturing processes~\cite{vaezi2013multiple}, 
without the need for complex manufacturing process and expensive
tooling~\cite{han2020recent,bandyopadhyay2018additive}.

The range of applications are vast and pivotal. In the biomedical engineering
field, \gls{mmam} enabled the production of 3D engineered tissue (3D spinal
cord~\cite{joung20183d}), biomedical devices such as microneedle
arrays~\cite{lu2015microstereolithography} and diagnostic
devices~\cite{li2018multimaterial}, multi-material cellular structures targeting
orthopedic implants~\cite{bartolomeu2020additive}, and
3D artificial models for preclinical or preoperative surgical
training~\cite{coelho2018multimaterial,cresswell2018approaches}, among others.
%
In the soft robotics field, where flexibility is key for complex actuations and
motions, \gls{mmam} enabled the production of pneumatically driven elastomeric
actuators~\cite{sydney2016biomimetic} and direct integration of functional components required for it (e.g.,
a silver-nanoparticle ink acting acting as a resistive heating
element~\cite{yuan20173d}).
%
In electronics \gls{mmam} is critical for direct manufacturing of 3D electronic
devices where electrically dissimilar materials including conductors,
semiconductors, and dielectrics are integrated
together~\cite{han2020recent}. Some examples are a 3D magnetic sensor with
integrated electronics components and conductive paths~\cite{espalin20143d}, stretchable strain or
pressure sensors~\cite{valentine2017hybrid} and a highly stretchable electronic LED board~\cite{huangagoh2018highly}, yielding high
potential for wearable electronics, and even an fully 3D printed and package
Li-ion battery~\cite{wei20183d}. 

To achieve this superior performance over \gls{am}, different materials or chemicals need to be physically
delivered to any point in the 3D space during the additive manufacturing. In
some processes, like direct 3D printing in Objet, \gls{fdm}, this is
relatively straight-forward to achieve as the materials are deposited in the
platform dot-by-dot or line-by-line via nozzles; to incorporate multi-material
fabrication multiple nozzles can be added~\cite{vaezi2013multiple}.

For multi-material fabrication of metals, a similar result could be achieved
through the use of LENS process or \gls{dld}, as they can use multiple
nozzles/hoppers in the part fabrication. For example, multi-material components
manufactured by \gls{lmd} has been demonstrated in literature~\cite{brueckner2018enhanced,shah2014parametric}.
However, in other processes, like \gls{lpbf}, and \gls{lom}, this is not trivial, as the materials are delivered as
whole layer by a scraper or as a solid sheet, requiring new material delivery
systems to be first developed~\cite{vaezi2013multiple}.

Nonetheless, \gls{lpbf}
provides higher precision, smaller feature size and the ability to produce
lightweight structures based on lattices, such as turbine blades, which cannot be easily achieved by
\gls{lmd}~\cite{liu2019additive,bartolomeu2021selective}.
On top of that, \gls{dld} is a
more difficult process to master due to added complexity of deposition control,
on top of the melt-pool control, which can cause variations in the laser spot
due to local increase of part's height as a result of the
deposition~\cite{shamsaei2015overview}.

Thus, \gls{lpbf} is the most well-suited option for the manufacturing of
multi-material components using metals and ceramics, which was further
demonstrated by several works~\cite{walker2022multi, anstaett2017fabrication,
  sing2015interfacial, LIU2014116}.

\subsection{Technological Overview}%
\label{sec:overview}
This section presents the technological challenges and the possible solutions
to tackle \gls{mmlpbf} current drawbacks.

Fig.~\ref{fig:mmlpbf-mach-stateArt} illustrates the overview of a \gls{mmlpbf}
system based on the commercial equipments available and the scientific
literature~\cite{aconityMachinesSite, slm500, eosM300, renishawAM500,
  trumpfTruprint500, auroraLabsRMP1, mussatto2022research}. The dashed
components are optional.
Upon careful examination of Fig.~\ref{fig:mmlpbf-mach-stateArt}, numerous
resemblances with the traditional \gls{lpbf} procedure
portrayed in Fig.~\ref{fig:sls-slm-process} can be observed. Such parallels
should be expected since the most straightforward technique for manufacturing
multi-material components using \gls{lpbf} involves modifying the powder delivery
system of a pre-existing commercial equipment~\cite{bareth2022Implem,
  schneck2022capability, nadimpali2019MMSteels, anstaett2017fabrication, edgar2015additive, liu2014interfacial}.
%
\begin{figure}[!hbtp]
  \centering
    \includegraphics[width=1.0\textwidth]{./img/mmlpbf-mach-stateArt.pdf}%
    \caption[MMLPBF system overview]{\gls{mmlpbf} system overview}%
    \label{fig:mmlpbf-mach-stateArt}
  \end{figure}

  The \gls{mmlpbf} system comprises:
  \begin{itemize}
  \item \underline{Powder bed}: reciprocating platform, moving along the z-axis,
    where the component is built.
  \item \underline{Powder delivery system}: mechanism by which powder is fed and
    delivered to the building platform.
    The precise and dependable placement of materials within the powder bed is
    contingent upon the proper functioning of this paramount system, rendering
    it indispensable to the process of multi-material fabrication. It must be
    carefully designed to limit cross contamination. Several technologies were
    applied to this pivotal system, namely, conventional
    recoating~\cite{bai2020dual, koopmann2019additive}, patterning
    drums~\cite{aerosintPatternDrum2019}, powder spreading with removal by
    suction~\cite{schneck2022capability, bareth2022Implem,
      bartolomeu2020additive, marques2022RocketEngine}, vibrating
    nozzle~\cite{zhang2019integrated, wei2019additive, demir2017multi}, hopper
    powder feeding~\cite{walker2022multi, auroraLabsRMP1}, alternating powder
    deposition~\cite{wang2022multi}, and electrophotographic powder
    deposition~\cite{stichel2020electrophotographic, foerster2020aspects,
      benning2018proof, stichel2018electrophotographic}.
  \item \underline{Powder recovery/recycling system}: another very common issue
    associated with the combination of multiple materials is
powder contamination, which is an inevitable consequence of material joining and
powder removal, unmelted areas of powder in the vicinity of the melt pool,
and oxidation. The reuse of this material is, therefore, even more challenging,
which is aggravated by its high cost. Powder recovery/recycling systems are applied to
mitigate this issue partially achieved by using the
variation of different physical properties between powders, such as particle size
distribution (sieving)~\cite{chivel2016new},  magnetism\cite{seidel2018status},
wettability~\cite{woidasky2017recyclingtechnik}, and particle inertia induced by
different powder densities~\cite{ullrich2013mechanische}.
    Efforts should be made to recuperate the surplus powder conveyed by the
    powder delivery system with the aim of minimizing powder contamination and
    facilitating its inertization, thereby achieving optimal levels of
    recyclability and contributinb for overall process sustainability and cost-effectiveness~\cite{obeidi2022LPBFGuidelines}.
  \item \underline{Temperature control system}: powder materials should be
    preheated to facilitate fusion or coalescence induced by the laser
    paths and to diminish temperature gradients which can weaken the part,
    leading to its collapse or fracture~\cite{thompson2015overview}.
    It is mandatory on the powder bed, and typically also included in the
    powder reservoirs.
  \item \underline{Atmosphere control system}: for powder materials, particularly metallic ones, it is essential to maintain a
controlled atmosphere to prevent oxidation. This is particularly crucial when
considering the elevated thermal gradients induced by laser operation and the
heating of the powder bed. To mitigate this effect, an enclosing and controlled
shielding system, which utilises an inert gas, typically argon or nitrogen, is
employed~\cite{thompson2015overview}.
  \item \underline{Laser}: the laser's systems are responsible for generation the
    beam (beam generation) and focusing it in the powder bed accordingly to the
    provided 2D coordinates (galvo scanning), inducing the melting
    of the powder materials. Optionally, additional lasers can be used. Firstly,
    to increase productivity. In this case, lasers are used with materials of
    the same type (e.g., metal alloys) and, thus, are usually just a duplicate
    with the same wavelenght. Hence, they can be combined through a
    single set of mirrors and focusing lens, and used
    simultaneously, moving along a X-Y gantry~\cite{nadimpali2019MMSteels}. Secondly, if different types of materials are
    used, e.g., metals and ceramics, lasers of different wavelengths are
    required, due to the variation in the sprectrum of radiation absorption and
    spot size.
  \item \underline{Control system}: controls the manufacturing process,
    commanded by an application. The User manages the process interacting with
    this application to provide the topological data and process parameters. The
    topological data (slices and paths information) is typically conveyed in
    the \texttt{*.SLI} format --- proprietary, used in the EOS and 3D Systems
    systems frameworks~\cite{zeng2013layer} --- or \gls{cli} or \texttt{*.ILT} formats --- open source, used, e.g.,
    in the Aconity or Renishaw frameworks~\cite{aconityMachinesSite,
      renishawAM500}. Aurora Labs equipments, on the other hand, require the
    machine instructions to be provided directly (G-Code)~\cite{nadimpali2019MMSteels}.
    The application will then generate the appropriate machine
    instructions, if required, and dispatch them. 
  \item \underline{Process Monitoring}: the \gls{lpbf} process benefits greatly
    of a closed-loop and adaptative control. This becomes even more true for
    \gls{mmlpbf} due to its increased complexity.
    The melt pool is monitored
    using \gls{ir} cameras, pyrometers, or laser speckles. The first two
    sensors are used to analyse the temperature evolution of the component, and
    allow to assess the stability of the process, but not on the material
    condition. On the other hand, \gls{lsp} analyses the change in the interference
patterns of the laser (speckles) to identify inhomonogeneities in 3D printing
and even ``invisible'' defects~\cite{chen2019laser}. The high-speed camera is
used to capture the melting process only for qualitative assessment of the build
process and powder distribution~\cite{obeidi2022LPBFGuidelines}.
Obviously, this requires a significant investment and resources, but could
greatly benefit the manufacturing quality.
  \end{itemize}

Next, the main \gls{mmlpbf} topics are analysed and discussed, namely, the
powder delivery system, the manufacturing chain and equipments, and the
manufacturing methodologies.

\subsection{Powder delivery systems}%
\label{sec:powd-deliv-syst}
One of most crucial parts of \gls{mmlpbf} fabrication is the correct and
adequate placement of the different materials in the building area, while
minimising the cross contamination between them. Thus, there are significant
efforts in the research and commercial domains to pursue more effective powder
delivery systems~\cite{mussatto2022research, aerosintPatternDrum2019}.

The simplest powder deposition system relies on the adaptation of \gls{3dmmlpbf}
process to use the conventional \gls{lpbf} system~\cite{bai2020dual, koopmann2019additive}: (1) the build process is halted
at the layer where the material transition is desired, (2) the powder is
changed, (3) the building is resumed. This imposes several limitations. First,
only a single material can be used at a time. Second, the material transitions
occur in a layer-wise manner\cite{mussatto2022research}, and the abrupt changes in the physical properties of the materials'
interfaces leads to defects and, consequently, stress concentrations, which can
induce fracture of the components~\cite{mahamood2015laser}.
Furthermore, powder
compactation and bed levelling require several passages of the recoating unit,
and the laser procedure should be repeated on the transition layer 2--3 times to
ensure better fusion between materials~\cite{bai2020dual}.

The next improvement step is ensuring that cross contamination is minimised
through the use of a proper removal method between materials' depositions. Thus,
the same limitations of the conventional method apply.
The simplest method to remove the powder is resorting to a vaccum
suction system~\cite{schneck2022capability}, which ideally should be dedicated
to each material~\cite{bartolomeu2020additive} to avoid cross contamination. Bareth et al.~\cite{bareth2022Implem} upgraded
a commercial unit of a AconityONE printer with a multi-material mobile plug-in
module to feed a second material to the building chamber (Fig.~\ref{fig:aconityONE-upgrade}). However, the same
suction system is used for both materials. Marques
et al.~\cite{marques2022RocketEngine}, on the other hand, utilised a dedicated
powder removal system.

% subfigures
\begin{figure}[htbp!]
  \centering
  %
  \begin{subfigure}[t]{0.44\textwidth}
  \centering
  \includegraphics[width=1.0\textwidth]{./img/aconityONE-upgrade-schem.png}
  \caption{Schematic}%
  \label{fig:aconityONE-upgrade-schem}
  \end{subfigure}
%
  \begin{subfigure}[t]{0.54\textwidth}
  \centering
  \includegraphics[width=1.0\textwidth]{./img/aconityONE-upgrade-mach.png}
  \caption{Implementation}%
  \label{fig:aconityONE-upgrade-mach}
\end{subfigure}
%
  % 
  \caption[Conventional LPBF printer upgraded with a multi-material
  mechanism]{Conventional LPBF printer upgraded with a multi-material mechanism --- 1: AconityONE; 2: process chamber; 3: scan head;
    4: powder deposit (material B); 5: powder conveyor; 6: recoater and suction
    unit; 7: powder slide; 8: cyclone separator; 9: vaccum pump; 10: electronic
    control unit
    (withdrawn from~\cite{bareth2022Implem})\footnotemark}%
  \label{fig:aconityONE-upgrade}
\end{figure}

\fnlicCCNCa{foot:aconityUpgrade}%

Another variation of the spreading principle is used by
alternating powder deposition systems. These systems utilise two opposing
recoaters, whether aligned~\cite{wang2020formation} or perpendicular~\cite{goll2019additive} to each other, to spread two materials and obtain
a inter-layer material variation.

Powder hoppers provide separate housing for delivering powders to the
building chamber, sectioning the build area. They can be employed for mono-material
processing or for blending multiple powders simultaneously to obtain a specific composition. They are typically placed externally and above the processing
chamber, and operated with piezoelectric transducers and solenoid valves to
regulate the unloading of powder~\cite{wang2022multi, chen2020influence}, which is then spread onto the powder bed by a
recoating system, such as a coating blade~\cite{wang2022multi}. This system allows, not only
layer-wise material transitions, but also discrete
and gradient material transition along the build
direction~\cite{demir2017multi}. Very recently, Walker et
al.~\cite{walker2022multi} installed a powder hopper system into a \gls{lpbf}
machine for graded alloy processing, with the capability to deposit specific powder of varying material composition in any 3-dimensional location (see Fig.~\ref{fig:mmlpbf-mach}).
The composition mixtures
are created prior to processing and separated into individual hoppers or the
powder supply, for multi-material processing.

%
\begin{figure}[!hbtp]
  \centering
  %
  \begin{subfigure}[t]{1.0\textwidth}
  \centering
  \includegraphics[width=0.6\textwidth]{./img/mmlpbf-mach.jpg}%
  \caption{Structure and process}%
  \label{fig:mmlpbf-multiHopper}
  \end{subfigure}
%
  \begin{subfigure}[t]{1.0\textwidth}
  \centering
  \includegraphics[width=0.6\textwidth]{./img/multiHopper-prototypes.jpg}
  \caption{Components produced: (a) concentric circles of Ti to Ta on the
    vertical plane; (b) IN 718 to GRCop-42 in the horizontal plane and (c) Ti to Ta in both the vertical and horizontal planes}%
  \label{fig:mmlpbf-multiHopper-prototypes}
\end{subfigure}

    \caption[MMLPBF machine with multiple hoppers]{\gls{mmlpbf} machine with
      multiple hoppers (withdrawn from~\cite{walker2022multi})\footnotemark}%
      %\fnref{foot:cc-lic}}%
      %\textsuperscript{\ref{foot:cc-lic}}%
    \label{fig:mmlpbf-mach}
\end{figure}
%
\fnlicReq{Elsevier}{5296470140711}

The ultrasonic vibration-assisted powder delivery system with a vortex suction
nozzle is currently the most researched approach for multiple powder
delivery~\cite{mussatto2022research}. The powders are selectively deposited by
means of nozzles with small orifices~\cite{stichel2014powder} regulated by controlling the electrical
pulses to the piezoelectric transducers~\cite{wei2019easy, zhang2019integrated, wei2019additive}. This allows for more precise control of
the powder flow rates. Wei et al.~\cite{wei2019additive} developed a vibrating
nozzle system capable of depositing up to six discrete powder materials within
one layer. Demir and
Previtali~\cite{demir2017multi} used a mixed approach and developed a double hopper powder delivery system
based on piezoelectric transducers which enabled the manufacturing of a
Fe/Al-12Si specimen, with an intermixed region between the two
materials (Fig.~\ref{fig:ultrasonic-dispensing}).
Kumar et al.~\cite{kumar2004direct} used glass pipettes as
`hopper-nozzles' to spread powder, by means of gas pressure or vibration
feed, allowing a precise powder delivery, without the need to vacuuming the
excess~\cite{kumar2004direct}.

\begin{figure}[!hbtp]
  \centering
  %
  \includegraphics[width=0.68\textwidth]{./img/ultrasonic-dispensing.jpg}
  \caption[Vibrating nozzle dispenser system]{Vibrating nozzle dispenser system:
    a) Concept of using multi-material transition zone for assembling Al-alloys
    with steel. b) In-house built prototype SLM system with multi-material
    processing capability. c) Design of the powder feeder system. d) Working
    principle of the powder feeder for mixing powders. e) Calibration curves of
    the delivered powder mass (m) of pure Fe and f) Al-12Si as a function of
    applied voltage (A) and vibration time ($t_v$). Error bars depict 95\%
    confidence interval for the mean. (withdrawn from~\cite{demir2017multi})\footnotemark}%
    \label{fig:ultrasonic-dispensing}
\end{figure}
%
\fnlicReq{Elsevier}{5536180654674}

Next, we shall discuss the most disruptive technologies.
The electrophotographic technology is used to build patterned layers
composed of multiple powders using metal powder transfer. The working principle
relies on the electrical charge of the powders, simillarly to common paper toner
printers xerographic process~\cite{van2005metal}.
First, a photoreceptor is uniformly charged using an electrical corona to a
specified charge density~\cite{bakkelund1997fabrication}. Following this, laser
exposure is used to selectively discharge the photoreceptor according to layer
data, resulting in an electrostatic image containing powder data of the
component for a given layer~\cite{psarommatis2022systematic,
  stichel2020electrophotographic}. The photoreceptor is then brought close to
the powder supplier, causing powder particles to attach to the appropriate
charged areas of the photoreceptor. The developed powder image is then
transferred onto the charged build substrate via electrostatic attraction
force~\cite{mehrpouya2022multimaterial}. In the final step, the photoreceptor is
cleaned with a blade to remove residual particles and discharged with a second
laser exposure before the start of the next powder deposition
cycle~\cite{benning2018proof}.

In 2019, Aerosint used the electrophotographic principle~\cite{wang2022recent} to develop a novel selective powder deposition technology that
enables the deposition of dry powder particles, such as polymer, metal, or
ceramic, to form a single layer containing at least two materials with low
waste~\cite{aerosintPatternDrum2019} (Fig.~\ref{fig:aerosint-recoater}). The technology employs rotating patterning drums, where each drum
deposits one material. This process selectively deposits a volume pixel (voxel)
of powder in a layer-by-layer fashion, with speeds up to 200 mm/s, and is less
sensitive to powder characteristics than conventional
systems~\cite{eckes2018multi}. Currently, Aerosint selective powder deposition
system is used only in Aconity machines --- AconityMIDI+ and AconityMICRO
\gls{lpbf} printers --- to enable multi-material fabrication~\cite{aerosintSite}.

%
\begin{figure}[!hbtp]
  \centering
  %
  \begin{subfigure}[t]{0.48\textwidth}
  \centering
  \includegraphics[width=1.0\textwidth]{./img/aerosint-process.png}%
  \caption{}%
  \label{fig:aerosint-proc}
  \end{subfigure}
  %
  \begin{subfigure}[t]{0.48\textwidth}
  \centering
  \includegraphics[width=1.0\textwidth]{./img/aerosint-mech.png}%
  \caption{}%
  \label{fig:aerosint-mech}
  \end{subfigure}
%
  %
  \begin{subfigure}[t]{0.48\textwidth}
  \centering
  \includegraphics[width=1.0\textwidth]{./img/aerosint-heatExchanger-bed.png}%
  \caption{}%
  \label{fig:aerosint-heatExchanger-bed}
\end{subfigure}
%
  \begin{subfigure}[t]{0.48\textwidth}
  \centering
  \includegraphics[width=0.6\textwidth]{./img/aerosint-heatExchanger-part.png}%
  \caption{}%
  \label{fig:aerosint-heatExchanger-part}
  \end{subfigure}
%
  \caption[Aerosint selective powder deposition system for MMLPBF]{
    Aerosint selective powder deposition system for MMLPBF: (a) Process~\cite{aerosintPatternDrum2019}; (b) Dual
    powder recoater~\cite{aerosintSite}; (c) CuCrZr--316L Heat Exchangers part printing on Aconity
    MIDI+ LPBF printer powered with Aerosint's multi-material recoater~\cite{aerosintHeatExchanger}; (d)
    printed part~\cite{aerosintSite}. Reproduced with permission, copyright Aerosint SA}%
      %\fnref{foot:cc-lic}}%
      %\textsuperscript{\ref{foot:cc-lic}}%
    \label{fig:aerosint-recoater}
\end{figure}
%

Mussatto~\cite{mussatto2022research} classified the powder deposition systems technologies accordingly to
the type of powder material used, the cross contamination level, material
transition between layers, the gradative powder deposition within one layer, the
discrete powder deposition within one layer, and the time to form a powder bed
(Table~\ref{tab:powder-deposit-systems}). The results show that the level of
productivity is low for all technologies except conventional and alternating
spreading. Furthermore, powder spreading with suction is overall the most well
performant. It can be used with all types of powder materials, and has the
lowest cross contamination, although with the significant shortcoming of not
allowing gradient powder deposition.

\begin{table}[!hbt]
\centering
\caption[Overview of the advantages and disadvantages of the various
multi-powder deposition systems for L-PBF]{Overview of the advantages and
  disadvantages of the various multi-powder deposition systems for L-PBF
  (adapted from~\cite{mussatto2022research})\footnotemark}%
\label{tab:powder-deposit-systems}
\resizebox{1.0\textwidth}{!}{%
\begin{tabular}{llllllllllll}
\hline
System & SP & PB & MP & PP & MEP & CP & PPPCCL & OPPLLT & GPDWOL & DPDWOL & LPTFPB \\
\hline
Conventional spreading & ✓ & ✓ &  & ✓ & ✓ & ✓ & L & ✓ &  &  & T \\
%\hline
Patterning Drums & ✓ &  & ✓ &  & ✓ & ✓ & H & ✓ &  & ✓ & S \\
%\hline
Spreading + suction & ✓ & ✓ & ✓ & ✓ & ✓ & ✓ & L & ✓ &  & ✓ & VS \\
%\hline
Vibrating Nozzle & ✓ & ✓ & ✓ &  & ✓ & ✓ & M & ✓ & ✓ & ✓ & VS \\
%\hline
Hopper feeding & ✓ & ✓ & ✓ &  & ✓ & ✓ & H & ✓ & ✓ & ✓ & S \\
%\hline
Alternating & ✓ & ✓ & ✓ & ✓ & ✓ & ✓ & M & ✓ &  &  & T \\
%\hline
Electrophotographic & ✓ &  & ✓ &  & ✓ & ✓ & H & ✓ &  & ✓ & VS \\ 
  \hline
  \multicolumn{12}{l}{SP = Single Powder; PB = Powder blends; MP = Multi powder} \\
  \multicolumn{12}{l}{PP = polymeric powders; MEP = Metallic Powders; CP = Ceramic Powders} \\
  \multicolumn{12}{l}{
  PPPCCL = Post Printing Powder Cross Contamination Level;
  L = Low; M = Medium; H = High;} \\
  \multicolumn{12}{l}{
  OPPLLT = One Powder Per Layer (Material Transition Between Layers)} \\
  \multicolumn{12}{l}{
  GPDWOL = Gradient Powder Deposition Within One Layer} \\
  \multicolumn{12}{l}{
  DPDWOL = Discrete Powder Deposition Within One Layer} \\
  \multicolumn{12}{l}{
  LPTFPB = Level of Productivity (Time to Form a Powder)}
\end{tabular}
}
\end{table}

\fnlicCCNCa{foot:mmlpbfPowderDepSys}%

The author further concluded that despite
many significant advances in the multi-material processing over the last few
years, the issue of powder cross-contamination remains a serious problem that
needs to be addressed. Furthermore, there is an urgent need for an in-depth
analysis of the \gls{mmlpbf} process so that the knowledge generated can be used
to optimise and improve the process to specific needs. 

\subsection{Manufacturing chain}%
\label{sec:manuf-chain-equipm}
Fig.~\ref{fig:mmlpbf-stateArt-Manuf-Chain} illustrates the \gls{mmlpbf}
manufacturing chain based on the scientific literature~\cite{seidel2022multi} and the commercial
equipment panorama~\cite{aconityMachinesSite, slm500, eosM300, renishawAM500,
  trumpfTruprint500, auroraLabsRMP1}. 
The manufacturing chain is divided into three stages: pre-process,
in-process, and post-process.
%
\begin{figure}[!hbtp]
  \centering
    \includegraphics[width=1.0\textwidth]{./img/mmlpbf-stateArt-Manuf-Chain.pdf}%
    \caption[MMLPBF manufacturing chain overview]{\gls{mmlpbf} manufacturing chain overview}%
    \label{fig:mmlpbf-stateArt-Manuf-Chain}
  \end{figure}


\subsubsection{Pre-Process}%
\label{sec:pre-process}
In the pre-process stage, the component is designed as an
assembly of multiple individual \gls{cad} models corresponding to each
material, and the transition between them, according to
the requirements specifications~\cite{altenhofen2018continuous, yao2018multi}.
This is a result of the limitation of most of the mainstream commercial
software to express material information, which hinders multi-material
design~\cite{wang2022recent}. On the other side of the sprectrum one has specialised software
for the generation of graded material transitions in volumetric models, like
GraMMaCAD (Graded Multi-Material CAD), which supports the assignment of
locally varying material properties within a 3D model~\cite{grammacad}.

The model is then simulated both at a
material scale and integrated (assembled) scale. The geometry of each model
is exported independently, usually in the \gls{stl} format, OBJ (Object file format), \gls{amf}, and PLY (polygon file
format)~\cite{loh2018overview}. A sanity check is performed on the geometry
file (and repaired) and the topology is optimised. The \gls{dfm} or
\gls{dfam} imposes constraints on the topology taking into consideration the
manufacturing process.

Then, in the \gls{cam} phase, the model is virtually
placed into the building platform, and the orientation is optimised, and, if
required, supports are generated for the component to ensure adequate
printing. Each material model is then sliced, and the paths are generated,
yielding the layer information typically in the open-source formats
(\gls{cli}, ILT --- Aconity~\cite{aconityMachinesSite}, Reinshaw~\cite{renishawAM500}), or the proprietary SLI (EOS, 3D
Systems)~\cite{zeng2013layer}. Each material model is then merged in the
manufacturing model. The topological data is mapped to the process parameters,
accordingly to each material, yielding the manufacturing instructions for
the equipment, typically in a G-Code simillar
format~\cite{nadimpali2019MMSteels}.

Fig.~\ref{fig:mmlpbf-preprocess-SW}
shows an extensive list of some of the many software applications for the
\gls{mmlpbf} pre-processing compiled by Moreno et.al~\cite{moreno2021design}. The vast majority of these applications are
closed-source and released under commercial licences.

% 
\begin{figure}[!hbtp]
  \centering
  \includegraphics[width=1.0\textwidth]{./img/mmlpbf-preprocess-SW.png}%
  \caption[MMLPBF Pre-Process software]{\gls{mmlpbf} Pre-process software
    (withdrawn from~\cite{moreno2021design}\footnotemark)}%
  \label{fig:mmlpbf-preprocess-SW}
\end{figure}
\fnlicCCNCa{foot:preproc-SW}%

The pre-processing procedure is time and effort consuming and it involves the
generation of large amounts of data~\cite{binder2018potentials,
  schneck2021review, wang2022recent}. Nonetheless, the available software
suite does not provide the automation required for the process.
Fig.~\ref{fig:mmlpbf-data-proc} shows one example of a custom manual data
pre-processing procedure devised by Wei et al.~\cite{wei20183d} to surpass the
lack of software tools for the \gls{mmlpbf} process. The parts were modelled and assembled, and
individual material geometry converted into the \gls{stl} format. Then, each
material is sliced and hatched using a proprietary \gls{cnc} CAM tool to
generate the toolpaths and G-Code for the powder dispensing system.
Furthermore, the process is complex and does not scale well to industrial applications~\cite{wang2022recent}.

\begin{figure}[!hbtp]
  \centering
 \includegraphics[width=0.5\textwidth]{./img/mmlpbf-data-proc.jpg} 
 \caption[Flowchart for data flow for \gls{mmlpbf}]{Flowchart for data flow for \gls{mmlpbf} (withdrawn from~\cite{wei20183d})\footnotemark}%
\label{fig:mmlpbf-data-proc}
\end{figure}
%
\fnlicReq{Elsevier}{5296410224943}

Thus, several methods have been proposed to address the shortcomings of the current
pre-processing procedure for the \gls{mmlpbf} process.
Firstly, there is the need for a data interface file that can convey both
geometrical and material information~\cite{wang2022recent}. \gls{stl} files are
a tesselation of facets and the corresponding normals without any material
data. Furthermore, it is unable to accurately represent holes,
porosity and discontinuities. As a result, back in 2009, Hiller and Lipson have proposed STL 2.0 to combine geometric and
material information~\cite{hiller2009stl}, but it never gained
traction~\cite{wang2022recent}. The OBJ file can store colour information, but
not material one~\cite{wang2022recent}. On the other hand, the \gls{amf} file
conveys geometric and material information, but occupies a large storage space
and is not yet mature~\cite{chenyan2019multi, shi2017status}. The PLY file can
express texture and colour, but not when a part contains different material
properties~\cite{xiaowei2017new}.

Hence, other file formats were developed for \gls{mmlpbf} fabrication, namely \gls{fav},
\gls{svx}, and \gls{3mf}, which can convey information about the material
gradient and micro-scale physical properties~\cite{loh2018overview}. The first
two are based in volume discretization using volume pixels (voxels): \gls{fav}
incorporates colours, materials, and connection strength information (Fig.~\ref{fig:fav}); \gls{svx}
comprises material allocation, density, and colour information~\cite{wang2022recent}. The \gls{3mf}
is a open source standard file format based on the \gls{xml} specification to
describe the intrinsic and extrinsic information of a model. However, it does
not support higher-order representations such as \gls{nurbs} (used by
GraMMaCAD~\cite{grammacad}) and \gls{step}~\cite{wang2022recent}.

Secondly, comes the slicing and path generation procedure. As aforementioned
(see Section~\ref{sec:monom-laser-powd}) the energy density is the primary
driver of melting and can be regulated through a combination of scanning
strategy (type, hatching distance, angle) and process parameters (laser power, laser
scanning speed). Several scanning strategies were tried out, namely simple,
alternate, stripes, island, sinusoidal, and
chessboard~\cite{obeidi2022LPBFGuidelines, wang2022recent}. Mussatto et
al.~\cite{mussatto2022laser} used Solidworks to 3D model the specimens which
were then sliced by the commercial slicing tool Netfabb Autodesk. In the same
work, the authors needed to generated the 3D models of the specimens using a
sine function and Excel Macro and the JavaScript programming language to
generate the sinusoidal hatching and export it to the \gls{cli} format.
Thus, the commercial \gls{cam} tools available are not able to produce custom
paths in a straightforward and flexible manner.

Lastly, we have the manufacturing file generation. Most commercial \gls{lpbf}
equipments are only available for mono-material fabrication~\cite{mussatto2022research} and, as
aforementioned, they require upgrading for multi-material fabrication. For
example, Nadimpali et al.~\cite{nadimpali2019MMSteels} upgraded an Aurora Labs
S-Titanium PRO to use two 150 W CO\textsubscript{2} lasers
simultaneously, but they needed to modify directly the G-Code instructions to
operate the machine. This defeats the purpose of pre-processing, as it requires
specialised staff to do the on-line programming of the machine with explicit
manufacturing instructions that would otherwise been generated by the
pre-processing procedure.

%
\begin{figure}[!hbtp]
  \centering
  %
  \begin{subfigure}[t]{0.48\textwidth}
  \centering
  \includegraphics[width=1.0\textwidth]{./img/fav1.jpg}%
  \caption{}%
  \label{fig:fav1}
  \end{subfigure}
  %
  \begin{subfigure}[t]{0.48\textwidth}
  \centering
  \includegraphics[width=1.0\textwidth]{./img/fav2.jpg}%
  \caption{}%
  \label{fig:fav2}
  \end{subfigure}
%
  %
  \begin{subfigure}[t]{0.7\textwidth}
  \centering
  \includegraphics[width=1.0\textwidth]{./img/fav3.jpg}%
  \caption{}%
  \label{fig:fav3}
\end{subfigure}
%
%
  \caption[Fabricated Voxel file format]{
    \gls{fav} file format~\cite{fujifilmFAV}: (a) Conceptual diagram showing voxels arranged
    three-dimensionally; (b) the FAV format can retain information on internal
    structure, colour, and material; (c) Simulations can be performed on voxel data as-is, and designs can be modified to reflect simulation results. Reproduced with permission, copyright
    FUJIFILM Business Innovation Japan Corp.}%
      %\fnref{foot:cc-lic}}%
      %\textsuperscript{\ref{foot:cc-lic}}%
    \label{fig:fav}
\end{figure}
%

\subsubsection{In-Process}%
\label{sec:process}
In the \underline{In-Process} stage, the manufacturing instructions are used to
control the \gls{mmlpbf} building cycle. Process monitoring and closed-loop
control are optional, but strongly recommended, to ensure optimal
manufacturing quality.

The most significant challenge in the \gls{mmlpbf}
process is to guarantee the accurate and effective delivery of the different
materials to the building chamber. However, only few commercial systems  can be used for multi-material fabrication
out of the box~\cite{aconityMachinesSite, aerosintPatternDrum2019, slm280}. Thus, in this section, we explore further the
panorama of available solutions, both in the commercial and research fields.

Table~\ref{tab:lpbf-commercial-equip} shows a list of most relevant \gls{lpbf}
commercial equipments, based on a survey over product specifications~\cite{aconityMINI, aconityMIDI, trumpf300, slm280, slm280Temp, renishawAM500, eosM300}. The
prices were obtained from quotations provided directly by the manufacturers, and
do not include taxes.
From this list, all equipments use a feedstock of metallic powder, but only the AconityMIDI+~\cite{aconityMIDI}
and the SLM 280 2.0~\cite{slm280} can be used for multi-material fabrication. The \gls{sw} stack for
these two equipments includes commercial proprietary software for pre-process or
in-process management. The entry-level product for Aconity --- AconityMINI ---
starts at 228 K €, thus it is expected that AconityMINI+ costs even more. On the
other hand, the
SLM 280 2.0 starts at 500 K €. This clearly demonstrates that the initial
investment in a \gls{mmlpbf} equipment is significantly high. Furthermore, in
the current setup, these equipments can only be used with metallic powders,
which limits the scope of applications.

\begin{table}[!hbtp]
\centering
\caption[LPBF commercial equipments]{\gls{lpbf} commercial
  equipments~\cite{aconityMINI, aconityMIDI, trumpf300, slm280, slm280Temp, renishawAM500, eosM300}}%
\label{tab:lpbf-commercial-equip}
\resizebox{1.0\textwidth}{!}{%
\begin{tabular}{lllllllllll}
\hline
Equipment & Build Volume & Materials & LT (µm) & MM & \# Lasers & PH (\degree C) & PM & SW Stack & NW (kg) & Cost (€) \\
\hline
AconityMINI & \diameter 250 x 250 mm & metals & 10 (min) & No & 1 (Fiber) & 800 (max) & Optional & Autodesk Fabb Premium & 850 & 228 K \\
 &   &  &  &  &  &  &  & AconityStudio &  &  \\
  \hline
AconityMIDI+ & \diameter 150 x 150 mm & metals & 10 (min) & Yes & 4 (Fiber) & 1000 (max) & Optional & Autodesk Fabb Premium & 1600 & N.D. \\
 &   &  &  &  &  &  &  & AconityStudio &  &  \\
  \hline
TruPrint3000 & \diameter 300 x 400 mm & metals & 20 & No & 4 (Fiber) & 200 (max) & Optional & Proprietary Tool Suite & 4300 & 950 K \\
  \hline
SLM 280 2.0 & 280 x 280 x 280 mm & metals & 20 (min) & Yes & 2 (Fiber) & 550 & Included & Proprietary Tool Suite & 1300 & 500 K \\
  \hline
RenAM 500Q/S & 250 x 250 x 350 mm & metals & 30 (min) & No & 4 (Fiber) & N.D. & Included & QuantAM & 1960 & N.D. \\
 & & & & & &  &  & RenAMP &  &  \\
 & & & & & &  &  & InfiniAM &  &  \\
  \hline
EOS M300-4 & 300 x 300 x 400 mm & metals & N.D. & No & 4 (Fiber) & N.D. & Included & EOSPRINT2 & 5500 & 1300 K \\
 & & & & & &  &  & EOS ParameterEditor &  &  \\
 & & & & & &  &  & EOSTATE Monitoring Suite &  &  \\
 & & & & & &  &  & EOSCONNECT Core &  &  \\
 & & & & & &  &  & EOSCONNECT Core &  &  \\
 & & & & & &  &  & EOSCONNECT MachinePark &  &  \\
 & & & & & &  &  & Materialise Magics Metal &  &  \\
 & & & & & &  &  & Packages and modules &  &  \\
  \hline
 \multicolumn{11}{l}{LT: Layer Thickness; MM: Multi-material; PH: Preheating; PM: Process Monitoring; SW: Software; NW: Net weight}
\end{tabular}
}
\end{table}

As a result of the lack of commercial solutions for \gls{mmlpbf} and/or their
cost, researchers tried to upgrade some of these equipments.
Liu et al.~\cite{liu2014interfacial} and Sing et al.~\cite{sing2015interfacial} upgraded a powder
feeder of a conventional \gls{lpbf} system --- SLS 250 HL from SLM Solutions ---
equipped with a 400 W fiber laser to store and deliver two metallic powder materials separately.
C18400 copper have been successfully
deposited on top of 316L stainless steel and AlSi10Mg, although the transition zone
between the two materials could not be controlled, which is the most critical in a multi-material part.
Anstaett et al.~\cite{anstaett2017fabrication} also upgraded a SLS 250 HL
\gls{lpbf} system to deposit two metallic powders, combining a Cu-alloy and a
tool steel to produce a multi-material component~\cite{anstaett2017fabrication}.

More recently, Nadimpali et al.~\cite{nadimpali2019MMSteels} upgraded an Aurora
Lavs S-Titanium Pro \gls{lpbf} equipment, due to its multiple hopper dispenser, to combine 150 W CO\textsubscript{2}
lasers simultaneously, with an average maximum power output of 255 W. The 
two beams pass separately through a set of optics before entering
into the focus lens resulting in a single spot with approximately $150~\mu m$
in diameter in the processing plane. The powders are mixed thanks to the powder
dispensation and sweeping mechanisms. However, the authors required to
explicitly modify the G-Code to operate the machine.
In 2022, Schneck et al.~\cite{schneck2022capability} also upgraded a SLM 250 HL
\gls{lpbf} equipment with a powder dispensing system consisting of a suction
module and blade coater to manufacture a prototype injection nozzle for an
internal combustion engine using metallic powders. The pre-processing was
manual, with each material of the 3D component modelled independently and
exported as \gls{stl} file, which was then sliced and converted to the
proprietary SLM file format using AutoFab \gls{cam} \gls{sw}.

Lastly, Bareth et al.~\cite{bareth2022Implem} upgraded an Aconity MINI
\gls{lpbf} equipment with a multi-material mobile plug-in
module to feed a second material to the building chamber (Fig.~\ref{fig:aconityONE-upgrade}). However, the same
suction system is used for both materials, which increases the cross
contamination and diminishes the powder recyclability rate. The multi-material
module was connected to the machine's \gls{plc}, which made it possible
to be controlled directly from the software by modifying the custom G-Code.

\subsubsection{Post-Process}%
\label{sec:post-process}
The \gls{lpbf} and \gls{mmlpbf} post-processing are identical and involve
several fundamental steps and challenges such as powder removal and recycling, part
separation, in-situ process monitoring, post-processing treatments and quality
control. However, there are currently no specific guidelines available for the
mechanical and thermal post-processing of multi-material
parts~\cite{schneck2021review}.

Separation of mixed powders represents an additional challenge, which requires
the removal of various powder materials from each other as well as the
elimination of process-related impurities.

The current state of the art in
material separation proves to be challenging and has not been implemented in
practice~\cite{binder2018potentials, wei2021recent}.
Seidel~\cite{seidel2022multi} evaluated the different working principles for
powder reciclability, such as, particle size, ferromagnetism, density, mass,
electrical conductivity and surface wettability. The author concluded that
ferromagnetic separation and sieving are the most promising technologies.

When validating the part, it is important to investigate mechanical and thermal
parameters and microstructural analyses in the joining zone. Several
investigations have been conducted to characterise the joining zone of 2D
specimens using optical methods such as optical and scanning electron microscope
(SEM) images to analyse the microstructure, strength, and
defects~\cite{anstaett2017fabrication, wei2019additive, bartolomeu2020additive,
  marques2022RocketEngine, mao2022effects}.

The last fundamental problem is related to process knowledge. It is critical to
understand the process thoroughly to obtain good quality
parts~\cite{fayazfar2018critical,tey2020additive,wei2021understanding}.

\subsubsection{Prospects for improvement}%
\label{sec:prosp-impr-stateArt}
The main takeway is that the \gls{mmlpbf} fabrication is a multi-objective
problem, requiring specially designed equipments and toolchains, and
experimental design.

Back in 2012, Gu et al.~\cite{gu2012laser} highlighted the need to create a
process knowledge database to handle the inherent complexity of the process and
support its evolution. Yet, ten years later Mussatto~\cite{mussatto2022research}
reports the gap was not still not addressed and there is an urgent need for an
in-depth analysis of the \gls{mmlpbf} process to ensure sustainable process improvement. 

It is essential to develop a multi-material simulation
software tool or an \gls{ai} prediction method of the optimal process parameters
beforehand to narrow the actual experimental range~\cite{wei2020overview}.
Last year, Exponential Technologies, in a collaboration with Aerosint, reported
the two-fold improvement in component density within 2 print jobs and 46 samples
printed without any prior statistical knowledge. This yielded a reliable
parameter set for printing Stainless Steel 316L and Inconel 625 combinations,
saving time and money~\cite{xtSAAM2022AI}.

Furthermore, machine learning can be used as a means for automated evaluation
and classification of sensor data to assist in process monitoring and
closed-loop control of the \gls{mmlpbf} equipment. However, the research is
still in a developing stage~\cite{mccann2021situ, yadav2020situ}.

Only a handful of commercial equipments support multi-material fabrication out
of the box, and all of them, use metallic powders, limiting the scope of
applications and, inherently, functional design. This may be a limitation of
some powder deposition systems, but it most linked to the fact that only fiber
lasers are used for the printing. Moreover, the significant initial investment
may be a obstacle, especially in the research field, further hindering the
process evolution.

The manufacturing chain is complex and the toolchain varies greatly among
manufacturers, specially within the pre-process stage.
Specific guidelines for \gls{mmlpbf} were not identified and most pre-processing
procedures use a manual method based on inadequate data format (\gls{stl})~\cite{schneck2021review}.
Further research is
required to assemble a coherent and straightforward pre-processing tool to
enable multi-material functional design, simulation, and optimisation. The most
promising approaches are based on volume discretization (voxels), like the
\gls{fav} format and the GraMMaCAD software. In the post-processing, further
research is required for the surface and heat treatments involving
multi-material components~\cite{schneck2021review}.

\section{Applications}%
\label{sec:applications}

After introducing the theoretical foundations for multi-material fabrication,
it's important to highlight the possible applications, as they usually drive
any given methodology, and they certainly should drive the functional design,
providing a more clear view of what is being accomplished and how.

The main application for functional design through the use of multiple materials
in the scope of this PhD project is in the biomedical field --- the hip implant.
This aims to address several issues encountered in the current hip implants that
lead to a short lifespan of these prosthetic implants, with forced
retreament surgeries spanning a short period of time (ten to fifteen years after
the implantation surgery)~\cite{soliman2022review}.

%And, according to the annual report \emph{Transparency Market
%Research}~\cite{market2015global}, there is
%an increasing demand trend: the increase in life expectancy of the population,
%the obesity and the increase of hip fractures will demand 6\% more hip
%implants in 2019 when compared to 2013.

The goal of the hip implant to mimic the natural behaviour of the bone to serve
as its suitable replacement; as such, and not surprisingly, the material
composition, properties and structure needs to be varied.
Fig.~\ref{fig:hip-implant} illustrates the hip implant constitution. The artificial acetabular cup and
femoral head replace the damaged natural articulation~\cite{derar2015recent}. Thus, here
allocated materials must have low friction and withstand wear and mechanical
loads. The femoral head is anchored in the femur by the stem.
The acetabular cup
is anchored in the pelvis and is composed of a shell in which a liner is
inserted that provided the load bearing articulating surface.
On other hand, shell and stem have to provide good bone
integration~\cite{eltit2019mechanisms}. Furthermore, the leaching of toxic
metallic ions, the tribocorrosion effect, and stress shielding, imposes
significant challenges on the design~\cite{rafiq2022review}.
%
This requires different materials and varying compositions to achieve the
desired individual component's function; however, the most important design goal is the
overall function of the hip implant, requiring especial attention to the
interfaces between components, but maintaining good properties for the
individual components also.

\begin{figure}[!hbtp]
  \centering
    \includegraphics[width=0.8\textwidth]{./img/hip-implant-new.jpg}
  \caption[Hip prosthesis and components detail]{Three main types of hip implants showing ball-and-socket joint regions: (A) Large head metal-on-metal (MoM) total hip implant. (B) MoM hip resurfacing. (C) Metal-on-polyethylene (MoP) total hip implant.
    (withdrawn from~\cite{eltit2019mechanisms}\footnotemark)}%
  \label{fig:hip-implant}
\end{figure}
%
% (Fig.~\ref{fig:hip-implant})
\fnlicCCNC{foot:hip}%

According to a recent review article~\cite{rafiq2022review}, crosslinked polyethylene, titanium alloys, CoCr alloys, stainless steel, zirconium-niobium alloy, tantalum, ceramics, and composite combinations are among the commonly used materials for prosthesis fabrication. In particular, bioceramics coatings such as DLC and TiN have demonstrated promising results in enhancing osseointegration. However, due to the significant disparity in mechanical properties among the current implant materials, the lifespan of hip implants may be considerably reduced.

The multi-material additive manufacturing, and more specifically \gls{mmlpbf},  can mitigate this problem by allowing
some of the assembled components to be merged in a single component, where the
characteristics of the interfaces are custom-tailored to obtain the desired local properties, obviously through careful design.

\section{Summary}%
\label{sec:summary-state}
This chapter provides an overview of the current state of the art in the realm
of multi-material fabrication of metals and composites and its contribution to
functional design. \gls{am} processes were examined and categorized, with a
particular focus on their usefulness in rapid prototyping and the manufacture of
high-value parts by facilitating material addition only where needed, thereby
achieving functional design objectives.
Hybrid
processes were briefly mentioned as a better, but more complex, match for
functional design. Therefore, the focus in this work is limited to \gls{am} processes.

Laser sources are an attractive option for \gls{am} processes due to their small spot diameter, high energy density, and precise energy control, making them well-suited for fabricating metallic parts with high melting points. \gls{lpbf} and \gls{dld} represent the two dominant approaches in this field, with \gls{lpbf} being best-suited for small and medium-sized multi-metal parts with high resolution and \gls{dld} being preferable for larger parts. Given the additional complexity of \gls{dld}, \gls{lpbf} processes currently represent the best commercial and industrial option for fabricating high-precision multi-material parts.

The \gls{lpbf} process was examined in detail, along with its requirements and working principle. Multi-material additive manufacturing (MMAM) was introduced as a highly desirable technology, facilitating the precise placement of material based on its functionality, thereby yielding custom-tailored parts for specific applications with improved mechanical properties and behavior in service beyond what is achievable by conventional methods. The potential applications of MMAM are wide-ranging and critical, particularly in the fields of biomedical engineering, soft robotics, and electronics.

Subsequently, the current panorama for the \gls{mmlpbf} process was presented,
highlighting the subsystems that make up the \gls{mmlpbf} equipment, with particular emphasis on the powder delivery system, which is critical for multi-material fabrication. Several working principles and available technologies for this system were discussed, with the powder spreading with suction being the most mature technology, while the most promising one is based on the electrophotographic principle, the patterning drum.

The manufacturing chain was examined, focusing on the most significant challenges, including material variation across several directions in the powder-deposit system, joining dissimilar materials, powder contamination, numerous process parameters affecting the quality of the produced part, lack of design guidelines, manual pre-processing, and the absence of post-processing treatment research. The most relevant commercial equipment capable of multi-material fabrication was presented, with only a handful of these able to handle metallic powders, limiting their scope of application. Furthermore, the initial investment required for these machines is considerably high, which may explain why the majority of research literature focuses on upgrading mono-material LPBF equipment with multi-material powder delivery mechanisms, despite the need for manual pre-processing and manual programming of the equipment.

The prospects for improvement across the manufacturing chain were outlined. Tools for designing and simulating multi-material components manufacturable by the \gls{mmlpbf} process are required, and machine learning could have a significant impact on the process, by mining process data to provide sets of parameters and build strategies for producing components with better mechanical properties in a cost-effective manner.

Finally, the specific applications of interest to this work were presented, with the hip implant being a notable example. There is high demand for improved quality implants due to their high replacement rate and invasive procedures. They require multiple materials, including metals and composites, to perform effectively in various regions of the body, making them a perfect showcase for \gls{mmlpbf}.

In conclusion, \gls{mmlpbf} fabrication presents a multi-objective problem requiring
a specialized equipment and toolchain, as well as experimental design.
The present state of \gls{mmlpbf} pre-processing techniques is primarily focused
on addressing specific issues within the multi-material processing domain, and
lacks a holistic approach that could enhance knowledge transfer throughout the
manufacturing process. Additionally, these techniques are closely connected with
proprietary solutions that hinder customisation and impede the evolution of the
process. Therefore, there is a need for a comprehensive methodology to produce
multi-material metallic and composite components that can handle the inherent
complexity of the MMLPBF process and leverage its knowledge. This entails
establishing a \gls{mmlpbf} process database, as suggested by Gu et al.\cite{gu2012laser} and Mussatto\cite{mussatto2022research}, that is accessible to all relevant parties in the manufacturing chain.

%%% Local Variables:
%%% mode: latex
%%% TeX-master: "../template"
%%% End:
