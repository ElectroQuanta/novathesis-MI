%!TEX root = ../template.tex
%%%%%%%%%%%%%%%%%%%%%%%%%%%%%%%%%%%%%%%%%%%%%%%%%%%%%%%%%%%%%%%%%%%
%% chapter1.tex
%% NOVA thesis document file
%%
%% Chapter with introduction
%%%%%%%%%%%%%%%%%%%%%%%%%%%%%%%%%%%%%%%%%%%%%%%%%%%%%%%%%%%%%%%%%%%

\typeout{NT FILE INTRO.tex}%

%\chapter{Introduction}
%Hello
%
%%\prependtographicspath{{Chapters/Figures/Covers/}}
%\definecolor{Schlgray}{gray}{0.95}

%%\foreach \style in \defaultstyles {
%%    \chapterstyle{\style}
%%    \chapter{\style\ Style}
%%}
%%
%%% Preview additional styles
%%\foreach \style in \additionalstyles {
%%    \chapterstyle{\style}
%%    \chapter{\style\ Style}
%%}


% CHAPTER - Introduction -------------------------
% Preview default styles
\chapter{Introduction}%
\label{cha:introduction}

% \begin{ntquote}[14cm][Galileo][Somewhere in a book or speach][Astronomer, physicist and engineer][en]
%  You cannot teach a man anything; you can only help him discover it in himself.%
% \end{ntquote}
% \begin{ntquote}[20cm][Alan Kay][Computer scientist]
% The best way to predict the future is to invent it.%
% \end{ntquote}
% Syntax - all arguments are optional and can be omitted from the last to the first
% \begin{ntquote}
        % [max width of citation box]
        % [author name]
        % [where/source]
        % [profession]
        % [lang]
% YOUR TEXT HERE

% \end{ntquote}
%
\begin{quote}
\begin{flushright}
``\emph{The best way to predict the future is to invent it.}'' \\
\textbf{-- Alan Kay}, computer scientist
\end{flushright}
\end{quote}
  
% \begin{alignedquote}{l}{Albert Einstein}
%     Imagination is more important than knowledge.
% \end{alignedquote}

%\glsxtrfullpl{uav}, \glsxtrfullpl{uas}, or more commonly drones, are a class of
%unmanned robotic vehicles that can execute flying missions and carry payloads,
%guided either by remote control stations or in an autonomous
%way~\cite{alladi2022UAVBlockain,glossner2021overview}.
%
%The \glsxtrfull{uav}s market is booming. In 2017 the North America market had a revenue of
%737 million USD dollars and a eight-fold increase is expected for 2026 with a
%staggering 6.7 billion USD dollars, with strong contributions from the sectors
%of agriculture and farming, and security and law
%enforcement~\cite{mohsan2022towards}.
%
%The versatility and utility of \glspl{uav} is evident in its wide
%range of applications, performing tasks with high added value and that would be
%somewhat hard or impossible for a person to achieve: rescue operations and
%saving lifes, agriculture and farming, building structures, pipeline
%inspections, delivering goods and medical supplies, video capturing and filming,
%surveying, inventory management, providing telecommunications in remote areas,
%among others~\cite{alladi2022UAVBlockain}.
%
%However, only recently regulations have been explicitly enforced on \glspl{uav},
%with many countries allowing drones to fly over populated areas (at altitudes
%lower than 150 m)~\cite{nassi2021sok}. This poses several safety and security
%concerns. First, because a drone's crash, unintentional or not, can severely
%harm people and goods. Secondly, because they are cyber-physical systems which
%are able to record sensitive and private information. Thus,
%it is critical to ensure the security of an \gls{uav} system.
%
%Furthermore, \gls{uav}'s applications have real-time constraints but with mixed
%criticality levels, i.e., the risk of failure is more severe in some case than others.
%For example, flight control (safety-critical) and video streaming (non-critical)
%coexist. A compromise in video software could cascade to flight systems, causing
%catastrophic failure—yet strict hardware separation increases weight and reduces
%flight time.
%
%This exemplifies a \gls{mcs}: systems executing tasks of differing criticality (safety) levels on shared hardware.
%However, the integration of mixed-criticality components onto a common hardware
%platform is not an easy task: the designer needs to meet the stringent
%\gls{swap-c} metrics and safety requirements
%provided by safety-critical standards, such as those for automotive~\cite{iso26262}, avionics~\cite{johnson1998sw},
%and railway~\cite{cenelec201250128} industries.
%While \gls{mcs} challenges are well-studied in automotive/avionics, UAVs face unique
%constraints: extreme SWaP-C limitations and the need for fail-operational safety
%in uncontrolled environments. Current solutions either compromise isolation
%(single-board) or add weight/energy overhead (multi-board).
%
%As a result of this integration
%complexity increases, requiring higher computational power, and thus, these
%hardware platforms are migrating from single cores to multi-cores and in the
%future many-core architectures~\cite{burns2022mixed}.
%This leads to the core challenge: reconciling \emph{partitioning} for safety
%assurance with \emph{sharing} for efficiency.
%Consequently, problems arise in the modelling, design, implementation, and
%verification of the required hardware and software~\cite{burns2022mixed}.
%Prior research diverges into theoretical (scheduling-focused) and practical (partitioning-focused) approaches, but their incompatibility leaves \gls{uav} consolidation unresolved~\cite{burns2022mixed}. 
%
%This thesis bridges this gap by proposing the \gls{sspfs}: a Bao
%hypervisor-based architecture consolidating mixed-criticality functions on a
%single \gls{uav} platform. This solutions demonstrated robust isolation (malicious non-critical
%crashes don't impact flight control), minimal overhead (<2\% in PX4 tasks's
%scheduling and camera's frame rate), and real-flight validation -- enabling secure, lightweight UAVs without safety compromises while meeting stringent \gls{swap-c} constraints.

%\section{Motivation}
%Mixed-criticality systems such as \gls{uav} applications have stringent
%requirements on temporal and criticality-level, but also in the security level.
%
%The present works addresses this by developing a trustworthy software stack for
%\gls{uav} applications that can

\section{Introduction}
\label{sec:intro}
Unmanned Aerial Vehicles (\glspl{uav}) are increasingly deployed across
safety-critical domains—from inspection and mapping to public safety and
logistics—where onboard autonomy coexists with mission applications under tight
\gls{swap-c} constraints. In these settings, the flight-control stack
(safety-critical) must run alongside bandwidth- and compute-intensive mission
software (non-critical), yet remain protected against timing interference,
memory corruption, and fault propagation. Conventional multi-board designs
(dedicated \gls{fmu} plus companion computer) provide isolation at the expense
of weight, power, and inter-board latency; single-board solutions based on
containers or general-purpose hypervisors improve integration but fall short on
analyzable real-time guarantees and small, auditable \gls{tcb}s.

This thesis investigates how to \emph{safely consolidate} mixed-criticality
\gls{uav} workloads on shared hardware while preserving temporal, spatial, and
fault-containment isolation. We adopt a static-partitioning hypervisor approach
centered on Bao, an open, lightweight type-I hypervisor, and evaluate it in a
representative video-surveillance use case. Our aim is a trustworthy reference
stack that is practical on commodity platforms, transparent by design, and
measurable end-to-end.

\section{Motivation}
\label{sec:motivation}
\noindent\textbf{Operational pressure.}
Field operations increasingly require live video, perception, and connectivity (e.g., LTE/5G), which compete for CPU, memory, cache, and I/O bandwidth with control loops. Offloading to a companion computer raises \gls{swap-c}, complicates power/thermal budgets, and introduces link latency/jitter between control and mission domains.

\noindent\textbf{Assurance pressure.}
Safety-critical flight control demands bounded worst-case latency and strong fault containment. Containerization shares a kernel and cannot, alone, provide spatial or fault isolation across domains. General-purpose virtualization offers better separation but typically carries a large \gls{tcb} and host dependencies that frustrate certification and analysis.

\noindent\textbf{Platform pressure.}
Commodity SBCs (e.g., Raspberry~Pi~4) and emerging heterogeneous SoCs make single-board integration attractive, but predictable device assignment, DMA safety, interrupt delivery, and cache interference remain practical hurdles. An approach that is small, analyzable, and open is needed to both \emph{demonstrate} and \emph{measure} trustworthy consolidation on such platforms.

\noindent\textbf{Design choice.}
Static partitioning (1:1 vCPU:pCPU, pass-through I/O, two-stage translation) minimizes hypervisor complexity and scheduling overheads, enabling clear timing models and a reduced attack surface. Bao provides this with a small codebase and active open-source development, making it an appropriate foundation for a reference \gls{uav} stack.

\section{Goals}
\label{sec:goals}
\noindent\textbf{Primary goal.}
Design, implement, and evaluate a trustworthy, open-source reference stack that consolidates safety-critical flight control and non-critical mission workloads on shared hardware using the Bao hypervisor, while preserving temporal, spatial, and fault-containment isolation under realistic video-surveillance missions.

\noindent\textbf{Objectives.}
\begin{enumerate}[leftmargin=*,itemsep=0.25em]
  \item \textit{Architecture}: Specify two baselines—\gls{uspfs} (unsupervised, single-board) and \gls{sspfs} (supervised atop Bao)—including guest/device mapping, boot flow, and inter-domain communication.
  \item \textit{Mechanisms}: Provide disciplined device assignment and a mailbox-supervision mechanism to safely share \gls{pcie}–attached peripherals across \glspl{vm}.
  \item \textit{Implementation}: Realize both systems on the chosen \gls{uavic} (Raspberry~Pi~4~+~PilotPi) with PX4 for flight control and a live video pipeline for the mission domain.
  \item \textit{Evaluation}: 
    \begin{enumerate}[label*=\alph*),itemsep=0.1em]
      \item Establish \gls{uspfs} baselines; quantify \gls{sspfs} overheads (CPU, memory, energy) using MiBench \glsxtrfull{aics}.
      \item Measure interference (cache, DMA/PCIe, interrupt paths) and the effect of mailbox supervision.
      \item Validate functionality and isolation in lab and real-flight (tracking accuracy, scheduling behavior, frame rate).
    \end{enumerate}
  \item \textit{Openness}: Release configurations and code to enable reproduction and community scrutiny.
\end{enumerate}

\noindent\textbf{Research questions.}
\begin{enumerate}[leftmargin=*,itemsep=0.25em]
  \item \textbf{RQ1:} Can static partitioning via Bao maintain flight-control timing while co-locating a live video workload on commodity hardware?
  \item \textbf{RQ2:} What is the performance/energy overhead of \gls{sspfs} versus \gls{uspfs}, and how does interference manifest across caches, interrupts, and I/O?
  \item \textbf{RQ3:} Does mailbox supervision enable safe, practical sharing of \gls{pcie} devices between domains without violating isolation?
\end{enumerate}

\noindent\textbf{Scope and assumptions.}
We target small/medium multirotors, focus on onboard consolidation (not ground-side networks), and assume standard platform firmware and drivers. Adversarial testing emphasizes isolation (faults, misbehaving guests) rather than radio jamming or GNSS spoofing.


\section{Goals}
The main goal of the present work is to develop of a trustworthy
open-source software stack for \gls{uav} applications in which security and
safety are paramount, thus narrowing the gap between open-source and commercial
solutions and contributing for the widespread adoption of secure and safe features.
%
To this end, we define the following objectives:
\begin{enumerate}
\item Analyze the state of the art in mixed-criticality \gls{uav} systems and
  identify improvement opportunities, with a focus on hypervisor-based
  supervision.
\item Select a representative mixed-criticality application (e.g., flight
  control with video surveillance) to provide a concrete use case for
  consolidating the \gls{mcs} software stacks on a single platform in a secure
  manner.
\item Derive application requirements and choose a target hardware platform that
  offers security primitives essential for hypervisor-based isolation (e.g.,
  virtual memory, memory protection, TrustZone, etc.) while meeting \gls{swap-c}
  constraints.
\item Design and implement an open-source software stack for \gls{uav}
  applications that leverages the Bao hypervisor, and apply it to the chosen
  \gls{mcs} use case to consolidate the software stacks while preserving
  isolation between domains.
\item Validate the solution and assess its resilience under malicious compromise
  attempts, expecting the supervised solution to contain faults while the
  unsupervised one propagates them, leading to system failure and
  potentially a catastrophic \gls{uav} crash.
\item Quantify the performance overhead introduced by supervision and the effect
  of inter-guest interference using a standard benchmark suite (e.g., MiBench
  \gls{aics}), and explore mitigation techniques (e.g., cache coloring).
\item Benchmark each guest separately using application-specific metrics, and
  use these results to compare the unsupervised and supervised solutions.
\item Evaluate both solutions in real-flight conditions and analyze their
  behavior when each system is compromised.
\end{enumerate}

\section{Document Structure}
\label{sec:doc-structure}
This thesis is organized into six chapters that progressively motivate, design, implement, and evaluate a trustworthy \gls{uav} software stack:

\begin{itemize}
\item
\textbf{Chapter \ref{ch:state-art} -- Background and Related Work:}
Introduces mixed-criticality systems and virtualization, with emphasis on
hypervisors (notably Bao). Surveys reference \gls{uav} hardware/software
(open-source and commercial), then reviews related work and identifies gaps in
security, safety, and \gls{swap-c} efficiency that motivate this thesis.

\item
\textbf{Chapter \ref{ch:design} -- Design:}
Presents the \glsxtrfull{sspfs} architecture, using the Bao hypervisor to
consolidate flight-control and companion functions on one platform. Contrasts
supervised \gls{sspfs} with unsupervised baselines -- multi-platform
(\gls{umpfs}) and single-platform (\gls{uspfs}) -- and justifies design choices
for a video-surveillance use case. Details hardware/guest mapping under static
partitioning and introduces a Bao mailbox-supervision mechanism to safely
mediate shared \gls{pcie} device access.

\item
\textbf{Chapter \ref{cha:implementation} -- Implementation:}
Describes the realization of \gls{uspfs} and \gls{sspfs} on the \gls{uavic}
platform (Raspberry Pi 4 + PilotPi): hardware bring-up, PX4 and video pipeline
integration, Bao configuration, \gls{vm} separation, and the mailbox-supervision
driver.

\item
\textbf{Chapter \ref{cha:evaluation} -- Evaluation:}
Assesses functionality, performance, and isolation. Establishes the \gls{uspfs}
baseline, then quantifies \gls{sspfs} overheads (MiBench \glsxtrfull{aics}),
guest-to-guest interference (including cache and device effects), and the impact
of mailbox supervision. Compares single- vs.\ dual-\gls{vm} setups using
guest-specific metrics (e.g., PX4 scheduling overhead, camera frame rate), and
validates both systems in real flight with an automated mission (tracking
accuracy and resource usage), including repetition of functional tests in
flight.

\item
\textbf{Chapter \ref{cha:concl} -- Conclusions and Future Work:}
Summarizes how \gls{sspfs} consolidates mixed-criticality stacks on shared
hardware with robust isolation and modest overhead. Outlines future directions:
broader \gls{uavic} support, advanced shared-resource management, and extended
testing (long-duration flights, adversarial scenarios).

\item
\textbf{Appendices:}
Provide supplemental material (e.g., UAV taxonomy). All configurations, code,
and demonstration videos are available online (see~\cite{thesis-sw-github}).
\end{itemize}



%%% Local Variables:
%%% mode: latex
%%% TeX-master: "../template"
%%% End:
