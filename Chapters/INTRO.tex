%!TEX root = ../template.tex
%%%%%%%%%%%%%%%%%%%%%%%%%%%%%%%%%%%%%%%%%%%%%%%%%%%%%%%%%%%%%%%%%%%
%% chapter1.tex
%% NOVA thesis document file
%%
%% Chapter with introduction
%%%%%%%%%%%%%%%%%%%%%%%%%%%%%%%%%%%%%%%%%%%%%%%%%%%%%%%%%%%%%%%%%%%

\typeout{NT FILE INTRO.tex}%

%\chapter{Introduction}
%Hello
%
%%\prependtographicspath{{Chapters/Figures/Covers/}}
%\definecolor{Schlgray}{gray}{0.95}

% CHAPTER - Introduction -------------------------
\chapter{Introduction}%
\label{cha:introduction}
\glsxtrfull{uav}s, \glsxtrfull{uas}s, or more commonly drones, are a class of
unmanned robotic vehicles that can execute flying missions and carry payloads,
guided either by remote control stations or in an autonomous
way~\cite{alladi2022UAVBlockain,glossner2021overview}.

The \glsxtrfull{uav}s market is booming. In 2017 the North America market had a revenue of
737 million USD dollars and a eight-fold increase is expected for 2026 with a
staggering 6.7 billion USD dollars, with strong contributions from the sectors
of agriculture and farming, and security and law
enforcement~\cite{mohsan2022towards}.

The versatility and utility of \glspl{uav} are well displayed is by its wide
range of applications, performing tasks with high added value and that would be
somewhat hard or impossible for a person to achieve: rescue operations and
saving lifes, agriculture and farming, building structures, pipeline
inspections, delivering goods and medical supplies, video capturing and filming,
surveying, inventory management, providing telecommunications in remote areas,
among others~\cite{alladi2022UAVBlockain}.

However, only recently regulations have been explicitly enforced on \glspl{uav},
with many countries allowing drones to fly over populated areas (at altitudes
lower than 150 m)~\cite{nassi2021sok}. This poses several safety and security
concerns. First, because a drone's crash, unintentional or not, can severely
harm people and goods. Secondly, because they are cyber-physical systems which
are able to record sensitive and private information. Thus,
it is critical to ensure the security of an \gls{uav} system.

Furthermore, \gls{uav}'s applications have real-time constraints but with mixed
criticality levels, i.e., the risk of failure is more severe in some case than others.
For example, lets consider an automotive infotainment system that displays
navigation and rear-view camera information. The navigation information is a
commodity for the user, and if, for example, the radio volume suddenly
increases, although uncomfortable, it poses no safety risk for the user. On the
other hand, the rear-view camera information assists the user in parking
maneuvers, and, if a failure occurs, the vehicle may collide causing damage or
harming people.

This is an example of a \gls{mcs}, i.e., a system comprised of computer
\gls{hw} and \gls{sw} that executes several tasks/applications of different
criticality (safety) levels. Furthermore, as this example illustrates, an increasingly relevant trend in the design of real-time and embedded systems
is the integration of mixed-criticality components onto a common hardware
platform while trying to meet the stringent \gls{swap-c} and safety requirements
provided by safety-critical standards, such as those for automotive~\cite{iso26262}, avionics~\cite{rtca1992software},
and railway~\cite{cenelec201250128} industries.

As a result of this integration
complexity increases, requiring higher computational power, and thus, these
\gls{hw} platforms are migrating from single cores to multi-cores and in the
future many-core architectures~\cite{burns2022mixed}. This leads to the
fundamental question of how to reconcile the
conflicting requirements of \emph{partitioning} for (safety) assurance and \emph{sharing} for
efficient resource usage. Consequently, problems arise in the modelling, design,
implementation, and verification of the required \gls{hw} and
\gls{sw}~\cite{burns2022mixed}.

Two major areas of research stem from this dichotomy: a more theorethical one,
largely focused on using criticality-specific \glspl{wcet} to schedule systems
at each criticality level, but at the expenses of high processor utilization; a
more practical one, focused on the safe partitioning of a system with the
sharing of computational and communication resources, but with increased
complexity.
Unfortunately, the combination of both areas is not easy: flexible scheduling
requires, at least, dynamic partitioning, whereas certified systems require
complete separation or, at least, static partitioning. This mismatch needs to be
addressed in future work~\cite{burns2022mixed}.

%\section{Motivation}
%Mixed-criticality systems such as \gls{uav} applications have stringent
%requirements on temporal and criticality-level, but also in the security level.
%
%The present works addresses this by developing a trustworthy \gls{sw} stack for
%\gls{uav} applications that can 

\section{Goals}
The main goal of the present work is the development of a trustworthy
open-source software stack for \gls{uav} applications, where security and safety
are paramount, thus closing the gap between open-source and commercial solutions
and contributing for the widespread adoption of secure and safe features.

To this end several main objectives have been outlined:
\begin{enumerate}
\item Identify the \gls{uav} application requirements and select the target
  \gls{hw} platform with several security primitives
  (e.g, virtual memory, memory protection, TrustZone, criptographic
  accelerators, etc.).
\item Design an open-source \gls{sw} software stack using as reference a
  \gls{sph}, namely Bao.
\item Provide support for autonomous mode (autopilot).
\item Implement the device drivers for UAV's control.
\item Perform the remote update of the system, through an \gls{ota} mechanism.
\item Assess and demonstrate the security advantages of the UAV's prototype over
  others open-source competitors.
%\item Utilize a platform with several \gls{hw} primitives focused on security
%  (e.g, virtual memory, memory protection, TrustZone, criptographic
%  accelerators, etc.).
\end{enumerate}

\section{Document structure}
This thesis is organised as follows:
In Chapter~\ref{ch:state-art}, some background is provided concerning mixed criticality
systems, its relevance and the challenges it poses, alongside with the current
approach to address its complexity.
Next, \glspl{uav} are discussed,
namely the reference hardware and software, both as open-source and commercial
solutions. Lastly, the related work is
presented, showing the current research directions and topics.%
%% the additive
%% manufacturing technology, \gls{lpbf}, and \gls{mmam} is presented, with special
%% focus on the last, namely on \gls{fgm} structures. Lastly, a brief overview of
%% the available methodologies in these fields are presented.

%% Chapter~\ref{ch:theor-found} lays out the theoretical foundations for this work,
%% namely the project development methodologies and associated tools,
%% markup languages, and database management systems.
%% %the \gls{lpbf} process in detail.
%% %This chapter can be skipped without 
%% 
%% Chapter~\ref{ch:prob-challenge} presents the multi-material design and production
%% problem and its challenges using the \gls{lpbf} technology. The methodology
%% devised for multi-material production via \gls{lpbf} technology is presented to tackle the high complexity of the process
%% and the lack of a supporting methodology, taking into account the key agents of
%% the process and leveraging the process information.
%% 
%% Chapter~\ref{ch:development} shows all the development phase of the
%% project. A specific workflow was instantiated from
%% the methodology, attending to the specific requirements and constraints of the
%% project. Based on this workflow, a toolchain was assembled, designing the
%% required software components. Finally, based on the requirements and constraints
%% of the process itself, the mechanical and electronic infrastructures were
%% designed, and on top of the last, the control software was designed.
%% 
%% In Chapter~\ref{ch:application}, the workflow, and the equipment were put to the test
%% to verify their suitability to the process and their performance for
%% multi-material component production. Additionally, production manufacturing
%% tests were also performed. Tests were used to validate the workflow and
%% equipment, pointing out also straightforward ways to adapt and implement custom
%% paths for the multi-material \gls{lpbf} process.
%% 
%% The Chapter~\ref{ch:conclusion} gives a summary of this thesis as well as
%% prospect for future work.
%% 
%% Lastly, the appendices contain detailed information about the toolchain, wiring schematics, use cases, EzCAD \gls{api}, and sequence
%% diagrams, and the annexes contain the paper submitted, stemming out of this work.

%%% Local Variables:
%%% mode: latex
%%% TeX-master: "../template"
%%% End:
