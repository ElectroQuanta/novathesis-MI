%!TEX root = ../template.tex
%%%%%%%%%%%%%%%%%%%%%%%%%%%%%%%%%%%%%%%%%%%%%%%%%%%%%%%%%%%%%%%%%%%
%% chapter1.tex
%% NOVA thesis document file
%%
%% Chapter with introduction
%%%%%%%%%%%%%%%%%%%%%%%%%%%%%%%%%%%%%%%%%%%%%%%%%%%%%%%%%%%%%%%%%%%

\typeout{NT FILE INTRO.tex}%

%\chapter{Introduction}
%Hello
%
%%\prependtographicspath{{Chapters/Figures/Covers/}}
%\definecolor{Schlgray}{gray}{0.95}

%%\foreach \style in \defaultstyles {
%%    \chapterstyle{\style}
%%    \chapter{\style\ Style}
%%}
%%
%%% Preview additional styles
%%\foreach \style in \additionalstyles {
%%    \chapterstyle{\style}
%%    \chapter{\style\ Style}
%%}


% CHAPTER - Introduction -------------------------
% Preview default styles
\chapter{Introduction}%
\label{cha:introduction}

% \begin{ntquote}[14cm][Galileo][Somewhere in a book or speach][Astronomer, physicist and engineer][en]
%  You cannot teach a man anything; you can only help him discover it in himself.%
% \end{ntquote}
% \begin{ntquote}[20cm][Alan Kay][Computer scientist]
% The best way to predict the future is to invent it.%
% \end{ntquote}
% Syntax - all arguments are optional and can be omitted from the last to the first
% \begin{ntquote}
        % [max width of citation box]
        % [author name]
        % [where/source]
        % [profession]
        % [lang]
% YOUR TEXT HERE

% \end{ntquote}
%
\begin{quote}
\begin{flushright}
``\emph{The best way to predict the future is to invent it.}'' \\
\textbf{-- Alan Kay}, computer scientist
\end{flushright}
\end{quote}
  
% \begin{alignedquote}{l}{Albert Einstein}
%     Imagination is more important than knowledge.
% \end{alignedquote}

\glsxtrfullpl{uav}, \glsxtrfullpl{uas}, or more commonly drones, are a class of
unmanned robotic vehicles that can execute flying missions and carry payloads,
guided either by remote control stations or in an autonomous
way~\cite{alladi2022UAVBlockain,glossner2021overview}.

The \glsxtrfull{uav}s market is booming. In 2017 the North America market had a revenue of
737 million USD dollars and a eight-fold increase is expected for 2026 with a
staggering 6.7 billion USD dollars, with strong contributions from the sectors
of agriculture and farming, and security and law
enforcement~\cite{mohsan2022towards}.

The versatility and utility of \glspl{uav} is evident in its wide
range of applications, performing tasks with high added value and that would be
somewhat hard or impossible for a person to achieve: rescue operations and
saving lifes, agriculture and farming, building structures, pipeline
inspections, delivering goods and medical supplies, video capturing and filming,
surveying, inventory management, providing telecommunications in remote areas,
among others~\cite{alladi2022UAVBlockain}.

However, only recently regulations have been explicitly enforced on \glspl{uav},
with many countries allowing drones to fly over populated areas (at altitudes
lower than 150 m)~\cite{nassi2021sok}. This poses several safety and security
concerns. First, because a drone's crash, unintentional or not, can severely
harm people and goods. Secondly, because they are cyber-physical systems which
are able to record sensitive and private information. Thus,
it is critical to ensure the security of an \gls{uav} system.

Furthermore, \gls{uav}'s applications have real-time constraints but with mixed
criticality levels, i.e., the risk of failure is more severe in some case than others.
For example, flight control (safety-critical) and video streaming (non-critical)
coexist. A compromise in video software could cascade to flight systems, causing
catastrophic failure—yet strict hardware separation increases weight and reduces
flight time.

This exemplifies a \gls{mcs}: systems executing tasks of differing criticality (safety) levels on shared hardware.
However, the integration of mixed-criticality components onto a common hardware
platform is not an easy task: the designer needs to meet the stringent
\gls{swap-c} metrics and safety requirements
provided by safety-critical standards, such as those for automotive~\cite{iso26262}, avionics~\cite{johnson1998sw},
and railway~\cite{cenelec201250128} industries.
While \gls{mcs} challenges are well-studied in automotive/avionics, UAVs face unique
constraints: extreme SWaP-C limitations and the need for fail-operational safety
in uncontrolled environments. Current solutions either compromise isolation
(single-board) or add weight/energy overhead (multi-board).

As a result of this integration
complexity increases, requiring higher computational power, and thus, these
hardware platforms are migrating from single cores to multi-cores and in the
future many-core architectures~\cite{burns2022mixed}.
This leads to the core challenge: reconciling \emph{partitioning} for safety
assurance with \emph{sharing} for efficiency.
Consequently, problems arise in the modelling, design, implementation, and
verification of the required hardware and software~\cite{burns2022mixed}.
Prior research diverges into theoretical (scheduling-focused) and practical (partitioning-focused) approaches, but their incompatibility leaves \gls{uav} consolidation unresolved~\cite{burns2022mixed}. 

% Two major areas of research stem from this dichotomy: a more theorethical one,
% largely focused on using criticality-specific \glspl{wcet} to schedule systems
% at each criticality level, but at the expenses of high processor utilization; a
% more practical one, focused on the safe partitioning of a system with the
% sharing of computational and communication resources, but with increased
% complexity.
% Unfortunately, the combination of both areas is not easy: flexible scheduling
% requires, at least, dynamic partitioning, whereas certified systems require
% complete separation or, at least, static partitioning. This mismatch needs to be
% addressed in future work~\cite{burns2022mixed}.

This thesis bridges this gap by proposing the \gls{sspfs}: a Bao
hypervisor-based architecture consolidating mixed-criticality functions on a
single UAV platform. This solutions demonstrated robust isolation (malicious non-critical
crashes don't impact flight control), minimal overhead (<2\% in PX4 tasks's
scheduling and camera's frame rate), and real-flight validation -- enabling secure, lightweight UAVs without safety compromises while meeting stringent \gls{swap-c} constraints.

%\section{Motivation}
%Mixed-criticality systems such as \gls{uav} applications have stringent
%requirements on temporal and criticality-level, but also in the security level.
%
%The present works addresses this by developing a trustworthy software stack for
%\gls{uav} applications that can 

\section{Goals}
The main goal of the present work is to develop of a trustworthy
open-source software stack for \gls{uav} applications in which security and
safety are paramount, thus narrowing the gap between open-source and commercial
solutions and contributing for the widespread adoption of secure and safe features.
%
To this end, we define the following objectives:
\begin{enumerate}
\item Analyze the state of the art in mixed-criticality \gls{uav} systems and
  identify improvement opportunities, with a focus on hypervisor-based
  supervision.
\item Select a representative mixed-criticality application (e.g., flight
  control with video surveillance) to provide a concrete use case for
  consolidating the \gls{mcs} software stacks on a single platform in a secure
  manner.
\item Derive application requirements and choose a target hardware platform that
  offers security primitives essential for hypervisor-based isolation (e.g.,
  virtual memory, memory protection, TrustZone, etc.) while meeting \gls{swap-c}
  constraints.
\item Design and implement an open-source software stack for \gls{uav}
  applications that leverages the Bao hypervisor, and apply it to the chosen
  \gls{mcs} use case to consolidate the software stacks while preserving
  isolation between domains.
\item Validate the solution and assess its resilience under malicious compromise
  attempts, expecting the supervised solution to contain faults while the
  unsupervised one propagates them, leading to system failure and
  potentially a catastrophic \gls{uav} crash.
\item Quantify the performance overhead introduced by supervision and the effect
  of inter-guest interference using a standard benchmark suite (e.g., MiBench
  \gls{aics}), and explore mitigation techniques (e.g., cache coloring).
\item Benchmark each guest separately using application-specific metrics, and
  use these results to compare the unsupervised and supervised solutions.
\item Evaluate both solutions in real-flight conditions and analyze their
  behavior when each system is compromised.
\end{enumerate}

\section{Document Structure}
\label{sec:doc-structure}
This thesis is organized into six chapters that systematically address the
design, implementation, and validation of a trustworthy \gls{uav} software
stack:

\begin{itemize}
\item
  Chapter \ref{ch:state-art} -- \textbf{Background and Related Work}:
Presents foundational concepts of mixed-criticality systems and virtualization,
with a special focus in the hypervisor technology, more specifically the Bao
hypervisor. Current
\gls{uav} hardware/software solutions are analyzed in depth in both the
open-source and commercial sphere. The critical gaps in security,
safety, and hardware efficiency are identified through comparative assessment of
open-source and commercial platforms. Lastly, the related work is presented, showing the current research directions and topics.

\item 
Chapter \ref{ch:design} -- \textbf{Design}:
Proposes the \glsxtrfull{sspfs} architecture leveraging the Bao hypervisor to
consolidate flight control and companion functions on a single hardware
platform. Contrasts this with unsupervised multi-platform (\gls{umpfs}) and
single-platform (\gls{uspfs}) approaches, analyzing trade-offs in weight
reduction, isolation guarantees, and security risks. The system architecture was
tailored for a video surveillance application based on the PX4 flight stack. Due
to the static partitioning nature of the Bao hypervisor, the hardware needed to
be selected and mapped across guests. However, as some collisions occurred, a
mechanism for mailbox access supervision in the Bao hypervisor was devised. This
mechanism enables both \glspl{vm} to seamlessly communicate with the
\gls{uavic}'s firmware.

\item 
Chapter \ref{cha:implementation} -- \textbf{Implementation}:
Details the realization of both \gls{uspfs} (unsupervised) and \gls{sspfs}
(supervised) solutions on the \gls{uavic} platform (Raspberry Pi 4 +
PilotPi). Covers hardware validation, PX4/video surveillance stack integration,
and Bao hypervisor deployment with VM separation. Includes the custom mailbox
supervision mechanism for shared device access.

\item 
Chapter \ref{cha:evaluation} --  \textbf{Evaluation}:
Conducts comprehensive functional tests, performance benchmarking (using MiBench
\gls{aics}), and real-flight validation. The behavior of the \gls{uspfs} and
\gls{sspfs} systems is analyzed when the system is compromised, assessing the
isolation effectiveness under malicious attacks.
The baseline performance of the \gls{uspfs} system is
established and the performance degradation of the \gls{sspfs} system as a
consequence of the introduction of the Bao's hypervisor is assessed using the
MiBench \glsxtrfull{aics}.
Furthermore, the impact of the interference between guests is analyzed, as well
as the impact of the mailbox driver patch. The guests are also benchmarked in
single- and dual-\gls{vm} configuration, considering guest-specific metrics, as
the PX4 tasks' scheduling overhead and the camera frame rate.
Lastly, both systems are evaluated in a real flight scenario, using an automated
mission to compare the position tracking and system resource usage across
systems. The functional tests are also repeated, but now in real flight.
    
\item 
Chapter \ref{cha:concl} -- \textbf{Conclusions and Future Work}:
Synthesizes findings on the \gls{sspfs}'s ability to consolidate the \gls{uav}'s
mixed-criticality stacks into a single hardware platform, while providing robust isolation and minimal overhead. Discusses prospects
for future work, including the support for different \gls{uavic} platforms,
advanced shared resource management, and more extensive and detailed testing,
especially in long-duration flights and attack scenarios.

\item
\textbf{Appendices}: Contain supplementary material including: \gls{uav}'s taxonomy, source code
listings, configuration details, and log files.
\end{itemize}


%%% Local Variables:
%%% mode: latex
%%% TeX-master: "../template"
%%% End:
