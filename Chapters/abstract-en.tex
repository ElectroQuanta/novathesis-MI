%!TEX root = ../template.tex
%%%%%%%%%%%%%%%%%%%%%%%%%%%%%%%%%%%%%%%%%%%%%%%%%%%%%%%%%%%%%%%%%%%%
%% abstract-en.tex
%% NOVA thesis document file
%%
%% Abstract in English([^%]*)
%%%%%%%%%%%%%%%%%%%%%%%%%%%%%%%%%%%%%%%%%%%%%%%%%%%%%%%%%%%%%%%%%%%%

\typeout{NT FILE abstract-en.tex}%

%The booming market for \glspl{uav} brings forth significant challenges
%concerning security and safety, often overlooked in current system designs. \gls{uav}
%applications inherently possess real-time constraints with mixed-criticality
%levels, meaning the risk of failure varies in severity. For instance, a failure
%in video surveillance, a non-critical function, could propagate to the critical
%flight controller, leading to catastrophic consequences like a crash and harm to
%people or goods. Traditional multi-platform flight stacks for \glspl{uav}, while
%addressing mixed-criticality by separating functions onto different hardware
%nodes, undesirably increase the \gls{uav}'s weight and footprint and often lack
%isolation guarantees. Moreover, existing open-source solutions often fall short
%in meeting mixed-criticality and security requirements, while commercial
%alternatives frequently lack transparency.
%%
%To address these critical issues, this work proposes the \glsxtrfull{sspfs}, an innovative approach that integrates the
%flight controller and companion computer functions on a single Raspberry Pi 4 +
%PilotPi shield platform under the Bao hypervisor. This consolidation is achieved by leveraging the Bao hypervisor, a
%lightweight, security, and safety-oriented bare-metal hypervisor designed for
%embedded real-time systems and mixed-criticality systems.
%Bao was extended with a custom mailbox supervisor to mediate firmware
%% transactions over the shared \glsxtrfull{pcie} bus.
%transactions over a shared bus.
%The PX4 autopilot was selected due to its open-source nature, extensive platform
%support, modular architecture, and widespread adoption in the industry.
%An \gls{uspfs} counterpart was also developed to quantify the benefits of the supervision layer.
%%
%Experimental results demonstrate that \gls{sspfs} provides strong isolation: injected
%faults in the non-critical Companion \gls{vm} lead to its failure, while the critical
%PX4 \gls{vm} maintains full functionality, keeping the \gls{uav}
%airborne. Benchmarking with MiBench shows a small performance degradation and
%improved cache partitioning reduces interference.
%The worst-case scheduling overhead for PX4 tasks was very low (2\%). The camera's frame rate in the Companion VM also
%showed no statistically significant overhead (worst-case of 2\%). In some
%instances, the \gls{sspfs} even outperformed the \gls{uspfs} system, attributed
%to the static partitioning of resources.
%Real-flight tests confirm accurate position tracking and a modest
%CPU overhead (≈6 \% average), validating
%that \gls{sspfs} meets real-time requirements without sacrificing safety. This
%study establishes a viable path toward trustworthy, consolidated
%mixed-criticality \gls{uav} systems with minimal performance impact.
%% by providing
%% a trustworthy open-source reference software stack.
%
%Keywords: UAV, Mixed-Criticality Systems, Hypervisor, Bao, Security, PX4.

% The rapid growth of Unmanned Aerial Vehicles (UAVs) brings mixed-criticality
% demands: safety-critical flight control must coexist with non-critical,
% resource-intensive mission software under strict size, weight, power, and cost
% \gls{swap-c} limits. Conventional multi-board stacks provide isolation by hardware
% separation but add weight and link latency. Nonetheless, the hardware separation
% is not enough, as a breach in an non-critical system can still cascade to the
% critical one, because there is no true isolation between domains.
% Ongoing
% efforts in the consolation focus either on closed-source components
% (e.g., PikeOS or CLARE hypervisor) or are unmaintained and application-specific
% and platform-dependent (FlyOS).

% Other approaches focus only on securing the mission software either by
% containerization (Auterion) or by deploying formally verified components (seL4 +
% CAmkES).

% This thesis designs and evaluates a trustworthy, open consolidation of
% mixed-criticality UAV workloads using Bao, a lightweight type-I hypervisor with
% static partitioning. We implement two reference stacks on a Raspberry Pi 4 +
% PilotPi platform: an unsupervised single-platform flight stack (USPFS) and a
% supervised single-platform flight stack (SSPFS) in which PX4 (flight control)
% and a live video pipeline (mission) run as separate VMs. To safely share
% peripherals, we introduce a mailbox-supervision mechanism in Bao that mediates
% firmware transactions over a shared bus, preventing unintended cross-domain
% access.

% Experiments show strong isolation: injected faults in the non-critical
% (Companion) VM cause only that VM to fail, while the PX4 VM maintains control
% and keeps the UAV airborne. Using MiBench Automotive/Industrial Control System
% (AICS) workloads, performance overheads are small: PX4 task scheduling overhead
% is ~2\% (worst case), camera frame-rate overhead is ~2\% and not statistically
% significant, and average CPU overhead in real-flight traces is ≈6\%. Static
% partitioning bounds interference; cache-management measures further reduce
% cross-domain effects.

% These results demonstrate that a small, analyzable hypervisor can consolidate
% UAV flight-control and mission workloads on commodity hardware with minimal
% overhead, providing a practical, open path toward trustworthy mixed-criticality
% UAV systems.

% Keywords: UAV, mixed-criticality systems, hypervisor, Bao, PX4, virtualization,
% SWaP-C.

The rapid growth of unmanned aerial vehicles (UAVs) brings mixed-criticality
demands: safety-critical flight control must coexist with non-critical,
resource-intensive mission software under strict size, weight, power, and cost
(SWaP-C) limits. Conventional multi-board stacks add weight and link latency—and
hardware separation alone is insufficient: compromises in non-critical nodes can
cascade into critical functions via shared interfaces. Ongoing consolidation
efforts either depend on closed-source hypervisors
% (e.g., PikeOS, CLARE)
or rely on unmaintained, application-/platform-specific designs.
% (e.g., FlyOS).
Other
approaches harden only the mission side using containerization
% (Auterion)
or formally verified components,
% (e.g., seL4 + CAmkES),
but shared-kernel containers
cannot guarantee temporal, spatial, or fault isolation, and mission-only
hardening leaves critical links and the autopilot unprotected. These assumptions
are unsafe given UAVs' extensive attack surface, the bug density observed in
open-source autopilots, and vulnerabilities reported in commercial real-time
operating systems.
%
This thesis designs and evaluates a trustworthy, open consolidation of
mixed-criticality UAV workloads using Bao, a lightweight type-I hypervisor with
static partitioning. We implement two stacks on Raspberry Pi 4 + PilotPi: an
unsupervised single-platform flight stack (USPFS) -- a baseline for comparison
that co-locates mission and flight control without supervision -- and a
supervised single-platform flight stack (SSPFS) where PX4 (flight control) and a
live video pipeline (mission) run as separate guests. To safely share peripherals, we add a
mailbox-supervision mechanism to Bao that mediates firmware transactions over a
shared bus, preventing unintended cross-domain access.
%
Experiments show supervision-driven isolation: injected faults in the
non-critical guest cause only that guest to fail, while the PX4 guest maintains
control and keeps the UAV airborne. Offline benchmarking with the MiBench suite
workloads shows small overheads, corroborated in real-flight tests: PX4
task-scheduling (≈2\%), camera frame-rate (≈2\%), and average
CPU load (≈6\%). Static partitioning bounds interference, and cache
coloring further mitigates cross-domain effects.
%
Overall, a small, open-source static-partitioning hypervisor, such as Bao,
consolidates flight-control and mission workloads into a single platform with
minimal overhead, offering a practical, open path to trustworthy
mixed-criticality UAV systems, even in the face of large attack surfaces where
component-level assurances alone are not sufficient.

% Keywords: UAV, mixed-criticality systems, hypervisor, Bao, PX4, virtualization, SWaP-C.

\keywords{
  UAV \and
  mixed-criticality systems \and
  hypervisor \and
  Bao \and
  PX4 \and
  virtualization \and
  SWaP-C
}
%Keywords: UAV, mixed-criticality systems, hypervisor, Bao, PX4, virtualization, SWaP-C.
