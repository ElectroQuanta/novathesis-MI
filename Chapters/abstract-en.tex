%!TEX root = ../template.tex
%%%%%%%%%%%%%%%%%%%%%%%%%%%%%%%%%%%%%%%%%%%%%%%%%%%%%%%%%%%%%%%%%%%%
%% abstract-en.tex
%% NOVA thesis document file
%%
%% Abstract in English([^%]*)
%%%%%%%%%%%%%%%%%%%%%%%%%%%%%%%%%%%%%%%%%%%%%%%%%%%%%%%%%%%%%%%%%%%%

\typeout{NT FILE abstract-en.tex}%
The rapid growth of unmanned aerial vehicles (UAVs) brings mixed-criticality
demands: safety-critical flight control must coexist with non-critical,
resource-intensive mission software under strict size, weight, power, and cost
(SWaP-C) limits. Conventional multi-board stacks add weight and link latency, and
hardware separation alone is insufficient: compromises in non-critical nodes can
cascade into critical functions via shared interfaces. Ongoing consolidation
efforts either depend on closed-source hypervisors
% (e.g., PikeOS, CLARE)
or rely on unmaintained, application-/platform-specific designs.
% (e.g., FlyOS).
Other
approaches harden only the mission side using containerization
% (Auterion)
or formally verified components,
% (e.g., seL4 + CAmkES),
but shared-kernel containers
cannot guarantee temporal, spatial, or fault isolation, and mission-only
hardening leaves critical links and the autopilot unprotected. These assumptions
are unsafe given UAVs' extensive attack surface, the bug density observed in
open-source autopilots, and vulnerabilities reported in commercial real-time
operating systems.
%
This thesis designs and evaluates a trustworthy, open consolidation of
mixed-criticality UAV workloads using Bao, a lightweight type-I hypervisor with
static partitioning. We implement two stacks on Raspberry Pi 4 + PilotPi: an
unsupervised single-platform flight stack (USPFS) -- a baseline for comparison
that co-locates mission and flight control without supervision -- and a
supervised single-platform flight stack (SSPFS) where PX4 (flight control) and a
live video pipeline (mission) run as separate guests. To safely share peripherals, we add a
mailbox-supervision mechanism to Bao that mediates firmware transactions over a
shared bus, preventing unintended cross-domain access.
%
Experiments show supervision-driven isolation: injected faults in the
non-critical guest cause only that guest to fail, while the PX4 guest maintains
control and keeps the UAV airborne. Offline benchmarking with the MiBench suite
indicates small overheads, corroborated in application tests: PX4
task-scheduling ≈2\% and camera frame-rate ≈2\%. Real-flight tests demonstrate accurate position tracking and modest average CPU load (≈6\%). Static partitioning bounds interference, and cache
coloring further mitigates cross-domain effects.
%
Overall, a small, open-source static-partitioning hypervisor, such as Bao,
consolidates flight-control and mission workloads into a single platform with
minimal overhead, offering a practical, open path to trustworthy
mixed-criticality UAV systems, even in the face of large attack surfaces where
component-level assurances alone are not sufficient.

% Keywords: UAV, mixed-criticality systems, hypervisor, Bao, PX4, virtualization, SWaP-C.

\keywords{
  UAV \and
  mixed-criticality systems \and
  hypervisor \and
  Bao \and
  PX4 \and
  virtualization \and
  SWaP-C
}
%Keywords: UAV, mixed-criticality systems, hypervisor, Bao, PX4, virtualization, SWaP-C.
