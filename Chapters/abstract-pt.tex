%!TEX root = ../template.tex
%%%%%%%%%%%%%%%%%%%%%%%%%%%%%%%%%%%%%%%%%%%%%%%%%%%%%%%%%%%%%%%%%%%%
%% abstract-pt.tex
%% NOVA thesis document file
%%
%% Abstract in Portuguese
%%%%%%%%%%%%%%%%%%%%%%%%%%%%%%%%%%%%%%%%%%%%%%%%%%%%%%%%%%%%%%%%%%%%

\typeout{NT FILE abstract-pt.tex}%

O crescente mercado de \glspl{uav} apresenta desafios significativos em termos de segurança operacional e contra ameaças, frequentemente negligenciados nos projetos atuais de sistemas. As aplicações de \glspl{uav} possuem inerentemente restrições de tempo real com níveis mistos de criticidade, significando que o risco de falha varia em severidade. Por exemplo, uma falha na vigilância por vídeo, função não crítica, poderia propagar-se para o controlador de voo crítico, levando a consequências catastróficas como uma queda e danos a pessoas ou bens. As pilhas de voo tradicionais multiplataforma para \glspl{uav}, embora abordem a criticidade mista ao separar funções em diferentes nós de hardware, aumentam indesejavelmente o peso e a dimensão do \gls{uav} e frequentemente carecem de garantias de isolamento. Adicionalmente, as soluções open-source existentes geralmente não atendem plenamente aos requisitos de criticidade mista e segurança, enquanto as alternativas comerciais frequentemente carecem de transparência.
%
Para enfrentar estas questões críticas, este trabalho propõe o \glsxtrfull{sspfs}, uma abordagem inovadora que integra as funções do controlador de voo e do computador companheiro numa única plataforma Raspberry Pi 4 + PilotPi shield sob o hipervisor Bao. Esta consolidação é alcançada mediante o hipervisor Bao, um hipervisor bare-metal leve, orientado para segurança operacional e contra ameaças, projetado para sistemas embebidos de tempo real e de criticidade mista. O Bao foi estendido com um supervisor de mailbox personalizado para mediar transações de firmware através de um barramento partilhado.
O piloto automático PX4 foi selecionado devido à sua natureza open-source, amplo suporte a plataformas, arquitetura modular e ampla adoção na indústria. Uma contraparte \gls{uspfs} também foi desenvolvida para quantificar os benefícios da camada de supervisão.
%
Resultados experimentais demonstram que o \gls{sspfs} fornece forte isolamento:
falhas injetadas na \gls{vm} não crítica (Companion) levam à sua falha, enquanto
a \gls{vm} crítica (PX4) mantém funcionalidade total, mantendo o \gls{uav} no
ar. Benchmarking com o MiBench mostra uma pequena degradação de desempenho, e o
particionamento melhorado da cache reduz interferências. A sobrecarga de
escalonamento em worst-case para tarefas do PX4 foi muito baixa (2\%). A taxa de
captura da câmara na Companion VM também não apresentou sobrecarga estatisticamente significativa (worst-case de 2\%). Em alguns casos, o \gls{sspfs} até superou o sistema \gls{uspfs}, atribuído ao particionamento estático de recursos. Testes de voo real confirmam acompanhamento preciso de posição e uma modesta sobrecarga de CPU (≈6\% média), validando que o \gls{sspfs} atende aos requisitos de tempo real sem sacrificar segurança. Este estudo estabelece um caminho viável para sistemas consolidados de criticidade mista para \glspl{uav} confiáveis, com impacto mínimo no desempenho.

% Palavras-chave do resumo em Português
% \begin{keywords}
% Palavra-chave 1, Palavra-chave 2, Palavra-chave 3, Palavra-chave 4
% \end{keywords}
\keywords{
  UAV \and
  Sistemas de Criticidade mista \and
  Hipervisor \and
  Bao \and
  Segurança \and
  PX4
}

%Keywords: UAV, Mixed-Criticality Systems, Hypervisor, Bao, Security, PX4.
% to add an extra black line
