%!TEX root = ../template.tex
%%%%%%%%%%%%%%%%%%%%%%%%%%%%%%%%%%%%%%%%%%%%%%%%%%%%%%%%%%%%%%%%%%%%
%% abstract-pt.tex
%% NOVA thesis document file
%%
%% Abstract in Portuguese
%%%%%%%%%%%%%%%%%%%%%%%%%%%%%%%%%%%%%%%%%%%%%%%%%%%%%%%%%%%%%%%%%%%%

\typeout{NT FILE abstract-pt.tex}%

O rápido crescimento dos veículos aéreos não tripulados (\emph{UAV})
impõe exigências de criticalidade mista: o controlo de voo (crítico) tem de
coexistir com o \emph{software} não-crítico da missão obedecendo aos limites de
tamanho, peso, potência e custo (\emph{SWaP-C}). As plataformas convencionais, \emph{multi-board},
adicionam peso e latência nas comunicações e a separação por \emph{hardware} é
insuficiente, devido às interfaces partilhadas. Os esforços de consolidação em curso
dependem de hipervisores proprietários ou assentam em implementações que
dependem da platforma/aplicação ou que foram abandonados.
%
Outras abordagens focam-se apenas na pilha de \emph{software} da missão usando
contentores ou componentes verificados formalmente. O contentor não garante
o isolamento adequado, e ambos focam-se apenas na pilha da missão,
deixando as comunicações e o piloto automático desprotegidos.
Tais suposições são inseguras dada a extensa superfície de ataque dos
\emph{UAVs} e as vulnerabilidades reportadas nos pilotos automáticos de código
aberto e sistemas operativos de tempo real comerciais.
%
No presente trabalho concebemos e avaliamos uma consolidação de pilhas de software criticidade mista em \emph{UAVs} recorrendo ao Bao, um
hipervisor de particionamento estático de tipo I. Implementamos duas pilhas de
software numa única plataforma (Raspberry Pi 4 + PilotPi): uma não
supervisionada (\emph{USPFS}), usada como base de comparação, e uma supervisionada
(\emph{SSPFS}), onde o PX4 (piloto automático) e uma aplicação de vídeo em tempo
real (missão) executam como \emph{guests} separados. Para partilhar periféricos
com segurança, adicionamos ao Bao um mecanismo de supervisão por \emph{mailbox}
que medeia transações de \emph{firmware} num barramento partilhado, prevenindo
acessos indevidos entre domínios.
%
As experiências demonstram que as falhas injetadas
no \emph{guest} não crítico provocam apenas a falha deste, enquanto o
PX4 mantém o controlo e a \emph{UAV} em voo. A avaliação realizada com o MiBench
e nos testes de aplicação, indicam penalizações reduzidas no desempenho: 2\% no escalonamento de tarefas do PX4 e na captura de
vídeo. Os testes de voo demonstraram a robustez do controlo de posição e um
aumento reduzido na carga do \emph{CPU} (6\%). O particionamento estático
promove o isolamento entre domínios e o \emph{cache coloring} mitiga a interferência entre eles.
%
Em suma, um hipervisor de particionamento estático, como o Bao, permite consolidar as
pilhas de \emph{software} de voo e de missão numa única plataforma com impacto
reduzido no desempenho, viabilizando
sistemas confiáveis de criticidade mista em \emph{UAV}s de código aberto.
% mesmo perante grandes superfícies de ataque em que
% garantias ao nível dos componentes, por si só, não são suficientes.

% Palavras-chave do resumo em Português
% \begin{keywords}
% Palavra-chave 1, Palavra-chave 2, Palavra-chave 3, Palavra-chave 4
% \end{keywords}
\keywords{
  UAV \and
  Sistemas de criticidade mista \and
  Hipervisor \and
  Bao \and
  PX4 \and
  Virtualização \and
  SWaP-C
}

%Keywords: UAV, Mixed-Criticality Systems, Hypervisor, Bao, Security, PX4.
% to add an extra black line
