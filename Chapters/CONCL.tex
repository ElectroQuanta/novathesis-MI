\typeout{NT FILE CONCL.tex}%
\chapter{Conclusions and Future Work}
\label{cha:concl}
%
\begin{quote}
\begin{flushright}
``\emph{The only way of discovering the limits of the possible is}\\
  \emph{to venture a little way past them into the impossible.}'' \\
\textbf{-- Arthur C. Clarke}, science fiction writer and inventor
\end{flushright}
\end{quote}

This chapter presents the findings with respect to the research questions, the
main conclusions, the resulting contributions, and prospects for future work.

\section{Findings w.r.t. Research Questions}
We summarize the main findings for each research question (RQ), based on offline
benchmarking and real-flight experiments.

\begin{itemize}
\item
\textbf{RQ1 (Timing under consolidation).}
Offline timing measurements (perf + MiBench \gls{aics}) show small performance
overheads attributable to supervision and that interference effects can be mitigated via
cache coloring. These findings are consistent with prior Bao evaluations~\cite{martins2023shedding}.

\item
\textbf{RQ2 (Fault containment / control authority).}
When faults were confined to the mission \gls{vm} (\gls{cpu}/memory abuse and
user-space resource-starvation patterns) \gls{sspfs} contained failures to
that domain: the PX4 \gls{vm} preserved control authority and the \gls{uav}
remained airborne. The unsupervised \gls{uspfs} baseline exhibited fault
propagation and system-level instability under comparable conditions, causing a
catastrophic \gls{uav} failure.

\item
\textbf{RQ3 (Overheads in offline and application-level metrics).}
Relative to \gls{uspfs}, \gls{sspfs} introduced small overheads: PX4 scheduling
overhead \(\approx 2\%\) (worst case) and camera frame-rate overhead \(\approx
2\%\) (not statistically significant across runs). Offline MiBench \gls{aics}
benchmarks showed modest slowdowns consistent with consolidation costs.

\item
\textbf{RQ4 (Position tracking and system resources during flight).}
Real-flight logs show accurate position tracking, i.e., trajectories closely
follow setpoints outside take-off/landing transients. Average \gls{cpu} overhead was
\(\approx 6\%\) and additional \gls{ram} usage for the flight \gls{vm} was small
(tens of kilobytes), indicating supervision costs remain modest at system scale.

\item
\textbf{RQ5 (Mailbox supervision for shared devices).}
The mailbox-supervision mechanism enabled safe, practical use of firmware-mediated services (e.g., \gls{pcie}-related): legitimate transactions completed with negligible extra latency, and no isolation violations were observed under stress scenarios.
\end{itemize}

\section{Conclusions}
This work designed, implemented, and evaluated a trustworthy open-source
software stack for \gls{uav} applications with an emphasis on security and
safety, aiming to narrow the gap between open-source and commercial offerings.
%
Conventional multi-board stacks provide physical separation but increase weight
and inter-board latency. Pure containerization on single boards improves
integration yet lacks kernel-level separation. Prior single-board consolidations
often depend on closed components or protect only mission software, leaving
critical paths exposed.

To overcome these limitations, we designed, implemented, and evaluated the
\glsxtrfull{sspfs}, which consolidates flight-control and companion workloads on
a single platform under the Bao hypervisor, demonstrating:
(i) robust isolation under realistic fault conditions,
(ii) low performance overhead on commodity hardware, and
(iii) practical mitigation of shared-resource interference (cache coloring) and
shared-firmware paths (mailbox supervision).

In summary, \gls{sspfs} leveraging Bao was shown to be an effective solution for
consolidating mixed-criticality stacks in \glspl{uav}, offering strong isolation
with low overhead. We achieved the main objective of this thesis: the design,
implementation, and validation of a trustworthy, open-source reference software
stack for \gls{uav} applications.

\section{Contributions}
The main contributions of the present work are:
\begin{enumerate}
  \item \textbf{A trustworthy consolidation architecture} (\gls{sspfs}) for co-locating flight control (PX4) and a live video pipeline on Raspberry~Pi~4 + PilotPi under Bao.
  \item \textbf{A supervised mailbox mechanism} for safe sharing of
    firmware-mediated peripherals across \glspl{vm} on the selected platform.
  \item \textbf{A comprehensive evaluation} combining offline MiBench \gls{aics} benchmarks and real-flight experiments, including targeted fault injection to validate isolation.
\end{enumerate}

% \section{Future Work}
% Based on the insights gained from this work, several
% avenues for future research and development can be
% pursued:
% \begin{itemize}
% \item \textbf{Support for different \gls{uavic} platforms:}
% Our consolidation of mixed-criticality stacks relied on a Raspberry~Pi running a
% Linux \gls{gpos}. However, Linux does not provide hard real-time guarantees;
% therefore, the \gls{uavic} platform should natively support the NuttX \gls{rtos}
% required by the PX4 autopilot and should also have a smaller footprint. As noted
% earlier, a promising heterogeneous option is the NXP i.MX~8M Nano~\cite{imx8mn}.
% In the future we plan to port NuttX into this platform,
% matching the software stacks' requirements with the capabilities of the computing elements.
% \item \textbf{A deeper system resources analysis and optimization:}
%   Additional data are needed to thoroughly assess resource usage and reduce the
%   influence of outliers in the current \gls{uavic}; therefore, more tests should
%   be performed under real-flight conditions.
%   The observed increase
%   in \gls{ram} usage attributed to the firmware's mailbox supervision requires
%   deeper investigation. Exploring alternative mechanisms for securely sharing
%   devices across \glspl{vm}, such as VirtIO~\cite{peixoto-virtio-2024}, could provide further insights and
%   potentially lead to more memory-efficient solutions. Bao already provides
%   extensive support for VirtIO~\cite{costa2022virtio,ribeiro2023virtio,rocha_mitigating_2023,peixoto-virtio-2024,baoRepo},
%   which facilitates this approach. Nevertheless, adopting more suitable
%   \gls{uavic} platforms could remove the need for inter-guest device sharing and
%   provide better resource management by default, e.g., by deploying PX4
%   on the NuttX \gls{rtos}.
% \item \textbf{Extended \gls{uav} benchmarking}
% To evaluate the \gls{uav} under real-flight conditions, we analyzed a limited
% set of PX4 tasks, which may not capture full-system behavior. We therefore need
% to collect and analyze additional data from relevant PX4 topics to support
% mission diagnostics and benchmarking, and to repeat these tests after deployment
% on a more suitable \gls{uavic} platform.
% Furthermore, while initial functional
% tests demonstrated isolation, future work should design and execute more
% sophisticated, long-duration attack scenarios and stress tests (e.g., under high
% guest interference), to further validate the trustworthiness and resilience of
% the \gls{sspfs} under adversarial conditions.
% \item \textbf{Advanced Shared Resource Management:}
%   While cache coloring was explored, its effectiveness was limited in certain
%   scenarios due to hardware constraints like \gls{dma} 
%   transactions. Future work should investigate and implement more sophisticated
%   state-of-the-art partitioning mechanisms, such as memory throttling, to
%   further mitigate interference from shared hardware resources like \glspl{llc}
%   and interconnects, which can breach temporal isolation. This is already in the
%   Bao's development roadmap~\cite{martins_et_al:OASIcs:2020:11779}.
% \item \textbf{Optimized Interrupt Virtualization:} The current Bao hypervisor
%   implementation supports Arm GICv2 and GICv3~\cite{baoRepo}, which requires the hypervisor
%   to re-inject interrupts into the guest \glspl{vm}. Future work could focus on
%   updating Bao to fully leverage the GICv4 specification~\cite{arm-gicv4}, which bypasses the
%   hypervisor for guest interrupt delivery, potentially leading to further
%   reductions in interrupt latency and overall complexity of interrupt
%   management. To pursue this venue we must support another \gls{uavic} platform,
%   as Raspberry Pi 4 (GICv2~\cite{rpi4-datasheet,arm-gic400}), or even the NXP
%   i.MX 8M Nano (GICv3~\cite{imx8mn-rm}), do not support GICv4.
% \end{itemize}

\section{Future Work}
Based on the insights gained from this work, several
avenues for future research and development can be
pursued:

\begin{itemize}

\item \textbf{Broader \gls{uavic} support (toward RT flight on smaller footprints).}
Our consolidation was demonstrated on Raspberry~Pi~4 + PilotPi with a Linux
\gls{gpos} flight guest. While effective, Linux does not provide hard real-time
guarantees. A next step is to target \emph{native} PX4 on NuttX (smaller
footprint, tighter timing), on a platform that supports it well (e.g., NXP
i.MX~8M Nano~\cite{imx8mn}). Concretely: (i) complete/maintain NuttX bring-up on
the chosen \gls{uavic}; (ii) map devices for pass-through under Bao; (iii)
re-evaluate timing and footprint versus the current Linux-based \gls{fmu} guest.

\item \textbf{Deeper system-resource accounting and memory optimizations.}
Our CPU/RAM results used PX4 \gls{uorb}-derived proxies. A host-side accounting
pass (hypervisor- and guest-level) would attribute overheads more precisely
(e.g., stage-2 page tables, Bao metadata/state, driver buffers). With that
visibility, we can pursue optimizations in the mailbox path and allocator
behavior. In parallel, we will evaluate alternative sharing mechanisms (e.g.,
VirtIO for suitable devices), which Bao already
supports~\cite{costa2022virtio,ribeiro2023virtio,rocha_mitigating_2023,peixoto-virtio-2024,baoRepo}. On
platforms where PX4 runs on NuttX with direct device ownership, we may avoid
inter-guest device sharing entirely.

\item \textbf{Extended UAV benchmarking and adversarial testing.}
The current flight evaluation analyzed a limited set of PX4 topics. We plan to
expand telemetry coverage (additional control, estimator, and sensor topics) and
to repeat the experiments on a NuttX-based \gls{uavic}. Beyond the validated
fault models, longer-duration and higher-intensity adversarial scenarios (e.g.,
coordinated CPU/memory pressure with I/O bursts)
will stress isolation over time and under interference.

\item \textbf{Advanced shared-resource management.}
Cache coloring mitigated some interference but is constrained by hardware
realities (e.g., \gls{dma} traffic). Next steps include memory-bandwidth
throttling which is already in-line with
Bao's roadmap~\cite{martins_et_al:OASIcs:2020:11779} and can be evaluated using
the same \gls{aics} plus flight stack setup.

\item \textbf{Optimized interrupt virtualization.}
Bao currently targets Arm GICv2/v3~\cite{baoRepo}, which requires hypervisor
re-injection. Supporting GICv4~\cite{arm-gicv4} would enable direct guest
delivery, reducing interrupt latency and hypervisor complexity. This requires a
different \gls{uavic} (Raspberry~Pi~4 is GICv2~\cite{rpi4-datasheet,arm-gic400};
i.MX~8M Nano is GICv3~\cite{imx8mn-rm}) or future \glspl{soc} with GICv4.

\end{itemize}



%%% Local Variables:
%%% mode: LaTeX
%%% TeX-master: "../template"
%%% End:
