\typeout{NT FILE CONCL.tex}%
\chapter{Conclusions and Future Work}
\label{cha:concl}
%
\begin{quote}
\begin{flushright}
``\emph{The only way of discovering the limits of the possible is}\\
  \emph{to venture a little way past them into the impossible.}'' \\
\textbf{-- Arthur C. Clarke}, science fiction writer and inventor
\end{flushright}
\end{quote}

In this chapter we outline the main conclusions about the present work, as well
as the prospects for future work.

\section{Conclusions}
The work presented focused on the development of a
trustworthy open-source software stack for \gls{uav} applications, with a paramount emphasis on security and safety. This effort aimed to bridge the gap between existing open-source and commercial solutions by contributing to the widespread adoption of secure and safe features in \gls{uav} technology.
A key finding was that conventional multi-platform flight stacks, while
addressing mixed-criticality at the hardware level, undesirably increase the
\gls{uav}'s weight and footprint. Furthermore, such systems typically lack isolation
guarantees, meaning that a compromise in a non-critical component, like video
surveillance, could propagate to the critical flight controller, potentially
leading to a catastrophic failure.

To overcome these limitations, the proposed \glsxtrfull{sspfs} integrates the flight controller and companion computer
functionalities onto a single hardware platform, leveraging the Bao
hypervisor. The unsupervised counterpart solution -- \glsxtrfull{uspfs} -- was
devised and tested to assess the impact of the supervision layer.
The \gls{uavic} platform selected
consisted of a Raspberry Pi 4 + PilotPi shield, supporting the open source
PX4 autopilot flight stack. The hardware mapping revealed the need to support a
supervised mailbox access to handle firmware's mailbox transactions in the
Raspberry Pi as a means to share some critical devices in the \gls{uavic}
platform, such as the \gls{pcie} bus.

The evaluation of the \gls{sspfs} demonstrated several critical
advantages:
\begin{itemize}
\item \textbf{Enhanced Isolation}: The functional tests clearly
  showed that the Bao hypervisor provides strong isolation guarantees between
  mixed-criticality systems. When a malicious kernel module or a resource
  exhaustion application was introduced into the non-critical Companion
  \gls{vm}, it caused that \gls{vm} to crash, but the critical PX4 \gls{fmu}
  \gls{vm} remained operational and the UAV stayed airborne. In contrast, the
  unsupervised system suffered a total collapse under similar attacks.
\item \textbf{Extensive benchmarking}: The supervised solution was benchmarked
  using the industrial standard MiBench suite, showing a very low performance
  degradation compared to the unsupervised one. The performance degradation can
  be significantly higher under interference, but cache partitioning via page
  coloring consistently reduced the performance degradation for the \gls{fmu}
  \gls{vm}, mitigating this effect.
\item
\textbf{Effective Mailbox Supervision}: A
  customized mechanism was devised and validated to manage shared access to the
  firmware mailbox between different guests under Bao's supervision. Although
  not ideal and specific to the \gls{uavic} platform selected (Raspberry Pi 4),
  this mechanism ensured seamless communication without compromising isolation.
\item \textbf{Minimal Performance Overhead}: The benchmarking revealed negligible
  performance degradation due to the introduction of the Bao hypervisor and the
  custom mailbox driver patch required for Raspberry Pi firmware
  interfacing. For PX4 tasks, the scheduling overhead was found to be very low,
  with a worst-case increase of only 2\%. The camera's frame rate in the
  Companion \gls{vm} also showed no statistically significant overhead, with
  most runs exhibiting degradation close to 0\% and a maximum of less than
  2\%. In some instances, the SSPFS even outperformed the unsupervised system
  due to the static partitioning of resources.
\item \textbf{Real-Flight Validation}: The automated missions in real-flight scenarios confirmed that the
  \gls{sspfs} system maintains accurate position tracking, unaffected by the
  supervision layer. The \gls{cpu} load increase was small (an average of 6\%),
  which is in line with the benchmarkings performed on the Bao hypervisor.
  While an increase in \gls{ram} usage was observed (99 KB), it was
  considered negligible given the available memory for the FMU \gls{vm} (144
  MB).
\end{itemize}

In summary, the \gls{sspfs} system, leveraging the Bao hypervisor, has been
demonstrated as a superior solution for consolidating mixed-criticality software
stacks in \glspl{uav}, offering robust isolation guarantees with low performance
overhead. We therefore achieved the main objective of this thesis: the design,
implementation, and validation of a trustworthy, open-source reference software
stack for \gls{uav} applications.

\section{Future Work}
Based on the insights gained from this work, several
avenues for future research and development can be
pursued:
\begin{itemize}
\item \textbf{Support for different \gls{uavic} platforms:}
Our consolidation of mixed-criticality stacks relied on a Raspberry~Pi running a
Linux \gls{gpos}. However, Linux does not provide hard real-time guarantees;
therefore, the \gls{uavic} platform should natively support the NuttX \gls{rtos}
required by the PX4 autopilot and should also have a smaller footprint. A
promising heterogeneous platform is the NXP i.MX~8M Nano~\cite{imx8mn}, which
integrates an Arm Cortex-M7 (well suited to NuttX) and a quad-core Arm
Cortex-A53 (well suited to a Linux \gls{os}). Porting NuttX to this platform
remains future work.
\item \textbf{A deeper system resources analysis and optimization:}
  Additional data are needed to thoroughly assess resource usage and reduce the
  influence of outliers in the current \gls{uavic}; therefore, more tests should
  be performed under real-flight conditions.
  The observed increase
  in \gls{ram} usage attributed to the firmware's mailbox supervision requires
  deeper investigation. Exploring alternative mechanisms for securely sharing
  devices across \glspl{vm}, such as VirtIO~\cite{peixoto-virtio-2024}, could provide further insights and
  potentially lead to more memory-efficient solutions. Bao already provides
  extensive support for
  VirtIO~\cite{costa2022virtio,ribeiro2023virtio,peixoto-virtio-2024,baoRepo},
  which facilitates this approach. Nevertheless, adopting more suitable
  \gls{uavic} platforms could remove the need for inter-guest device sharing and
  provide better resource management by default, e.g., by deploying PX4
  on the NuttX \gls{rtos}.
\item \textbf{Extended \gls{uav} benchmarking}
To evaluate the \gls{uav} under real-flight conditions, we analyzed a limited
set of PX4 tasks, which may not capture full-system behavior. We therefore need
to collect and analyze additional data from relevant PX4 topics to support
mission diagnostics and benchmarking, and to repeat these tests after deployment
on a more suitable \gls{uavic} platform.
Furthermore, while initial functional
tests demonstrated isolation, future work should design and execute more
sophisticated, long-duration attack scenarios and stress tests (e.g., under high
guest interference), to further validate the trustworthiness and resilience of
the \gls{sspfs} under adversarial conditions.
\item \textbf{Advanced Shared Resource Management:}
  While cache coloring was explored, its effectiveness was limited in certain
  scenarios due to hardware constraints like \gls{dma} 
  transactions. Future work should investigate and implement more sophisticated
  state-of-the-art partitioning mechanisms, such as memory throttling, to
  further mitigate interference from shared hardware resources like \glspl{llc}
  and interconnects, which can breach temporal isolation. This is already in the
  Bao's development roadmap~\cite{martins_et_al:OASIcs:2020:11779}.
\item \textbf{Optimized Interrupt Virtualization:} The current Bao hypervisor
  implementation supports Arm GICv2 and GICv3~\cite{baoRepo}, which requires the hypervisor
  to re-inject interrupts into the guest \glspl{vm}. Future work could focus on
  updating Bao to fully leverage the GICv4 specification\cite{arm-gicv4}, which bypasses the
  hypervisor for guest interrupt delivery, potentially leading to further
  reductions in interrupt latency and overall complexity of interrupt
  management. To pursue this venue we must support another \gls{uavic} platform,
  as Raspberry Pi 4 (GICv2~\cite{rpi4-datasheet,arm-gic400}), or even the NXP
  i.MX 8M Nano (GICv3~\cite{imx8mn-rm}), do not support GICv4.
\end{itemize}


%%% Local Variables:
%%% mode: LaTeX
%%% TeX-master: "../template"
%%% End:
