\typeout{NT FILE CONCL.tex}%
\chapter{Conclusions and Future Work}
\label{cha:concl}
%
\begin{quote}
\begin{flushright}
``\emph{The only way of discovering the limits of the possible is}\\
  \emph{to venture a little way past them into the impossible.}'' \\
\textbf{-- Arthur C. Clarke}, science fiction writer and inventor
\end{flushright}
\end{quote}

In this chapter the main conclusions about the present work are outlined, as
well as the prospects for future work.

\section{Conclusions}
The work presented focused on the development of a
trustworthy open-source software stack for \gls{uav} applications, with a paramount emphasis on security and safety. This effort aimed to bridge the gap between existing open-source and commercial solutions by contributing to the widespread adoption of secure and safe features in \gls{uav} technology.
A key finding was that conventional multi-platform flight stacks, while
addressing mixed-criticality at the hardware level, undesirably increase the
\gls{uav}'s weight and footprint. Furthermore, such systems typically lack isolation
guarantees, meaning that a compromise in a non-critical component, like video
surveillance, could propagate to the critical flight controller, potentially
leading to catastrophic failure.

To overcome these limitations, the proposed \glsxtrfull{sspfs} integrates the flight controller and companion computer
functionalities onto a single hardware platform, leveraging the Bao
hypervisor. The unsupervised counterpart solution -- \glsxtrfull{uspfs} -- was devised and tested to
assess the impact of the supervision layer. The \gls{uavic} platform selected
consisted of a Raspberry Pi 4 + PilotPi shield, supporting the open source
PX4 autopilot flight stack. The hardware mapping revealed the need to support a
supervised mailbox access to handle firmware's mailbox transactions in the
Raspberry Pi as a means to share some critical devices in the \gls{uavic}
platform, such as the \gls{pcie} bus.

The evaluation of the \gls{sspfs} demonstrated several critical
advantages:
\begin{itemize}
\item \textbf{Enhanced Isolation}: The functional tests clearly
  showed that the Bao hypervisor provides strong isolation guarantees between
  mixed-criticality systems. When a malicious kernel module or a resource
  exhaustion application was introduced into the non-critical Companion
  \gls{vm}, it caused that VM to crash, but the critical PX4 \gls{fmu} \gls{vm}
  remained operational and the UAV stayed airborne. In contrast, the
  unsupervised system suffered a total collapse under similar attacks.
\item \textbf{Extensive benchmarking}: The supervised solution was benchmarked
  using the industrial standard MiBench suite, showing a very low performance
  degradation compared to the unsupervised one. The performance degradation can
  be significantly higher under interference, but cache partitioning via page
  coloring consistently reduced the performance degradation for the \gls{fmu}
  \gls{vm}, mitigating this effect.
\item
\textbf{Effective Mailbox Supervision}: A
  customized mechanism was devised and validated to manage shared access to the
  firmware mailbox between different guests under Bao's supervision. Although
  not ideal and specific to the \gls{uavic} platform selected (Raspberry Pi 4),
  this mechanism ensured seamless communication without compromising isolation.
\item \textbf{Minimal Performance Overhead}: The benchmarking revealed negligible
  performance degradation due to the introduction of the Bao hypervisor and the
  custom mailbox driver patch required for Raspberry Pi firmware
  interfacing. For PX4 tasks, the scheduling overhead was found to be very low,
  with a worst-case increase of only 2\%. The camera's frame rate in the
  Companion \gls{vm} also showed no statistically significant overhead, with
  most runs exhibiting degradation close to 0\% and a maximum of less than
  2\%. In some instances, the SSPFS even outperformed the unsupervised system
  due to the static partitioning of resources.
\item \textbf{Real-Flight Validation}: The automated missions in real-flight scenarios confirmed that the
  SSPFS system maintains accurate position tracking, unaffected by the
  supervision layer. The \gls{cpu} load increase was small (an average of 6\%),
  which is in line with the benchmarkings performed on the Bao hypervisor.
  While an increase in RAM usage was observed (99 KB), it was
  considered negligible given the available memory for the FMU \gls{vm} (144
  MB).
\end{itemize}

In summary, the \gls{sspfs} system, leveraging the Bao hypervisor, has been
demonstrated as a superior solution for consolidating mixed-criticality software
stacks in \glspl{uav}, offering robust isolation guarantees with low performance
overhead.

\section{Future Work}
Based on the insights gained from this work, several
avenues for future research and development can be
pursued:
\begin{itemize}
\item \textbf{Support for different \gls{uavic} platforms:} the consolidation of
  the mixed-criticality stacks relied on a Raspberry Pi with the Linux
  \gls{gpos}.
  Ideally, the \gls{uavic} platform should support the NuttX \gls{rtos},
  required by the PX4 autopilot, and have a smaller footprint.
\item \textbf{Advanced Shared Resource Management:}
  While cache coloring was explored, its effectiveness was limited in certain
  scenarios due to hardware constraints like \gls{dma} 
  transactions. Future work should investigate and implement more sophisticated
  state-of-the-art partitioning mechanisms, such as memory throttling, to
  further mitigate interference from shared hardware resources like \glspl{llc}
  and interconnects, which can breach temporal isolation.
\item \textbf{Optimized Interrupt Virtualization:} The current Bao hypervisor
  implementation supports Arm GICv2 and GICv3, which necessitates the hypervisor
  to re-inject interrupts into guest \glspl{vm}. Future work could focus on updating
  Bao to fully leverage the GICv4 specification, which bypasses the hypervisor
  for guest interrupt delivery, potentially leading to further reductions in
  interrupt latency and overall complexity of interrupt management.
\item \textbf{Detailed \gls{ram} Usage Analysis and Optimization:} The observed increase
  in \gls{ram} usage attributed to the firmware's mailbox supervision warrants deeper
  investigation. Exploring alternative mechanisms for securely sharing devices
  across \glspl{vm} could provide further insights and potentially lead to more
  memory-efficient solutions.
\item \textbf{Extended Functional and Performance
    Testing:} While initial functional tests demonstrate isolation, future work
  could involve designing and executing more sophisticated and long-duration
  attack scenarios and stress tests. This would further validate the
  trustworthiness and resilience of the \gls{sspfs}  under diverse and sustained
  malicious activities.
\end{itemize}


%%% Local Variables:
%%% mode: LaTeX
%%% TeX-master: "../template"
%%% End:
